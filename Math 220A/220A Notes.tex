\documentclass[x11names,reqno,14pt]{extarticle}
% Choomno Moos
% Portland State University
% Choom@pdx.edu


%% stupid experiment %%
%%%%%%%%%%%%% PACKAGES %%%%%%%%%%%%%

%%%% SYMBOLS AND MATH %%%%
\let\oldvec\vec
\usepackage{authblk}	% author block customization
\usepackage{microtype}	% makes stuff look real nice
\usepackage{amssymb} 	% math symbols
\usepackage{siunitx} 	% for SI units, and the degree symbol
\usepackage{mathrsfs}	% provides script fonts like mathscr
\usepackage{mathtools}	% extension to amsmath, also loads amsmath
\usepackage{esint}		% extended set of integrals
\mathtoolsset{showonlyrefs} % equation numbers only shown when referenced
\usepackage{amsthm}		% theorem environments
\usepackage{relsize}	%font size commands
\usepackage{bm}			% provides bold math
\usepackage{bbm}		% for blackboard bold 1

%%%% FIGURES %%%%
\usepackage{graphicx} % for including pictures
\usepackage{float} % allows [H] option on figures, so that they appear where they are typed in code
\usepackage{caption}
\usepackage{hyperref}
%\usepackage{titling}
\usepackage{tikz} % for drawing
\usetikzlibrary{shapes,arrows,chains,positioning,cd,decorations.pathreplacing,decorations.markings,hobby,knots,braids}
\usepackage{subcaption}	% subfigure environment in figures

%%%% MISC %%%%
\usepackage{enumitem} % for lists and itemizations
\setlist[enumerate]{leftmargin=*,label=\bf \arabic*.}

\usepackage{multicol}
\usepackage{multirow}
\usepackage{url}
\usepackage[symbol]{footmisc}
\renewcommand{\thefootnote}{\fnsymbol{footnote}}
\usepackage{lastpage} % provides the total number of pages for the "X of LastPage" page numbering
\usepackage{fancyhdr}
\usepackage{manfnt}
\usepackage{nicefrac}
%\usepackage{fontspec}
%\usepackage{polyglossia}
%\setmainlanguage{english}
%\setotherlanguages{khmer}
%\newfontfamily\khmerfont[Script=Khmer]{Khmer Busra}

%%% Khmer script commands for math %%%
%\newcommand{\ka}{\text{\textkhmer{ក}}}
%\newcommand{\ko}{\text{\textkhmer{ត}}}
%\newcommand{\kha}{\text{\textkhmer{ខ}}}

%\usepackage[
%backend=biber,
% numeric
%style=numeric,
% APA
%bibstyle=apa,
%citestyle=authoryear,
%]{biblatex}

\usepackage[explicit]{titlesec}
%%%%%%%% SOME CODE FOR REDECLARING %%%%%%%%%%

\makeatletter
\newcommand\RedeclareMathOperator{%
	\@ifstar{\def\rmo@s{m}\rmo@redeclare}{\def\rmo@s{o}\rmo@redeclare}%
}
% this is taken from \renew@command
\newcommand\rmo@redeclare[2]{%
	\begingroup \escapechar\m@ne\xdef\@gtempa{{\string#1}}\endgroup
	\expandafter\@ifundefined\@gtempa
	{\@latex@error{\noexpand#1undefined}\@ehc}%
	\relax
	\expandafter\rmo@declmathop\rmo@s{#1}{#2}}
% This is just \@declmathop without \@ifdefinable
\newcommand\rmo@declmathop[3]{%
	\DeclareRobustCommand{#2}{\qopname\newmcodes@#1{#3}}%
}
\@onlypreamble\RedeclareMathOperator
\makeatother

\makeatletter
\newcommand*{\relrelbarsep}{.386ex}
\newcommand*{\relrelbar}{%
	\mathrel{%
		\mathpalette\@relrelbar\relrelbarsep
	}%
}
\newcommand*{\@relrelbar}[2]{%
	\raise#2\hbox to 0pt{$\m@th#1\relbar$\hss}%
	\lower#2\hbox{$\m@th#1\relbar$}%
}
\providecommand*{\rightrightarrowsfill@}{%
	\arrowfill@\relrelbar\relrelbar\rightrightarrows
}
\providecommand*{\leftleftarrowsfill@}{%
	\arrowfill@\leftleftarrows\relrelbar\relrelbar
}
\providecommand*{\xrightrightarrows}[2][]{%
	\ext@arrow 0359\rightrightarrowsfill@{#1}{#2}%
}
\providecommand*{\xleftleftarrows}[2][]{%
	\ext@arrow 3095\leftleftarrowsfill@{#1}{#2}%
}
\makeatother

%%%%%%%% NEW COMMANDS %%%%%%%%%%

% settings
\newcommand{\N}{\mathbb{N}}                     	% Natural numbers
\newcommand{\Z}{\mathbb{Z}}                     	% Integers
\newcommand{\Q}{\mathbb{Q}}                     	% Rationals
\newcommand{\R}{\mathbb{R}}                     	% Reals
\newcommand{\C}{\mathbb{C}}                     	% Complex numbers
\newcommand{\K}{\mathbb{K}}							% Scalars
\newcommand{\F}{\mathbb{F}}                     	% Arbitrary Field
\newcommand{\E}{\mathbb{E}}                     	% Euclidean topological space
\renewcommand{\H}{{\mathbb{H}}}                   	% Quaternions / Half space
\newcommand{\RP}{{\mathbb{RP}}}                       % Real projective space
\newcommand{\CP}{{\mathbb{CP}}}                       % Complex projective space
\newcommand{\Mat}{{\mathrm{Mat}}}						% Matrix ring
\newcommand{\M}{\mathcal{M}}
\newcommand{\GL}{{\mathrm{GL}}}
\newcommand{\SL}{{\mathrm{SL}}}

\newcommand{\tgl}{\mathfrak{gl}}
\newcommand{\tsl}{\mathfrak{sl}}                  % Lie algebras; i.e., tangent space of SO/SL/SU
\newcommand{\tso}{\mathfrak{so}}
\newcommand{\tsu}{\mathfrak{sl}}


% typography
\newcommand{\noi}{\noindent}						% Removes indent
\newcommand{\tbf}[1]{\textbf{#1}}					% Boldface
\newcommand{\mc}[1]{\mathcal{#1}}               	% Calligraphic
\newcommand{\ms}[1]{\mathscr{#1}}               	% Script
\newcommand{\mbb}[1]{\mathbb{#1}}               	% Blackboard bold


% (in)equalities
\newcommand{\eqdef}{\overset{\mathrm{def}}{=}}		% Definition equals
\newcommand{\sub}{\subseteq}						% Changes default symbol from proper to improper
\newcommand{\psub}{\subset}						% Preferred proper subset symbol

% Categories
\newcommand{\catname}[1]{{\text{\sffamily {#1}}}}

\newcommand{\Cat}{{\catname{C}}}
\newcommand{\cat}[1]{{\catname{\ifblank{#1}{C}{#1}}}}
\newcommand{\CAT}{{\catname{Cat}}}
\newcommand{\Set}{{\catname{Set}}}

\newcommand{\Top}{{\catname{Top}}}
\newcommand{\Met}{{\catname{Met}}}
\newcommand{\PL}{{\catname{PL}}}
\newcommand{\Man}{{\catname{Man}}}
\newcommand{\Diff}{{\catname{Diff}}}

\newcommand{\Grp}{{\catname{Grp}}}
\newcommand{\Grpd}{{\catname{Grpd}}}
\newcommand{\Ab}{{\catname{Ab}}}
\newcommand{\Ring}{{\catname{Ring}}}
\newcommand{\CRing}{{\catname{CRing}}}
\newcommand{\Mod}{{\mhyphen\catname{Mod}}}
\newcommand{\Alg}{{\mhyphen\catname{Alg}}}
\newcommand{\Field}{{\catname{Field}}}
\newcommand{\Vect}{{\catname{Vect}}}
\newcommand{\Hilb}{{\catname{Hilb}}}
\newcommand{\Ch}{{\catname{Ch}}}

\newcommand{\Hom}{{\mathrm{Hom}}}
\newcommand{\End}{{\mathrm{End}}}
\newcommand{\Aut}{{\mathrm{Aut}}}
\newcommand{\Obj}{{\mathrm{Obj}}}
\newcommand{\op}{{\mathrm{op}}}

% Norms, inner products
\delimitershortfall=-1sp
\newcommand{\widecdot}{\, \cdot \,}
\newcommand\emptyarg{{}\cdot{}}
\DeclarePairedDelimiterX{\norm}[1]{\Vert}{\Vert}{\ifblank{#1}{\emptyarg}{#1}}
\DeclarePairedDelimiterX{\abs}[1]\vert\vert{\ifblank{#1}{\emptyarg}{#1}}
\DeclarePairedDelimiterX\inn[1]\langle\rangle{\ifblank{#1}{\emptyarg,\emptyarg}{#1}}
\DeclarePairedDelimiterX\cur[1]\{\}{\ifblank{#1}{\emptyarg,\emptyarg}{#1}}
\DeclarePairedDelimiterX\pa[1](){\ifblank{#1}{\emptyarg}{#1}}
\DeclarePairedDelimiterX\brak[1][]{\ifblank{#1}{\emptyarg}{#1}}
\DeclarePairedDelimiterX{\an}[1]\langle\rangle{\ifblank{#1}{\emptyarg}{#1}}
\DeclarePairedDelimiterX{\bra}[1]\langle\vert{\ifblank{#1}{\emptyarg}{#1}}
\DeclarePairedDelimiterX{\ket}[1]\vert\rangle{\ifblank{#1}{\emptyarg}{#1}}

% mathmode text operators
\RedeclareMathOperator{\Re}{\operatorname{Re}}		% Real part
\RedeclareMathOperator{\Im}{\operatorname{Im}}		% Imaginary part
\DeclareMathOperator{\Stab}{\mathrm{Stab}}
\DeclareMathOperator{\Orb}{\mathrm{Orb}}
\DeclareMathOperator{\Id}{\mathrm{Id}}
\DeclareMathOperator{\vspan}{\mathrm{span}}			% Vector span
\DeclareMathOperator{\tr}{\mathrm{tr}}
\DeclareMathOperator{\adj}{\mathrm{adj}}
\DeclareMathOperator{\diag}{\mathrm{diag}}
\DeclareMathOperator{\eq}{\mathrm{eq}}
\DeclareMathOperator{\coeq}{\mathrm{coeq}}
\DeclareMathOperator{\coker}{\mathrm{coker}}
\DeclareMathOperator{\dom}{\mathrm{dom}}
\DeclareMathOperator{\cod}{\mathrm{codom}}
\DeclareMathOperator{\im}{\mathrm{im}}
\DeclareMathOperator{\Dim}{\mathrm{dim}}
\DeclareMathOperator{\codim}{\mathrm{codim}}
\DeclareMathOperator{\Sym}{\mathrm{Sym}}
\DeclareMathOperator{\lcm}{\mathrm{lcm}}
\DeclareMathOperator{\Inn}{\mathrm{Inn}}
\DeclareMathOperator{\sgn}{sgn}						% sgn operator
\DeclareMathOperator{\intr}{\text{int}}             % Interior
\DeclareMathOperator{\co}{\mathrm{co}}				% dual/convex Hull
\DeclareMathOperator{\Ann}{\mathrm{Ann}}
\DeclareMathOperator{\Tor}{\mathrm{Tor}}


% misc symbols
\newcommand{\divides}{\big\lvert}
\newcommand{\grad}{\nabla}
\newcommand{\veps}{\varepsilon}						% Preferred epsilon
\newcommand{\vphi}{\varphi}
\newcommand{\del}{\partial}							% Differential/Boundary
\renewcommand{\emptyset}{\text{\O}}					% Traditional emptyset symbol
\newcommand{\tril}{\triangleleft}					% Quandle operation
\newcommand{\nabt}{\widetilde{\nabla}}				% Contravariant derivative
\newcommand{\later}{$\textcolor{red}{\blacksquare}$}% Laziness indicator

% misc
\mathchardef\mhyphen="2D							% mathomode hyphen
\renewcommand{\mod}[1]{\ (\mathrm{mod}\ #1)}
\renewcommand{\bar}[1]{\overline{#1}}				% Closure/conjugate
\renewcommand\qedsymbol{$\blacksquare$} 			% Changes default qed in proof environment
%%%%% raised chi
\DeclareRobustCommand{\rchi}{{\mathpalette\irchi\relax}}
\newcommand{\irchi}[2]{\raisebox{\depth}{$#1\chi$}}
\newcommand\concat{+\kern-1.3ex+\kern0.8ex}

% Arrows
\newcommand{\weak}{\rightharpoonup}					% Weak convergence
\newcommand{\weakstar}{\overset{*}{\rightharpoonup}}% Weak-star convergence
\newcommand{\inclusion}{\hookrightarrow}			% Inclusion/injective map
\renewcommand{\natural}{\twoheadrightarrow}				% Natural map

% Environments
\theoremstyle{plain}
\newtheorem{thm}{Theorem}[section]
%\newtheorem{lem}[thm]{Lemma}
\newtheorem{lem}{Lemma}
\newtheorem*{lems}{Lemma}
\newtheorem{cor}[thm]{Corollary}
\newtheorem{prop}{Proposition}
\newtheorem*{claim}{Claim}
\newtheorem*{cors}{Corollary}
\newtheorem*{props}{Proposition}
\newtheorem*{conj}{Conjecture}

\theoremstyle{definition}
\newtheorem{defn}{Definition}[section]
\newtheorem*{defns}{Definition}
\newtheorem{exm}{Example}[section]
\newtheorem{exer}{Exercise}[section]

\theoremstyle{remark}
\newtheorem*{rem}{Remark}

\newtheorem*{solnx}{Solution}
\newenvironment{soln}
    {\pushQED{\qed}\renewcommand{\qedsymbol}{$\Diamond$}\solnx}
    {\popQED\endsolnx}%

% Macros
\newcommand{\restr}[1]{_{\mkern 1mu \vrule height 2ex\mkern2mu #1}}
\newcommand{\Upushout}[5]{
    \begin{tikzcd}[ampersand replacement = \&]
    \&#2\ar[rd,"\iota_{#2}"]\ar[rrd,bend left,"f"]\&\&\\
    #1\ar[ur,"#4"]\ar[dr,"#5"]\&\&#2\oplus_{#1} #3\ar[r,dashed,"\vphi"]\&Z\\
    \&#3\ar[ur,"\iota_{#3}"']\ar[rru,bend right,"g"']\&\&
    \end{tikzcd}
}
\newcommand{\exactshort}[5]{
		\begin{tikzcd}[ampersand replacement = \&]
			0\ar[r]\&#1\ar[r,"#2"]\& #3 \ar[r,"#4"]\& #5 \ar[r]\&0
		\end{tikzcd}
}
\newcommand{\product}[6]{
		\begin{tikzcd}[ampersand replacement = \&]
			#1 \& #2 \ar[l,"#4"'] \\
			#3 \ar[u,"#5"] \ar[ur,"#6"']
		\end{tikzcd}
}
\newcommand{\coproduct}[6]{
		\begin{tikzcd}[ampersand replacement = \&]
			#1 \ar[r,"#4"] \ar[d,"#5"'] \& #2 \ar[dl,"#6"] \\
			#3
		\end{tikzcd}
}
%%%%%%%%%%%% PAGE FORMATTING %%%%%%%%%

\usepackage{geometry}
    \geometry{
		left=15mm,
		right=15mm,
		top=15mm,
		bottom=15mm	
		}

\usepackage{color} % to do: change to xcolor
\usepackage{listings}
\lstset{
    basicstyle=\ttfamily,columns=fullflexible,keepspaces=true
}
\usepackage{setspace}
\usepackage{setspace}
\usepackage{mdframed}
\usepackage{booktabs}
\usepackage[document]{ragged2e}

\pagestyle{fancy}{
	\fancyhead[L]{Fall 2022}
	\fancyhead[C]{220A - Groups}
	\fancyhead[R]{John White}
  
  \fancyfoot[R]{\footnotesize Page \thepage \ of \pageref{LastPage}}
	\fancyfoot[C]{}
	}
\fancypagestyle{firststyle}{
     \fancyhead[L]{}
     \fancyhead[R]{}
     \fancyhead[C]{}
     \renewcommand{\headrulewidth}{0pt}
	\fancyfoot[R]{\footnotesize Page \thepage \ of \pageref{LastPage}}
}
\newcommand{\pmat}[4]{\begin{pmatrix} #1 & #2 \\ #3 & #4 \end{pmatrix}}

\title{220A - Groups}
\author{John White}
\date{Fall 2022}


\begin{document}

\section*{Lecture 1}

Algebra consists, generally speaking, of Groups, Rings, and Fields. These are covered, respectively, in 220A, 220B, and 220C. 


We will begin with the minimal possible axioms, and go on to more extensive axioms. If you start with very few axioms, there are too many examples. If you go too far with axioms, you have too many examples. Group theory is all about binary operations. 

\defn 

Let $S$ be a set. A \underline{binary operation} on $S$ is a function $f:S\times S \to S$. The image of $(a, b) \in S \times S$ under $f$ is often denoted by $a * b$, or simply $ab$. 

\defn
 
A binary operation is called \underline{associative} if, for each $a, b, c \in S$, we have 
\[
(ab)c = a(bc)
\]

\defn

An element $e \in S$ is an \underline{identity} if $ex = xe = x$ for all $x \in S$

\defn 
A \underline{monoid} is a set $S$ equipped with a binary operation which is associative, and which admits an identity. 

\section*{Lecture 2}

Given $x_1, \dots, x_n \in M$, with $M$ a monoid, we may inductively define
\[
\prod_{i=1}^nx_i :=(\prod_{i=1}^{n - 1}x_i)x_n
\]

Fact: In a commutative monoid, given any bijection $\psi:\{1,\dots,n\}\to\{1,\dots,n\}$, 
\[
\prod_{i=1}^nx_{\psi(i)} = \prod_{i=1}^nx_i
\]

\defn

A \underline{group} $G$ is a monoid such that $\forall x \in G$, $\exists y \in G$ such that $xy = yx = e$. Here, $y$ is called an inverse of $x$. Note, we have not yet proved that $y$ is unique. 

\claim

For any $x \in G$, for $G$ a group, $x^{-1}$ is unique. 

\proof

Let $y, y'$ be inverses to $x$. Then 
\[
y = ey = (y'x)y = y'(xy) = y'e = y'
\]

\qed

\exm

Let $G$ be a group. Let $S$ be a nonempty set. 

The set of maps from $S$ to $G$, denoted $M(S, G)$, is a group as follows. 

Given $f, g \in M(S, G)$, define $fg(x) = f(x)g(x)$. 

Identity: $e(x) = e \forall x \in S$. 

Inverses: For an $f \in M(S, G)$, $f^{-1}$ is the function such that $(f^{-1})(x) = f(x)^{-1}$. 

\exm

Given a nonempty set $S$, define $\operatorname{Perm}_S$ as the set of all bijective maps from $S$ to $S$. This forms a group under composition. 

When $|S|$ is a finite set of size $n$, we identify $\operatorname{Perm}_S$ with $S_n$. 

\exm

Let $k$ be a field, and $V$ a vector space over $k$. Then $GL(V)$ and $GL(n, k)$ are groups. 

\exm

$\Q$ is a group. $\Q^\times$, which is $\Q \backslash\{0\}$, is a group under multiplication. 

This works for any field. 

\exm

$\Z$ is an additive group. $\{1, -1\}$ forms a multiplicative group. 

\defn

A group $G$ is called \underline{cyclic} if $\exists a \in G$ such that every $x \in G$ is of the form $a^n$ for some $n \in \Z$. Such an $a$ is called a generator. 

So $\Z$ is (additively) cyclic with generators $\pm 1$.

\defn

The \underline{order} of $G$ is $|G|$ if $G$ is finite, and $\infty$ otherwise. 

\defn

Let $N \in \N$. Then the $N$th \underline{roots of unity} form a multiplicative subgroup of $\C$. It is cyclic, and generated by $e^{\frac{2\pi i k}{N}}$. 

An element $e^{\frac{2\pi ik}{N}}$ is called a primitive if it is a generator. 

For example, for $N = 4$, our group would be $\{\pm 1, \pm i\}$. The primitives would be $\pm i$. 

$\{e^{\frac{2\pi i k}{N}}\}$ is a generator of the $N$th roots of unity if and only if $k$ is coprime to $N$. 

\subsection*{}

Let $P$ be a regular polygon with $N$ sides. Let $\Aut(P) = \{$ automorphisms of $P\}$. The automorphisms of $P$ form a group, and it is generated by a reflection and a rotation. 

\defn

A subset $S \subseteq G$ generates $G$ if every $x \in G$ can be written $x = s_1^{a_1}\cdots s_n^{a_n}$ with $s_i \in S$ for each $i$. 

Let $\sigma, \tau \in \Aut(P)$. We can see that $\tau\sigma\tau^{-1} = \tau\sigma\tau = \sigma^{-1} = \sigma^{n - 1}$. 

If $G$ is generated by $S$, any expression of the form $s_1^{a_1}\cdots s_n^{a_n}$, with $s_i \in S$ and $a_i \in \Z$ is called a word in the generators. 

If $W$ is a set of words in $S$, then 
\[
\langle S \mid \rangle
\]
is the free group on $S$ quotiented by the normal subgroup generated by all elements of $W$. This is called a group presentation of the resulting group. 

So $\Aut(P) = \langle \sigma, \tau, \sigma^n = \tau^2 = (\tau\sigma)^n = e\rangle$

\section*{Lecture 3}

\defn

A monoid $M$ is called \underline{commutative/abelian} if  for all $x, y \in M$, $xy = yx$. 

$N\subset M$ is called a \underline{submonoid} iff $x,y \in N\implies xy \in N$, and $e \in N$. 

A function $f:M\to N$ is a \underline{homomorphism of monoids} if $f(xy) = f(x)f(y)$ for all $x, y \in M$. 

\defn

A \underline{commutative/abelian group} and a \underline{homomorphism of groups} are defined exactly as above. A \underline{subgroup} is defined similarly as above, but with the added requirement that $x\in N$ implies $x^{-1}\in N$. 

\exm

Let $\mu_N$ denote the $N$th roots of unity, generated by $e^{\frac{2\pi i}{N}}.$ We have a morphism $f:\Z\to\mu_N$ given by $f(k) = e^{\frac{2\pi ik}{N}}$. 

\exm

We have a $\pi:D_n\to\mu_N$ given by $\pi(\sigma)= -\pi(\tau) = -1$, $\pi(e) = e$. 

\defn

Given $\phi:G\to H$ a homomorphism of groups, the \underline{kernel} of $\phi$, denoted $\ker(\phi)$, is defined by
\[
\ker(\phi) := \{g \in G \mid \phi(g) = e \}
\]
Similarly, we define the \underline{image}, $\Im(\phi)$, by 
\[
\Im(\phi) = \{h \in H \mid \exists g \in G, \phi(g) = h \}
\]

\defn

If $N$ is a subgroup of $G$, then we say that $N$ is \underline{normal} if for all $g \in G$ and $n \in N$, we have $gng^{-1} \in N$. 

\lem Given a morphism $\pi: G\to H$, then 
\begin{enumerate}
\item $\pi$ is injective if and only if $\ker(\pi) = \{e\}$. In this case, we say $\pi:G\hookrightarrow H$. 
\item $\pi$ is surjective if and only if $\Im(a) = H$. In this case, we say $\pi:G\Rightarrow H$. 
\end{enumerate}

\proof

The second assertion is obvious, but we shall prove the first. Suppose that $\pi$ is injective. Then clearly $\ker(\pi) = \{e\}$. Now, suppose that $\ker(\pi)$ is trivial. Suppose that $\pi(a) = \pi(b)$. Then $\pi(a)\pi(b)^{-1} = e = \pi(ab^{-1})$, so $ab^{-1} = e$, so $a = b$. 

\qed

\prop

Given a group homomorphism $\pi: G\to H$, the kernel of $\pi$ is a normal subgroup of $G$. 

\proof

Choose $x \in G, y \in \ker(\pi)$. Then 
\[
\pi(xyx^{-1}) = \pi(x)e\pi(x)^{-1} = e
\]
So $xyx^{-1} \in \ker(\pi)$. 

\defn

Given $H\unlhd G$, we define the \underline{quotient group} as follows. 

For any $g \in G$, we denote the left coset of $H$ containing $g$ as $gH$. The set of left cosets forms a group, with the group operation given by 
\[
(gh)(g'H) = (gg')H
\]

This is not, a priori, well defined, so we will prove that now. That is, 

\lem 

$gH = g'H$ if and only if $g(g'^{-1}) \in H$. 

\proof

First, assume $gH = g'H$. Then $g \in g'H$, i.e there is some $h \in H$ such that $g = g'h$. Thus, $gg'^{-1} = h \in H$. Now, assume $g(g'^{-1}) \in H$. Then choose $x \in gH$. There is an $h \in H$ such that $g = g'h$. 

Then $g'g^{-1}x = g'g^{-1}gh = g'h$, and so $x = g'h\underbrace{(g'g^{-1})}_{\in H}\in g'H$.

This is one inclusion, and the other follows similarly. 

\claim If $H$ is normal, then $G/H$, the set of left cosets of $H$, inherits a group structure from $G$ as follows: $(gH)(g'H) := gg'H$

\proof

For well-definedness, assume that $gH = ghH$. Note that $(ghg')(gg')^{-1} = ghg'^{-1} \in H$. So by a previous claim, $gH = ghH$.



\section*{Lecture 4}


\defn

Let $G$ be a group. The \underline{automorphism group} $\Aut(G)$ of $G$ is defined as the group of bijective homomorphisms $f:G\to G$, with the group operation defined as function compositions. 

For a given $x \in G$, there is an automorphism given by $g \mapsto x^{-1}gx$, which is conjugation by $x$. Any automorphism which can be given by conjugation is called an \underline{inner automorphism}, and they form a normal subgroup of $\Aut(G)$. 

\claim Consider $f:G\to\Aut(G)$ defined by $x \mapsto (g \mapsto x^{-1}gx)$. This is a group homomorphism. 

\proof

For any $g_1, g_2 \in G$, and any $x \in G$, we have

\[
f_x(g_1g_2) = x^{-1}g_1g_2x = x^{-1}g_1xx^{-1}g_2x =f_x(g_1)f_x(g_2)
\]

So, $f_x \in \Aut(G)$. Now, let $x, y \in G$ and $g \in G$. Then

\[
f_{xy}(g) = (xy)^{-1}gxy = y^{-1}x^{-1}gxy = y^{-1}(x^{-1}gx)y = (f_x \circ f_y) (g)
\]

\qed

\defn 

The \underline{opposite group} of a group $G$ has the same underlying set as $G$, whose group operation (denoted $*_{op}$) is defined by

\[
(x*_{op}y) := yx
\]

\claim This is associative

\proof

We have 

\[
(x*_{op}y)*_{op}z = (yx)*_{op}z = zyx = x*_{op}(zy) = x*_{op}(y*_{op}z)
\]

\qed

Fact: $H$ normal is equivalent to every left coset of $H$ is a right coset.

Notation: $H\unlhd G$ means $H$ is a \underline{subgroup} of $G$. 

$H\lhd G$ means $H$ is a \underline{normal subgroup} of $G$. 

\lem Let $K\unlhd H \unlhd Q$. Let $\{x_i\}$ be a set of left coset representatives for $H\unlhd G$, i.e. $\{x_iH\}$ is a complete, non-repeating set of left cosets. 

Let $\{y_i\}$ be a set of left coset representatives for $K\unlhd H$. 

Then $\{y_ix_i\}$ is a set of left coset representatives for $K \unlhd G$. 

\cor

If $(G:H)$ denotes the number of left cosets of $H$ in $G$, then 
\[
(G:K) = (G:H)(H:K)
\]

\proof

We will prove the lemma, and the corollary will follow. 

First, why is it a complete list of cosets? First, if $g \in G$, then $\exists ! x_i$ such that $g \in x_iH$, i.e. $g = x_ih$ for some $h \in H$. Further, for $h \in H$, $\exists!y_j$ such that $h \in y_jK$. Then $g = x_ih \in x_iy_jK$.

Why is it unique? Exercise 

\qed

\defn

$|G| = (G:\{e\})$. 

\cor

For normal subgroups, $|G| = |G/N|\cdot|N|$

\defn

If $g \in G$, then the \underline{order of the element $g$} is defined as the smallest integer $k$ such that $g^k = e$. If there is no such $k$, then we say the order is $\oo$. 

\section*{Lecture 4}

Recall: $(G:H)$ is the index of $H$ in $G$, which is the number of left cosets of $H$ in $G$. We know that
\[
\frac{|G|}{|H|} = \frac{(G:e)}{(H:e)}
\]

\cor

If $g \in G$, $o(g) < \oo$, $|G|<\oo$, then $o(g) \mid |G|$. 

\cor If $G$ is a group with $|G| = p$, a prime number, then $o(g) = 1$ or $p$ for all $g \in G$. 

\lem

There are exactly 1 element of order 1 in any group. 

\proof

We know $o(e) = 1$. If $e, e'$ are both units in $G$, then $e =ee' = e'$. 

\cor

If $|G| = p$, then $G$ is cyclic. 

\proof Choose $g \neq e$. Then $o(g) = p$, so $\{g^0,g^1,\dots, g^{p - 1}\}$ is $p$ distinct elements in $G$, so $g$ must generate $G$. 

Classification of groups of order $n$. 

\begin{align*}
n = 1 & &G = \{e\} \\
n = 2 & &\text{ prime} \\
n = 3 & &\text{ prime} \\
? \\
\end{align*}

We will classify all groups of order $4$ by force, without the proper tools (eg semidirect products, sylow theorems). 

If there exists a $g$ such that $o(g) = 4$, then $G$ is cyclic. Otherwise, all $g \neq e$ have order 2. Let $G = \{e, a, b, c\}$. If $ab = a$, then $b = c$, and similarly if $ab = b$. So $ab = c$. 

Thus, we know every group of order $4$. 

\begin{center}
\begin{tabular}{c | c c c c }
 & e & a & b & c \\
\hline
e & e & a & b & c \\
a & a & e & c & b \\
b & b & c & e & a \\
c & c & b & a & e \\
\end{tabular}
\end{center}
The above is the \it{Cayley table} of the group. 

\claim The $G$ above, the \underline{Klein $4$-group}, is isomorphic to $\Z/2\Z \times \Z/2\Z$. 

\defn

For $H, K$ groups, we define their direct product $H\times K$ by 
\[
H\times K = \{(h, k) \mid (h,k)(h',k') = (hh', kk')\}
\] 

Continuing with the classification, 5 is prime, so we move on to $n = 6$. If it's non-cyclic, then $o(g) \in \{1, 2, 3\}$ for all $g \in G$. 

We can classify them by setting up a semidirect product...

Let $H, K$ be groups, and let $\psi:H\to\Aut(N)$. We define the semidirecN product by $H\ltimes N$. The underlying set is $H\times N$, and the group operation is given by 
\[
(x_1,h_1)(x_2,h_2) = (x_1\psi(h_2)(x_2),h_1h_2)
\]




$\vdots$ 

7 is prime. Consider $n = 8$. 

The abelian groups of this order are $\Z/8\Z$, $\Z/4\Z\times\Z/2\Z$, and $\Z/2\Z\times\Z/2\Z\times\Z/2\Z$.  

If it isn't abelian, it could be $D_4$ or $Q_8$. $Q_8$ is the quaternion group, consisting of $\{\pm 1, \pm i, \pm j, \pm k\}$, with $i^2 = j^2 = k^2 = ijk = -1$. 

It is called this because it is the group of units of $\H$, the quaternions. $H$ is for William Rowe Hamilton, who first wrote this down, and whose name scans exactly the same as ``Alexander Hamilton." 

Let's move on to $n = 9$. It could be cyclic, and if it is cyclic but non-ableian, then it could be $\Z/3\Z\times\Z/3\Z$. It could be that there exists a different group such that $g\neq e \implies o(g) = 3$. This turns out not to be the case, but proving it with only what we have now would be tough. 

If $|G| = 10$, then $G$ is either cyclic, a product of cyclics, or dihedral. 

\underline{Basic Constructions}
\begin{enumerate}
\item Theorem about finitely generated abelian groups
\item Semidirect product constructions: If $N\lhd G$, then $G$ acts on $N$ by conjugation: $\psi:G\to\Aut(N)$, $x\mapsto(g\mapsto xgx^{-1})$. 

For a semidirect product, we need $H \lhd G$ such that $|G| = |N||H|$. 
\item Special constructions.  
\end{enumerate}

One of the big steps for the class project will be finding normal subgroups. 

\defn A group $G$ is \underline{simple} if it has no nontrivial normal subgroups. 

Strategy to understand all finite groups: 
\begin{enumerate}
\item Find simple ones. This one has been achieved through work starting in the 1960s through the 1980s, and 10000 journal pages. 
\item Assemble them into more complicated ones
\end{enumerate}

There are a few big classes of finite simple groups: 
\begin{itemize}
\item Alternating group $A_n$, for $n \geq 3$
\item Finite groups of Lie type, groups of matrices over finite fields.  
\end{itemize}

There are also the sporadic groups, of which there are 26, which don't fit into any other category. They include the monster group, the largest simple group, and the baby monster group, the second largest simple group. 

\section*{Lecture 5}

Correction to a statement made last time: 

There is something called the $J$-invariant of elliptic curves. An elliptic curve is a torus with a complex structure. This is $\C/\Lambda$, where $\Lambda$ is a free $\Z$-module of rank $2$. Let $\tau$ be one of the generators. We want all things of the form $\Lambda_\tau = \{m + n\tau\}$. 

For a change of basis in $\Lambda$, $\begin{pmatrix} a & b \\ c & d \end{pmatrix} \in \SL_2(\Z)$. I don't really know what the heck he's talking about lol, I'm gonna go fiddle with the preamble. 

Groups can be represented on vector spaces. We want to define ``multiplication by elements of $G$" on elements of $V$, such that $g\cdot(v_1 + v_2) = g\cdot v_1 + g\cdot v_2$, nad $(g\cdot(cv)) = c(g\cdot v)$, where $c \in k$, the ground field of $V$. 

This gives a homomoprhism $G\to \GL(V)$. 

More generally, groups can ``act" on sets (ignoring any other structure). Recall that for a set $S$, we define $\operatorname{Perm}(S)$ as the set of all bijective $f:S\to S$. 

\defn

A \underline{group action} on the set $S$ is a group homomorphism $\pi:G\to\operatorname{Perm}(S)$. We denote for $g \in G$, $s \in S$, we denote $\pi(g)(s)$ by $g\cdot s$. Note that $(g_1\cdot g_2)\cdot s = g_1 \cdot (g_2 \cdot s)$. 

\defn

If $G\to\operatorname{Perm}(S)$ is a group action, the \underline{orbit} of the action containing $s \in S$ is 
\[
\{g\cdot s \mid g \in G\}
\]
Note that $S = S_1\coprod S_2 \coprod \cdots$ is a decomposition of $S$ into disjoint orbits. We can then obtain $G\to \operatorname{Perm}(S_i)$. 

Consider $n\times n$ matrices that are invertible and have a single $1$ and all the rest $0$ in each row and each column. This is a subset of $\GL(n, \mathbb{F})$. We can represent a group as matrices by constructing a basis indexed by the group elements $\{v_{g_1}, v_{g_2}, \dots, v_{g_d}\}$, and define $\pi(g)(v_{g_i}) = v_{gg_i}$. 

In this way we can identify $S_3$ with the set of $3\times 3$ matrices which act on the basis $\{v_{s_1}, v_{s_2}, v_{s_3}\}$ according to the permutation of the indices. 

So, given $\sigma \in jS_3$, there is a $\pi(\sigma)\in \GL(3, \mathbb{F})$, and $(6, \operatorname{char}\mathbb{F})$. 

Any element of $S_n$ can be decomposed into cycles, like so
\begin{align*}
S_n = & \{1, \sigma(1), \sigma^2(1), \dots, \sigma^n(1) \} \\
	\cup	& \{i, \sigma(i), \dots \} \\
	\cup	& \vdots \\
\end{align*}






























\end{document}