\documentclass[x11names,reqno,14pt]{extarticle}
% Choomno Moos
% Portland State University
% Choom@pdx.edu


%% stupid experiment %%
%%%%%%%%%%%%% PACKAGES %%%%%%%%%%%%%

%%%% SYMBOLS AND MATH %%%%
\let\oldvec\vec
\usepackage{authblk}	% author block customization
\usepackage{microtype}	% makes stuff look real nice
\usepackage{amssymb} 	% math symbols
\usepackage{siunitx} 	% for SI units, and the degree symbol
\usepackage{mathrsfs}	% provides script fonts like mathscr
\usepackage{mathtools}	% extension to amsmath, also loads amsmath
\usepackage{esint}		% extended set of integrals
\mathtoolsset{showonlyrefs} % equation numbers only shown when referenced
\usepackage{amsthm}		% theorem environments
\usepackage{relsize}	%font size commands
\usepackage{bm}			% provides bold math
\usepackage{bbm}		% for blackboard bold 1

%%%% FIGURES %%%%
\usepackage{graphicx} % for including pictures
\usepackage{float} % allows [H] option on figures, so that they appear where they are typed in code
\usepackage{caption}
\usepackage{hyperref}
%\usepackage{titling}
\usepackage{tikz} % for drawing
\usetikzlibrary{shapes,arrows,chains,positioning,cd,decorations.pathreplacing,decorations.markings,hobby,knots,braids}
\usepackage{subcaption}	% subfigure environment in figures

%%%% MISC %%%%
\usepackage{enumitem} % for lists and itemizations
\setlist[enumerate]{leftmargin=*,label=\bf \arabic*.}

\usepackage{multicol}
\usepackage{multirow}
\usepackage{url}
\usepackage[symbol]{footmisc}
\renewcommand{\thefootnote}{\fnsymbol{footnote}}
\usepackage{lastpage} % provides the total number of pages for the "X of LastPage" page numbering
\usepackage{fancyhdr}
\usepackage{manfnt}
\usepackage{nicefrac}
%\usepackage{fontspec}
%\usepackage{polyglossia}
%\setmainlanguage{english}
%\setotherlanguages{khmer}
%\newfontfamily\khmerfont[Script=Khmer]{Khmer Busra}

%%% Khmer script commands for math %%%
%\newcommand{\ka}{\text{\textkhmer{ក}}}
%\newcommand{\ko}{\text{\textkhmer{ត}}}
%\newcommand{\kha}{\text{\textkhmer{ខ}}}

%\usepackage[
%backend=biber,
% numeric
%style=numeric,
% APA
%bibstyle=apa,
%citestyle=authoryear,
%]{biblatex}

\usepackage[explicit]{titlesec}
%%%%%%%% SOME CODE FOR REDECLARING %%%%%%%%%%

\makeatletter
\newcommand\RedeclareMathOperator{%
	\@ifstar{\def\rmo@s{m}\rmo@redeclare}{\def\rmo@s{o}\rmo@redeclare}%
}
% this is taken from \renew@command
\newcommand\rmo@redeclare[2]{%
	\begingroup \escapechar\m@ne\xdef\@gtempa{{\string#1}}\endgroup
	\expandafter\@ifundefined\@gtempa
	{\@latex@error{\noexpand#1undefined}\@ehc}%
	\relax
	\expandafter\rmo@declmathop\rmo@s{#1}{#2}}
% This is just \@declmathop without \@ifdefinable
\newcommand\rmo@declmathop[3]{%
	\DeclareRobustCommand{#2}{\qopname\newmcodes@#1{#3}}%
}
\@onlypreamble\RedeclareMathOperator
\makeatother

\makeatletter
\newcommand*{\relrelbarsep}{.386ex}
\newcommand*{\relrelbar}{%
	\mathrel{%
		\mathpalette\@relrelbar\relrelbarsep
	}%
}
\newcommand*{\@relrelbar}[2]{%
	\raise#2\hbox to 0pt{$\m@th#1\relbar$\hss}%
	\lower#2\hbox{$\m@th#1\relbar$}%
}
\providecommand*{\rightrightarrowsfill@}{%
	\arrowfill@\relrelbar\relrelbar\rightrightarrows
}
\providecommand*{\leftleftarrowsfill@}{%
	\arrowfill@\leftleftarrows\relrelbar\relrelbar
}
\providecommand*{\xrightrightarrows}[2][]{%
	\ext@arrow 0359\rightrightarrowsfill@{#1}{#2}%
}
\providecommand*{\xleftleftarrows}[2][]{%
	\ext@arrow 3095\leftleftarrowsfill@{#1}{#2}%
}
\makeatother

%%%%%%%% NEW COMMANDS %%%%%%%%%%

% settings
\newcommand{\N}{\mathbb{N}}                     	% Natural numbers
\newcommand{\Z}{\mathbb{Z}}                     	% Integers
\newcommand{\Q}{\mathbb{Q}}                     	% Rationals
\newcommand{\R}{\mathbb{R}}                     	% Reals
\newcommand{\C}{\mathbb{C}}                     	% Complex numbers
\newcommand{\K}{\mathbb{K}}							% Scalars
\newcommand{\F}{\mathbb{F}}                     	% Arbitrary Field
\newcommand{\E}{\mathbb{E}}                     	% Euclidean topological space
\renewcommand{\H}{{\mathbb{H}}}                   	% Quaternions / Half space
\newcommand{\RP}{{\mathbb{RP}}}                       % Real projective space
\newcommand{\CP}{{\mathbb{CP}}}                       % Complex projective space
\newcommand{\Mat}{{\mathrm{Mat}}}						% Matrix ring
\newcommand{\M}{\mathcal{M}}
\newcommand{\GL}{{\mathrm{GL}}}
\newcommand{\SL}{{\mathrm{SL}}}

\newcommand{\tgl}{\mathfrak{gl}}
\newcommand{\tsl}{\mathfrak{sl}}                  % Lie algebras; i.e., tangent space of SO/SL/SU
\newcommand{\tso}{\mathfrak{so}}
\newcommand{\tsu}{\mathfrak{sl}}


% typography
\newcommand{\noi}{\noindent}						% Removes indent
\newcommand{\tbf}[1]{\textbf{#1}}					% Boldface
\newcommand{\mc}[1]{\mathcal{#1}}               	% Calligraphic
\newcommand{\ms}[1]{\mathscr{#1}}               	% Script
\newcommand{\mbb}[1]{\mathbb{#1}}               	% Blackboard bold


% (in)equalities
\newcommand{\eqdef}{\overset{\mathrm{def}}{=}}		% Definition equals
\newcommand{\sub}{\subseteq}						% Changes default symbol from proper to improper
\newcommand{\psub}{\subset}						% Preferred proper subset symbol

% Categories
\newcommand{\catname}[1]{{\text{\sffamily {#1}}}}

\newcommand{\Cat}{{\catname{C}}}
\newcommand{\cat}[1]{{\catname{\ifblank{#1}{C}{#1}}}}
\newcommand{\CAT}{{\catname{Cat}}}
\newcommand{\Set}{{\catname{Set}}}

\newcommand{\Top}{{\catname{Top}}}
\newcommand{\Met}{{\catname{Met}}}
\newcommand{\PL}{{\catname{PL}}}
\newcommand{\Man}{{\catname{Man}}}
\newcommand{\Diff}{{\catname{Diff}}}

\newcommand{\Grp}{{\catname{Grp}}}
\newcommand{\Grpd}{{\catname{Grpd}}}
\newcommand{\Ab}{{\catname{Ab}}}
\newcommand{\Ring}{{\catname{Ring}}}
\newcommand{\CRing}{{\catname{CRing}}}
\newcommand{\Mod}{{\mhyphen\catname{Mod}}}
\newcommand{\Alg}{{\mhyphen\catname{Alg}}}
\newcommand{\Field}{{\catname{Field}}}
\newcommand{\Vect}{{\catname{Vect}}}
\newcommand{\Hilb}{{\catname{Hilb}}}
\newcommand{\Ch}{{\catname{Ch}}}

\newcommand{\Hom}{{\mathrm{Hom}}}
\newcommand{\End}{{\mathrm{End}}}
\newcommand{\Aut}{{\mathrm{Aut}}}
\newcommand{\Obj}{{\mathrm{Obj}}}
\newcommand{\op}{{\mathrm{op}}}

% Norms, inner products
\delimitershortfall=-1sp
\newcommand{\widecdot}{\, \cdot \,}
\newcommand\emptyarg{{}\cdot{}}
\DeclarePairedDelimiterX{\norm}[1]{\Vert}{\Vert}{\ifblank{#1}{\emptyarg}{#1}}
\DeclarePairedDelimiterX{\abs}[1]\vert\vert{\ifblank{#1}{\emptyarg}{#1}}
\DeclarePairedDelimiterX\inn[1]\langle\rangle{\ifblank{#1}{\emptyarg,\emptyarg}{#1}}
\DeclarePairedDelimiterX\cur[1]\{\}{\ifblank{#1}{\emptyarg,\emptyarg}{#1}}
\DeclarePairedDelimiterX\pa[1](){\ifblank{#1}{\emptyarg}{#1}}
\DeclarePairedDelimiterX\brak[1][]{\ifblank{#1}{\emptyarg}{#1}}
\DeclarePairedDelimiterX{\an}[1]\langle\rangle{\ifblank{#1}{\emptyarg}{#1}}
\DeclarePairedDelimiterX{\bra}[1]\langle\vert{\ifblank{#1}{\emptyarg}{#1}}
\DeclarePairedDelimiterX{\ket}[1]\vert\rangle{\ifblank{#1}{\emptyarg}{#1}}

% mathmode text operators
\RedeclareMathOperator{\Re}{\operatorname{Re}}		% Real part
\RedeclareMathOperator{\Im}{\operatorname{Im}}		% Imaginary part
\DeclareMathOperator{\Stab}{\mathrm{Stab}}
\DeclareMathOperator{\Orb}{\mathrm{Orb}}
\DeclareMathOperator{\Id}{\mathrm{Id}}
\DeclareMathOperator{\vspan}{\mathrm{span}}			% Vector span
\DeclareMathOperator{\tr}{\mathrm{tr}}
\DeclareMathOperator{\adj}{\mathrm{adj}}
\DeclareMathOperator{\diag}{\mathrm{diag}}
\DeclareMathOperator{\eq}{\mathrm{eq}}
\DeclareMathOperator{\coeq}{\mathrm{coeq}}
\DeclareMathOperator{\coker}{\mathrm{coker}}
\DeclareMathOperator{\dom}{\mathrm{dom}}
\DeclareMathOperator{\cod}{\mathrm{codom}}
\DeclareMathOperator{\im}{\mathrm{im}}
\DeclareMathOperator{\Dim}{\mathrm{dim}}
\DeclareMathOperator{\codim}{\mathrm{codim}}
\DeclareMathOperator{\Sym}{\mathrm{Sym}}
\DeclareMathOperator{\lcm}{\mathrm{lcm}}
\DeclareMathOperator{\Inn}{\mathrm{Inn}}
\DeclareMathOperator{\sgn}{sgn}						% sgn operator
\DeclareMathOperator{\intr}{\text{int}}             % Interior
\DeclareMathOperator{\co}{\mathrm{co}}				% dual/convex Hull
\DeclareMathOperator{\Ann}{\mathrm{Ann}}
\DeclareMathOperator{\Tor}{\mathrm{Tor}}


% misc symbols
\newcommand{\divides}{\big\lvert}
\newcommand{\grad}{\nabla}
\newcommand{\veps}{\varepsilon}						% Preferred epsilon
\newcommand{\vphi}{\varphi}
\newcommand{\del}{\partial}							% Differential/Boundary
\renewcommand{\emptyset}{\text{\O}}					% Traditional emptyset symbol
\newcommand{\tril}{\triangleleft}					% Quandle operation
\newcommand{\nabt}{\widetilde{\nabla}}				% Contravariant derivative
\newcommand{\later}{$\textcolor{red}{\blacksquare}$}% Laziness indicator

% misc
\mathchardef\mhyphen="2D							% mathomode hyphen
\renewcommand{\mod}[1]{\ (\mathrm{mod}\ #1)}
\renewcommand{\bar}[1]{\overline{#1}}				% Closure/conjugate
\renewcommand\qedsymbol{$\blacksquare$} 			% Changes default qed in proof environment
%%%%% raised chi
\DeclareRobustCommand{\rchi}{{\mathpalette\irchi\relax}}
\newcommand{\irchi}[2]{\raisebox{\depth}{$#1\chi$}}
\newcommand\concat{+\kern-1.3ex+\kern0.8ex}

% Arrows
\newcommand{\weak}{\rightharpoonup}					% Weak convergence
\newcommand{\weakstar}{\overset{*}{\rightharpoonup}}% Weak-star convergence
\newcommand{\inclusion}{\hookrightarrow}			% Inclusion/injective map
\renewcommand{\natural}{\twoheadrightarrow}				% Natural map

% Environments
\theoremstyle{plain}
\newtheorem{thm}{Theorem}[section]
%\newtheorem{lem}[thm]{Lemma}
\newtheorem{lem}{Lemma}
\newtheorem*{lems}{Lemma}
\newtheorem{cor}[thm]{Corollary}
\newtheorem{prop}{Proposition}
\newtheorem*{claim}{Claim}
\newtheorem*{cors}{Corollary}
\newtheorem*{props}{Proposition}
\newtheorem*{conj}{Conjecture}

\theoremstyle{definition}
\newtheorem{defn}{Definition}[section]
\newtheorem*{defns}{Definition}
\newtheorem{exm}{Example}[section]
\newtheorem{exer}{Exercise}[section]

\theoremstyle{remark}
\newtheorem*{rem}{Remark}

\newtheorem*{solnx}{Solution}
\newenvironment{soln}
    {\pushQED{\qed}\renewcommand{\qedsymbol}{$\Diamond$}\solnx}
    {\popQED\endsolnx}%

% Macros
\newcommand{\restr}[1]{_{\mkern 1mu \vrule height 2ex\mkern2mu #1}}
\newcommand{\Upushout}[5]{
    \begin{tikzcd}[ampersand replacement = \&]
    \&#2\ar[rd,"\iota_{#2}"]\ar[rrd,bend left,"f"]\&\&\\
    #1\ar[ur,"#4"]\ar[dr,"#5"]\&\&#2\oplus_{#1} #3\ar[r,dashed,"\vphi"]\&Z\\
    \&#3\ar[ur,"\iota_{#3}"']\ar[rru,bend right,"g"']\&\&
    \end{tikzcd}
}
\newcommand{\exactshort}[5]{
		\begin{tikzcd}[ampersand replacement = \&]
			0\ar[r]\&#1\ar[r,"#2"]\& #3 \ar[r,"#4"]\& #5 \ar[r]\&0
		\end{tikzcd}
}
\newcommand{\product}[6]{
		\begin{tikzcd}[ampersand replacement = \&]
			#1 \& #2 \ar[l,"#4"'] \\
			#3 \ar[u,"#5"] \ar[ur,"#6"']
		\end{tikzcd}
}
\newcommand{\coproduct}[6]{
		\begin{tikzcd}[ampersand replacement = \&]
			#1 \ar[r,"#4"] \ar[d,"#5"'] \& #2 \ar[dl,"#6"] \\
			#3
		\end{tikzcd}
}
%%%%%%%%%%%% PAGE FORMATTING %%%%%%%%%

\usepackage{geometry}
    \geometry{
		left=15mm,
		right=15mm,
		top=15mm,
		bottom=15mm	
		}

\usepackage{color} % to do: change to xcolor
\usepackage{listings}
\lstset{
    basicstyle=\ttfamily,columns=fullflexible,keepspaces=true
}
\usepackage{setspace}
\usepackage{setspace}
\usepackage{mdframed}
\usepackage{booktabs}
\newcommand*{\oo}{{\infty}}
\DeclareMathOperator{\cl}{cl}
\usepackage[document]{ragged2e}
\usepackage{amsmath}
\pagestyle{fancy}{
	\fancyhead[L]{Fall 2022}
	\fancyhead[C]{221A - General Topology}
	\fancyhead[R]{John White}
  
  \fancyfoot[R]{\footnotesize Page \thepage \ of \pageref{LastPage}}
	\fancyfoot[C]{}
	}
\fancypagestyle{firststyle}{
     \fancyhead[L]{}
     \fancyhead[R]{}
     \fancyhead[C]{}
     \renewcommand{\headrulewidth}{0pt}
	\fancyfoot[R]{\footnotesize Page \thepage \ of \pageref{LastPage}}
}

\title{220A - Groups}
\author{John White}
\date{Fall 2022}




\begin{document}

\section*{Lecture 1}

\defn

A \underline{metric space} is a set $X$ equipped with a function $d:X\times X \to \R$, which satisfies the following axioms:
\begin{enumerate}
	\item For any $x, y \in X$, $d(x, y) = d(y, x)$
	\item For any $x, y, z \in X$, we have $d(x, y) \leq d(x, z) + d(z, y)$. This is called the ``triangle inequality"
	\item For any $x, y \in X$, $d(x, y) = 0$ exactly when $x = y$

\end{enumerate}

\exm

For $x, y \in \R^n$, 
\[
d(x, y) \eqdef \left(\sum_{i=1}^n(x_i - y_i)^2\right)^{\frac{1}{2}}
\]

This is called the Euclidean distance. 2 can be replaced with any real $r \geq 1$, and it will still be a metric. 

\exm 

In this example, $C[0, 1]$ is the set of all continuous functions $f:[0, 1]\to\R$. Here, 
\[
d(f, g) \eqdef \sup_{x\in[0,1]}|f(x)- g(x)|
\]

\exm

Let $X = \N$, the natural numbers, including $0$. Let $p$ be a fixed prime number. The $p$-adic metric on $\N$ is defined by 
\[
d_p(a, b) \eqdef \frac{1}{p^\alpha} 
\]
Where $p^\alpha$ is the largest power of $p$ which divides $|a - b|$. So two naturals are ``close" if their difference is divisible by a high power of $p$. 

\claim This is a metric

\proof
The 1st and 3rd axioms are clear. So we must prove the triangle inequality. We will consider the three quantities $d_p(a, b), d_p(a, t)$, and $d_p(b, t)$, where $a, b, t \in \N$. 

Suppose $p^\beta$ divides both $a - t$ and $t - b$. Then $p^\beta$ divides $(a - t) + (t - b) = a - b$. Therefore, 
\[
d_p(a, b) \leq \frac{1}{p^\beta} \leq \max(d_p(a, t), d_p(t, b)) \leq d_p(a, t) + d_p(t, b)
\]
\qed

\defn

Let $(X, d_X), (Y, d_Y)$ be two metric spaces. For a function $f:X\to Y$, we say that $f$ is \underline{continuous} at $x_0 \in X$ if, for all $\epsilon > 0$, there is a $\delta > 0$ such that 
\[
0 < |d_X(x_0, x)| < \delta \implies |d_Y(f(x_0), f(x))| < \epsilon
\]

A function $f:X\to Y$ is said to be continuous if it is continuous at $x$ for all $x \in X$. 

\exm

Consider a map $(\N, d_5) \to (\N, d_5)$ defined by 
\[
x\mapsto x^2
\]

Is this continuous? 

At 0, to be continuous, then for any $x$, if we want to get within a small distance of 0, then $x$ has to be divisible by large powers of $5$. 

What about at $11$?

This is continuous. 

\exm

What about $(\N,d_5) \to (\N, d_{17})$. 

\section*{Lecture 2}

\thm

If $f:(X, d_X)\to (Y, d_Y)$ and $g:(Y, d_Y)\to(Z, d_Z)$ are both continuous, then $g \circ f:(X, d_X)\to(Z, d_Z)$

\proof

Fix $x \in X$ and $\varepsilon>0$. Choose $\delta_1 > 0$ so that if $d_Y(f(x), y) < \delta_1$, then $d_Z(gf(x), g(y)) < \varepsilon$. 

By continuity of $f$, we may then choose a $\delta_0 > 0$ such that if $d_X(x, x') < \delta_2$, then $d_Y(f(x), f(x')) < \delta_1$. 

\qed

\defn For a metric space $(X, d_X)$, and a real $r > 0$, the open $r$-ball around a point $x$ is defined as
\[
B_r(x) = \{x' \in X \mid d(x, x') < r\}
\]

Exercise: State and prove some theorem about the existence of a function from $X \times X' \to Y \times Y'$, given a function $f:X\to Y$ and $g:X'\to Y'$. 

\exm{Balls}

\begin{enumerate}
\item In $\R^2$, consider 
\[
d_r\left(\begin{pmatrix} x_1 \\ y_1 \end{pmatrix}, \begin{pmatrix} x_2 \\ y_2 \end{pmatrix} \right) = \left(\sum_{i=1}^2 (x_i - y_i)^r\right)^{\frac{1}{r}}
\]

For $r = 2$, this is the usual euclidean distance. For $r = 1$, the balls look like diamonds. In the limit, as $r\to \infty$, the metric will approach what is known as the ``box metric," in which the distance between any point and 0 is it's largest coordinate. 
\item On $C[0, 1]$, the set of continuous functions from $[0, 1]$ to $\R$, we have the sup metric: 
\[
d(f, g) = \sup_{x\in[0, 1]}|f(x) - g(x)|
\]

\item We also have
\[
d_1(f, g) = \int_0^1d|f(x) - g(x)|dx
\]



\end{enumerate}

\defn

For a metric space $(X, d_X)$, suppose that $U \subseteq X$ is said to be ``open" if, for any $x \in U$, there is a $\varepsilon>0$ such that $B_\varepsilon(x) \subseteq U$. 

\lem 

$B_\varepsilon(x)$ is always open. 

\proof

Let $y \in B_\varepsilon(x)$. Let $t = d(x, y)$. By construction, $t <\varepsilon$. Let $\delta = \varepsilon - t$. Consider $B_\delta(y)$. For any $z \in B_\delta(y)$, we have by the triangle inequality
\[
d(x, z) \leq d(x, y) + d(y, z) < \varepsilon - t + t = \varepsilon
\]
and so $z \in B_\varepsilon(x)$. $z$ was arbitrary, so we are done.

\qed

\section*{Lecture 3}

\defn Let $(X, d)$ be a metric space. A set $U\subseteq X$ is said to be \underline{open} if for all $x \in U$, there is an $\varepsilon > 0$ such that $B(x, \varepsilon) \subseteq U$. 

\thm Let $\{U_\alpha\}_{\alpha\in A}$ be a collection of open sets. Then

\begin{enumerate}
\item $\cup_{\alpha\in A}U_\alpha$ is open. 
\item Let $U_1, \dots, U_n$ be a finite subcollection. Then $\cap_{i=1}^\oo U_i$ is open. 
\end{enumerate}

\proof




\section*{Lecture 4}

\defn 

Two metrics $d_1, d_2$ are said to be \underline{equivalent} on a space $X$ if any set which is open under the $d_1$-induced topology is also open under the $d_2$-induced topology, and vice versa. 

\exm 

In $\R^2$, 
\begin{align*}
d_2(0, (x, y)) & = (x^2 + y^2)^{\frac{1}{2}}\\
d_1(0, (x, y)) & = |x| + |y| \\
d_{\infty}(0, (x, y)) & = \max\{|x|, |y|\} \\
\end{align*}

How do these metric's unit balls compare? In fact, $d_1$'s is within $d_2$'s, which is within $d_{\oo}$'s.

But all of these balls contain a ball of radius $\frac{1}{2}$ in any of the three metric. Thus, these are equivalent. 

\defn

Two metrics $d_1, d_2$ are called \underline{Lipschitz equivalent} if there exists some $k \in \R$ such that, for all $x, y \in X$, we have

\[
\frac{1}{k}d_2(x, y) 
< d_1(x, y) < kd_2(x, y)
\]

\exm

This is a non-example. The 5-adic and the 17-adics are not equivalent. 

\exm

Let 
\begin{align*}
d_1(f, g) & = \int_0^1|f(x) - g(x)|\,dx \\
d_\oo(f, g) & = \sup_{x\in[0, 1]}|f(x) - g(x)| \\
\end{align*}

One controls for area, one controls for the maximum value of $f$. We have
\[
B^{d_\oo}_\varepsilon (0) \subset B^{d_1}_\varepsilon(0)
\] 
This is because if we control for the maximum size of $f$, we can surely control for the area under it. However, no matter how much we limit the area under $f$, there is some $f$ which has that much area which has a sup greater than some $\varepsilon$ which is fixed. 

\thm

Let $d_1, d_2$ be equivalent metrics on $X$. Then the following are equivalent
\begin{enumerate}
\item $f:X\to Y$ is $d_1, d_Y$ continuous if and only if it is $d_2, d_Y$ continuous. 
\item $g:Z\to X$ is $d_Z, d_1$ continuous if and only if if is $d_Z, d_2$ continuous. 
\end{enumerate}

\proof

\begin{enumerate}
\item Let $U \subset Y$ be open. We know that the preimage $f^{-1}(U)$ is open under the $d_1$ metric. But by the equivalence of $d_1, d_2$, $f^{-1}(U)$ must also be open under the $d_2$ metric. The reverse argument also holds. 
\item Let $U \subset X$ be an open set under the $d_1$ metric. By continuity of $g$, we know $g^{-1}(U)$ is open. However, $U$ must also be open under the $d_2$ metric, meaning that $g$ must be continuous with respect to both metrics. 
\end{enumerate}

\qed

Recall: If $\{U_\alpha\}_{\alpha\in A}$ is a collection of open sets, then $\cup_{\alpha\in A}U_\alpha$ is open. 

We can associate to any set $R\subset X$ an open set, called the interior of $R$. 

\defn

For any $\R\subset X$, we define it's \underline{interior} by 
\[
\operatorname{int}(R) = \bigcup_{U\text{ open}, U \subseteq R}U
\]

We can say several things about $\operatorname{int}(R)$. 
\begin{enumerate}
\item $\operatorname{int}(R)$ is open. 
\item If $U$ is open, then $\operatorname{int}(U) = U$, and vice versa. 
\item Suppose $A \subseteq B$. Then $\operatorname{int}(A) \subseteq \intr(B)$. 
\end{enumerate}

Recall: If $\{C_\alpha\}_{\alpha\in A}$ are all closed, then $\cap_{\alpha\in A}C_\alpha$ is closed. 

\defn 

If $R\subseteq X$, then the \underline{closure} of $R$, denoted by $\bar{R}$, or sometimes $\cl(R)$, is defined as 
\[
\bar{R}\eqdef \bigcap_{C\text{ closed}, C\supseteq R}C
\]

Analagously, 
\begin{enumerate}
\item $\bar{R}$ is closed for any $R$. 
\item $R$ is closed if and only if $R = \bar{R}$. 
\item If $A \subset B$, then $\cl(A) \subset \cl(B)$. 
\end{enumerate}

\prop

Let $x \in \cl(R)$. Then, for all $\varepsilon>0$, $B_\varepsilon(x)\cap R \neq \varnothing$, and vice-versa. 

We will prove this next lecture. 

\section*{Lecture 4}

\proof 

First, suppose that $x\not\in \bar{A}$. The complement of $\bar{A}$ is open, so there exists a $\varepsilon>0$ such that $B_{\varepsilon}(x)\subseteq (\bar{A})^c$, and so $B_{\varepsilon}(x) \cap \bar{A} = \varnothing$. 

Now, suppose that there exists some $\varepsilon> 0$ such that $B_{\varepsilon}(x) \cap A = \varnothing$. Then $(B_{\varepsilon}(x))^c$ is a closed set containing $A$ which does not contain $x$, thus $\bar{A}$ cannot contain $x$. 

\qed

Now, it is time for the main event. 

\subsection*{\underline{Topological Spaces}}

Let $X$ be a set. Let $\ms{P}(X)$ denote the power set of $X$, which is the set of all subsets of $X$. 

\defn

Let $\ms{T} \subset\ms{P}(X)$. $\ms{T}$ is a \underline{topology on $X$} if it has the following properties: 
\begin{enumerate}
\item $\varnothing, X \in \ms{T}$. 
\item If $\{U_\alpha\}_{\alpha\in A}$ with each $U_\alpha \in \ms{T}$, then $\bigcup_{\alpha\in A}U_\alpha \in \ms{T}$. 
\item If $A, B \in \ms{T}$, then $A \cap B \in \ms{T}$. Of course, we can ``strengthen" this to the equivalent statement: if $U_1, \dots, U_k \in \ms{T}$, then $\cap_{i=1}^kU_i \in \ms{T}$
\end{enumerate}

Elements of $\ms{T}$ are called ``open sets." 

\exm Here are some simple examples. 


\begin{enumerate}
\item Open sets in $(X, d)$ form a topology (hence the definition)
\item $\ms{T} = \ms{P}(X)$ forms the discrete topology. 
\item $\ms{T} = \{\varnothing, X\}$ forms the indescrete topolgy.
\item If $X = \{0, 1\}$, then $\ms{T} = \{\varnothing, X, \{0\}\}$ forms a topology. 
\item We can form the \underline{Zariski topology} on $\R$ by specifying the closed sets, which satisfy a similar but slightly different set of axioms. We define the closed sets to be $\varnothing, \R$, and any finite collection of points. 

More generally, the Zariski topology on some ring $R$ is specified by its closed sets, which are the solution locii of some set of polynomials in $R$. 
\item Let $X = \R$, $\ms{T} = \{(r, \oo)\mid r\in\R\} \cup \{\varnothing, \R\}$. 
\end{enumerate}

We have now defined a set of objects (topological spaces). Now, we want to define morphisms, the maps between spaces. 

\defn Let $(X, \ms{T}_X), (Y, \ms{T}_Y)$ be topological spaces. We say a function $f:X\to Y$ is \underline{continuous} if, for all $V\in\ms{T}_Y$, $f^{-1}(V) \in \ms{T}_X$. In other words, the preimage of any open subset of $Y$ is an open subset of $X$. 

\exm

\begin{enumerate} Some baby examples
\item Any function $f:(X, \text{discrete})\to (Y, \ms{T}_Y)$ is continuous as long as $X$ has the discrete metric, as the preimage of any open set will be a subset of $X$, all of which are open under the discrete topology. 
\item Any function $f:(X,\ms{T}_X)\to(Y,\text{ discrete})$ is continous for a similar reason. 
\item $\Id_X:(X, \ms{T}_X)\to(X,\ms{T}_X)$ is continuous. 
\end{enumerate}

\thm 

Let $f:(X,\ms{T}_X)\to(Y,\ms{T}_Y)$ and $g:(Y,\ms{T}_Y)\to(Z,\ms{T}_Z)$ be continuous functions. Then $g \circ f:(X,\ms{T}_X)\to(Z,\ms{T}_Z)$ is continuous. 

In other words, the composition of continuous functions is continuous. 

\proof

Pick $W \in \ms{T}_Z$. Then $g^{-1}(W) \in \ms{T}_Y$ since $g$ is continuous. So, because $f$ is continuous, 
\[
(g\circ f)^{-1}(W) = f^{-1}(g^{-1}(W)) \in\ms{T}_X
\]
Hence, $g \circ f$ is continuous. 

\defn 

For $\ms{T}$ a topology on $X$, a \underline{basis} for $\ms{T}$ is a subset $\ms{B}\subseteq\ms{T}$ such that every set in $\ms{T}$ is the union of sets in $\ms{B}$. 

\section*{Lecture 5}

\prop

Suppose $f:(X, \ms{T}_X) \to (Y, \ms{T}_Y)$ is a function. Then $f$ is continuous if and only if $f^{-1}(B) \in \ms{T}_X$ for all $B \in \ms{B}$, with $\ms{B}$ a basis for $\ms{T}_Y$.

\proof

The forward direction is trivial, as a basis consists of sets which are all open. 

Now, suppose that $f^{-1}(B) \in \ms{T}_X$ for all $B \in \ms{B}$. Let $V\in \ms{T}_Y$. Write it as $\cup_{\alpha\in A}B_\alpha = V$ for $B_{\alpha}\in \ms{B}$. 

Then $f^{-1}(V) = f^{-1}(\cup_{\alpha\in A}B_\alpha) = \cup_{\alpha\in A}f^{-1}(B_\alpha)$. This is the union of open subsets of $X$, so $f^{-1}(V)$ is open. 

\qed

\exm

Consider $B_{\frac{p}{q}}(\frac{r}{s})$, with $p, q, r, s \in \Z$, $q, s \neq 0$. In other words, the balls of rational radius and rational center. This collection of sets forms a basis for $\R$ using the standard (or, as Darren calls it, the ``Mother's Knee") topology. 

Exercise: Write down the definition of interiors and closures, and check the following lemma is true: 

\lem 

$x \in \cl(A)$ if and only if, for all open $U \ni x$, $U\cap A \neq \varnothing$. 

\exm

Consider $(\N,$ Zariski). What is the closure of the collection of prime numbers under this topology? 

Recall that the Zariski topology on $\N$ defines closed sets to be either empty, $\N$, or finite. So, for example, the closure of the integers from $1$ to $10$ is itself. 

But the only closed set that can contain an infinite set is $\N$, so the closure of the primes is $\N$. More generally, for any infinite subset of $\N$, the closure is $\N$. 

\subsection*{\underline{Subobjects and product objects}}

\subsubsection*{\underline{Subobjects}}

We have now defined our objects and morphisms, so let's talk about subobjects and product objects. 

Let $(X, \ms{T}_X)$ be a topological space, and $A\subseteq X$ a nonempty subset. 

Denote the inclusion map $\iota:A\to X$. We want to topologize $A$ so that $\iota$ is continuous, and is as small as possible, in the sense that any topology for which $\iota$ is continuous includes this topology on $A$. 

For any open $U\subseteq X$, we want $\iota^{-1}(U)$ to be open. But $\iota^{-1}(U) = U \cap A$. 

\defn

For $(X, \ms{T}_X)$ a topological space and $A\subset X$ a subset, then the \underline{subspace topology} on $A$, $\ms{T}_A$, consists of 
\[
\ms{T}_A \eqdef \{U \cap A \mid U \in \ms{T}_X\}
\]

\prop

Suppose we have a commutative diagram of the form 
\[
\begin{tikzcd}
(Z, \ms{T}_Z) \arrow[rd, "\iota \circ g"'] \arrow[r, "g"] & A \arrow[d, "\iota"] \\
& X \\
\end{tikzcd}
\]

Then $\iota$ is continuous if and only if $\iota \circ g$ is continuous. 

\proof

One direction is trivial, as we know the composition of continuous functions is continuous. Now, suppose that $\iota \circ g$ is continuous. 

Let $W \in \ms{T}_A$. We know $W = W^*\cap A$ for some $W^* \in \ms{T}_X$. $\iota \circ g$ is continuous, so $(\iota \circ g)^{-1}W^*\in\ms{T}_Z$. 

So, $\ms{T}_Z\ni(\iota\circ g)^{-1} = g^{-1}(\iota^{-1}(W^*)) = g^{-1}(W)$. 

\qed

So we can see that if we want the above to hold, we are forced into our definition of $\ms{T}_A$. 

\prop

$\ms{T}_A$ is the only topology so that the previous proposition is true for all spaces $(Z, \ms{T}_Z)$ and functions $g$. 

\proof

Suppose that $\ms{T}_{\mu}$ ($\mu$ for ``mystery" topology) such that the previous proposition holds for all choices of $(Z, \ms{T}_Z)$ and $g:(Z,\ms{T}_Z)\to A$.  

Consider 
\[
\begin{tikzcd}
(A,\ms{T}_A)\arrow[rd, "\iota \circ \Id"'] \arrow[r, "\Id"] & (A, \ms{T}_\mu) \arrow[d, "\iota"] \\
& (X, \ms{T}_X) \\
\end{tikzcd}
\]

\section*{Lecture 6}

We are trying to prove that $\ms{T}_A$ is the unique topology on $A$ such that
\[
\begin{tikzcd}
(A, \ms{T}_A) \arrow[rd,"\iota\circ\Id"']\arrow[r, "\Id"] & (A, \ms{T}_\mu)\arrow[d, "\iota"] \\
& (X, \ms{T}_X) \\
\end{tikzcd}
\]

\proof

Suppose $\ms{T}_\mu$ is such a topology on $A$. Then 
\[
\begin{tikzcd}
(A, \ms{T}_A)\arrow[rd, "\iota\circ\Id"']\arrow[r,"\Id"] & (A, \ms{T}_\mu) \arrow[d, "\iota"] \\ & (X, \ms{T}_X) 
\end{tikzcd}
\]
We have $\iota\circ\Id$ is continuous, so $\begin{tikzcd}( A,\ms{T}_A)\arrow[r, "\Id"] & (A,\ms{T}_\mu) \end{tikzcd}$ is continuous. So if $W\in\ms{T}_\mu$, then $W \in \ms{T}_A$. Therefore $\ms{T}_\mu\subseteq\ms{T}_A$. 

We know that $\iota\circ\Id$ is continuous. Let $W \in \ms{T}_X$. By continuity, $(\iota\circ\Id)^{-1}W\in\ms{T}_\mu \implies(\Id)^{-1}\circ\iota^{-1} W \in \ms{T}_\mu$. 

So then $(\Id)^{-1}(W\cap A)\in\ms{T}_\mu$, and $W\cap A \in \ms{T}_\mu$. Therefore for any $W \in \ms{T}_\mu$, $W \cap A \in \ms{T}_\mu$, so $\ms{T}_A\subset\ms{T}_\mu$. 

\subsubsection*{\underline{Product Objects}}

Let $(X, \ms{T}_X), (Y,\ms{T}_Y)$ be topological spaces. We want to topologize $X\times Y$. 

Well, if that holds a topology, then the projections damn well better be continuous. In other words, we want to arrange such that
\[
\begin{tikzcd}
X\times Y \arrow[d, "\rho_Y"'] \arrow[r, "\rho_X"] & X \\
Y & \\
\end{tikzcd}
\]
both $\rho_X, \rho_Y$ are continuous. 

Let $U\in\ms{T}_X$. Then $\rho_X^{-1}U = U \times Y$. Similarly, for $V\in\ms{T}_Y$, $\rho_Y^{-1}V = X\times V$. 

Thus, for the projections to be continuous, we need that the intersection of $U\times Y$ and $X \times V$ are in $\ms{T}_{X\times Y}$ for all $U\in\ms{T}_X, V\in\ms{T}_Y$. The sets of this form, $X\times V \cap U\times Y$, form a basis for a topology. 

In other words, the product topology has basis $\{U\times V\mid U\in\ms{T}_X, V\in\ms{T}_Y\}$. 

\thm
Consider the commutative diagram

\begin{center}
\begin{tikzcd}
	& Z \ar[ld,"\rho_X\circ g"'] \ar[rd,"\rho_Y\circ g"] \ar[d,"g"]& \\
X & X\times Y \ar[l,"\pi_j"] \ar[r,"\pi_Y"']& Y
\end{tikzcd}
\end{center}

Then $g$ is continuous if and only if $\rho_X\circ g$ and $\rho_Y\circ g$ are continuous. 

\proof

Next time

\section*{Lecture 7}

\subsection*{\underline{Relevant digression}}

Let $(X_i, \ms{T}_i)_{i\geq1}$ be a family of topological spaces. A basis for a topology on $\prod_{i\geq1}X_i$ is $\{\prod_{i\geq1}u_i \mid u_i \in \ms{T}_i\}$

We can ask that the projection $\prod_{i\geq1}X_i\to X_j$ is continuous for each $j$, but then 
\begin{align*}
p_j^{-1}(V_j) & = X_1\times\cdots\times X_{j - 1}\times V_j \times X_{j + 1} \times \cdots \\
\end{align*}

\defn

Let $(X_i, \ms{T}_i)_{i\geq1}$ be a family of topological spaces. Then the 

\underline{Tychonoff's product topology} has a basis consisting of sets of the form $\prod_{i\geq1}V_i$, where $V_j \subseteq X_j$, and $V_j = X_j$ for all but finitely many $j$. 

\exm

Let $p$ be prime. Then $\Z/p^k$ will denote $\Z/p^k\Z$. We may equip it with the discrete topology. Consider $\prod_{k\geq1}\Z/p^k$. Let's put a metric on this. We will define
\[
d(x, y) = \sum_{k\geq1}\frac{1}{2^k}d_k(x_k, y_k)
\]
It is easy to convince yourself this is a metric. 

Note that there exists a map $\Z\to\prod_{k\geq1}\Z/p^k$, given by $x \mapsto (x\mod p, x\mod {p^2}, \dots)$

An integer $m$ is close to zero in this metric if $m$ is divisible by large powers of $p$. 

This is (basically) the $p$-adic metric. 

\underline{Exercise}: What is the closure of $\Z$ in this metric space? 

\subsection*{\underline{Hausdorff Spaces}}

\defn

Here is what Darren calls a ``reasonable definition" of convergence of a sequence in a general topological space. 

Let $x_n$ be a seequence in $(X,\ms{T})$. We say that $x_n$  \underline{converges to} $x$ if, for any $x \in U \in \ms{T}$, there exists an $N$ such that $x_k \in U$ for all $k \geq N$. 

\exm

Let $\mc{P}$ be the set of prime numbers, and let $\N$ have the Zariski topology. What does $\mc{P}$ converge to? 

Consider $193$. An open neighborhood $U$ of this would be a set which contains $193$, and all but finitely many primes. So, for any such neighborhood, there will be an $N$ such that $p_k \in U$ for all $k \geq N$. So, the sequence $x_k = p_k$ converges to $193$. But $193$ was arbitrary, so the sequence converges to any natural number. 

The problem here is that the open sets are too big. The Hausdorff condition will get us around this. 

\defn

A topological space $(X, \ms{T})$, is said to be \underline{Hausdorff} if, for any $x, y \in X$, when $x \neq y$, there exists $U_x, U_y \in \ms{T}$, such that $x \in U_x$, $y \in U_y$, and $U_x \cap U_y = \varnothing$. 

\lem


\begin{enumerate}[label=(\alph*)]
\item Metric spaces are Hausdorff
\item If $(x_n)$ has a limit in a Hausdorff space $(X, \ms{T})$, then it's unique 
\end{enumerate}

\proof
\begin{enumerate}[label=(\alph*)]
\item Pick $x \neq y$ in $(X, d)$ with $d(x, y) = \varepsilon>0$. Consider $B_{\frac{\varepsilon}{3}}(x)$ and $B_{\frac{\varepsilon}{3}}(y)$. These obviously are disjoint by the triangle inequality.
\item Let $x_n$ converge to $x$, and let $y \neq x$ be some other point besides $x$. Then, there is some neighborhood of $x$ which is disjoint from a neighborhood of $y$. Eventually, every $x_k$ is in this neighborhood, meaning none are in the neighborhood of $y$. Thus, $x_n$ cannot converge to $y$.  
\end{enumerate}

\thm

\begin{enumerate}[label = (\alph*)]
\item Suppose $X, Y$ are Hausdorff spaces. Then $X\times Y$ is also Hausdorff. 
\item If $A \subseteq X$, and $X$ is Hausdorff, then $A$ is also Hausdorff under the subspace topology. 
\item If $(X, \ms{T}_X)$ is Hausdorff, and $X$ is homeomorphic to $(Y, \ms{T}_Y)$, then $Y$ is also Hausdorff. 
\end{enumerate}

\defn

Let $(X, \ms{T}_X), (Y, \ms{T}_Y)$ be topological spaces. $X$ and $Y$ are said to be \underline{homeomorphic} if there exist continuous maps $f:X\to Y$, $g:Y\to X$, such that $g \circ f = \Id_X$ and $f \circ g = \Id_Y$.




















\end{document}