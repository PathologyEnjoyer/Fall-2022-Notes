\documentclass[x11names,reqno,14pt]{extarticle}
% Choomno Moos
% Portland State University
% Choom@pdx.edu


%% stupid experiment %%
%%%%%%%%%%%%% PACKAGES %%%%%%%%%%%%%

%%%% SYMBOLS AND MATH %%%%
\let\oldvec\vec
\usepackage{authblk}	% author block customization
\usepackage{microtype}	% makes stuff look real nice
\usepackage{amssymb} 	% math symbols
\usepackage{siunitx} 	% for SI units, and the degree symbol
\usepackage{mathrsfs}	% provides script fonts like mathscr
\usepackage{mathtools}	% extension to amsmath, also loads amsmath
\usepackage{esint}		% extended set of integrals
\mathtoolsset{showonlyrefs} % equation numbers only shown when referenced
\usepackage{amsthm}		% theorem environments
\usepackage{relsize}	%font size commands
\usepackage{bm}			% provides bold math
\usepackage{bbm}		% for blackboard bold 1

%%%% FIGURES %%%%
\usepackage{graphicx} % for including pictures
\usepackage{float} % allows [H] option on figures, so that they appear where they are typed in code
\usepackage{caption}
\usepackage{hyperref}
%\usepackage{titling}
\usepackage{tikz} % for drawing
\usetikzlibrary{shapes,arrows,chains,positioning,cd,decorations.pathreplacing,decorations.markings,hobby,knots,braids}
\usepackage{subcaption}	% subfigure environment in figures

%%%% MISC %%%%
\usepackage{enumitem} % for lists and itemizations
\setlist[enumerate]{leftmargin=*,label=\bf \arabic*.}

\usepackage{multicol}
\usepackage{multirow}
\usepackage{url}
\usepackage[symbol]{footmisc}
\renewcommand{\thefootnote}{\fnsymbol{footnote}}
\usepackage{lastpage} % provides the total number of pages for the "X of LastPage" page numbering
\usepackage{fancyhdr}
\usepackage{manfnt}
\usepackage{nicefrac}
%\usepackage{fontspec}
%\usepackage{polyglossia}
%\setmainlanguage{english}
%\setotherlanguages{khmer}
%\newfontfamily\khmerfont[Script=Khmer]{Khmer Busra}

%%% Khmer script commands for math %%%
%\newcommand{\ka}{\text{\textkhmer{ក}}}
%\newcommand{\ko}{\text{\textkhmer{ត}}}
%\newcommand{\kha}{\text{\textkhmer{ខ}}}

%\usepackage[
%backend=biber,
% numeric
%style=numeric,
% APA
%bibstyle=apa,
%citestyle=authoryear,
%]{biblatex}

\usepackage[explicit]{titlesec}
%%%%%%%% SOME CODE FOR REDECLARING %%%%%%%%%%

\makeatletter
\newcommand\RedeclareMathOperator{%
	\@ifstar{\def\rmo@s{m}\rmo@redeclare}{\def\rmo@s{o}\rmo@redeclare}%
}
% this is taken from \renew@command
\newcommand\rmo@redeclare[2]{%
	\begingroup \escapechar\m@ne\xdef\@gtempa{{\string#1}}\endgroup
	\expandafter\@ifundefined\@gtempa
	{\@latex@error{\noexpand#1undefined}\@ehc}%
	\relax
	\expandafter\rmo@declmathop\rmo@s{#1}{#2}}
% This is just \@declmathop without \@ifdefinable
\newcommand\rmo@declmathop[3]{%
	\DeclareRobustCommand{#2}{\qopname\newmcodes@#1{#3}}%
}
\@onlypreamble\RedeclareMathOperator
\makeatother

\makeatletter
\newcommand*{\relrelbarsep}{.386ex}
\newcommand*{\relrelbar}{%
	\mathrel{%
		\mathpalette\@relrelbar\relrelbarsep
	}%
}
\newcommand*{\@relrelbar}[2]{%
	\raise#2\hbox to 0pt{$\m@th#1\relbar$\hss}%
	\lower#2\hbox{$\m@th#1\relbar$}%
}
\providecommand*{\rightrightarrowsfill@}{%
	\arrowfill@\relrelbar\relrelbar\rightrightarrows
}
\providecommand*{\leftleftarrowsfill@}{%
	\arrowfill@\leftleftarrows\relrelbar\relrelbar
}
\providecommand*{\xrightrightarrows}[2][]{%
	\ext@arrow 0359\rightrightarrowsfill@{#1}{#2}%
}
\providecommand*{\xleftleftarrows}[2][]{%
	\ext@arrow 3095\leftleftarrowsfill@{#1}{#2}%
}
\makeatother

%%%%%%%% NEW COMMANDS %%%%%%%%%%

% settings
\newcommand{\N}{\mathbb{N}}                     	% Natural numbers
\newcommand{\Z}{\mathbb{Z}}                     	% Integers
\newcommand{\Q}{\mathbb{Q}}                     	% Rationals
\newcommand{\R}{\mathbb{R}}                     	% Reals
\newcommand{\C}{\mathbb{C}}                     	% Complex numbers
\newcommand{\K}{\mathbb{K}}							% Scalars
\newcommand{\F}{\mathbb{F}}                     	% Arbitrary Field
\newcommand{\E}{\mathbb{E}}                     	% Euclidean topological space
\renewcommand{\H}{{\mathbb{H}}}                   	% Quaternions / Half space
\newcommand{\RP}{{\mathbb{RP}}}                       % Real projective space
\newcommand{\CP}{{\mathbb{CP}}}                       % Complex projective space
\newcommand{\Mat}{{\mathrm{Mat}}}						% Matrix ring
\newcommand{\M}{\mathcal{M}}
\newcommand{\GL}{{\mathrm{GL}}}
\newcommand{\SL}{{\mathrm{SL}}}

\newcommand{\tgl}{\mathfrak{gl}}
\newcommand{\tsl}{\mathfrak{sl}}                  % Lie algebras; i.e., tangent space of SO/SL/SU
\newcommand{\tso}{\mathfrak{so}}
\newcommand{\tsu}{\mathfrak{sl}}


% typography
\newcommand{\noi}{\noindent}						% Removes indent
\newcommand{\tbf}[1]{\textbf{#1}}					% Boldface
\newcommand{\mc}[1]{\mathcal{#1}}               	% Calligraphic
\newcommand{\ms}[1]{\mathscr{#1}}               	% Script
\newcommand{\mbb}[1]{\mathbb{#1}}               	% Blackboard bold


% (in)equalities
\newcommand{\eqdef}{\overset{\mathrm{def}}{=}}		% Definition equals
\newcommand{\sub}{\subseteq}						% Changes default symbol from proper to improper
\newcommand{\psub}{\subset}						% Preferred proper subset symbol

% Categories
\newcommand{\catname}[1]{{\text{\sffamily {#1}}}}

\newcommand{\Cat}{{\catname{C}}}
\newcommand{\cat}[1]{{\catname{\ifblank{#1}{C}{#1}}}}
\newcommand{\CAT}{{\catname{Cat}}}
\newcommand{\Set}{{\catname{Set}}}

\newcommand{\Top}{{\catname{Top}}}
\newcommand{\Met}{{\catname{Met}}}
\newcommand{\PL}{{\catname{PL}}}
\newcommand{\Man}{{\catname{Man}}}
\newcommand{\Diff}{{\catname{Diff}}}

\newcommand{\Grp}{{\catname{Grp}}}
\newcommand{\Grpd}{{\catname{Grpd}}}
\newcommand{\Ab}{{\catname{Ab}}}
\newcommand{\Ring}{{\catname{Ring}}}
\newcommand{\CRing}{{\catname{CRing}}}
\newcommand{\Mod}{{\mhyphen\catname{Mod}}}
\newcommand{\Alg}{{\mhyphen\catname{Alg}}}
\newcommand{\Field}{{\catname{Field}}}
\newcommand{\Vect}{{\catname{Vect}}}
\newcommand{\Hilb}{{\catname{Hilb}}}
\newcommand{\Ch}{{\catname{Ch}}}

\newcommand{\Hom}{{\mathrm{Hom}}}
\newcommand{\End}{{\mathrm{End}}}
\newcommand{\Aut}{{\mathrm{Aut}}}
\newcommand{\Obj}{{\mathrm{Obj}}}
\newcommand{\op}{{\mathrm{op}}}

% Norms, inner products
\delimitershortfall=-1sp
\newcommand{\widecdot}{\, \cdot \,}
\newcommand\emptyarg{{}\cdot{}}
\DeclarePairedDelimiterX{\norm}[1]{\Vert}{\Vert}{\ifblank{#1}{\emptyarg}{#1}}
\DeclarePairedDelimiterX{\abs}[1]\vert\vert{\ifblank{#1}{\emptyarg}{#1}}
\DeclarePairedDelimiterX\inn[1]\langle\rangle{\ifblank{#1}{\emptyarg,\emptyarg}{#1}}
\DeclarePairedDelimiterX\cur[1]\{\}{\ifblank{#1}{\emptyarg,\emptyarg}{#1}}
\DeclarePairedDelimiterX\pa[1](){\ifblank{#1}{\emptyarg}{#1}}
\DeclarePairedDelimiterX\brak[1][]{\ifblank{#1}{\emptyarg}{#1}}
\DeclarePairedDelimiterX{\an}[1]\langle\rangle{\ifblank{#1}{\emptyarg}{#1}}
\DeclarePairedDelimiterX{\bra}[1]\langle\vert{\ifblank{#1}{\emptyarg}{#1}}
\DeclarePairedDelimiterX{\ket}[1]\vert\rangle{\ifblank{#1}{\emptyarg}{#1}}

% mathmode text operators
\RedeclareMathOperator{\Re}{\operatorname{Re}}		% Real part
\RedeclareMathOperator{\Im}{\operatorname{Im}}		% Imaginary part
\DeclareMathOperator{\Stab}{\mathrm{Stab}}
\DeclareMathOperator{\Orb}{\mathrm{Orb}}
\DeclareMathOperator{\Id}{\mathrm{Id}}
\DeclareMathOperator{\vspan}{\mathrm{span}}			% Vector span
\DeclareMathOperator{\tr}{\mathrm{tr}}
\DeclareMathOperator{\adj}{\mathrm{adj}}
\DeclareMathOperator{\diag}{\mathrm{diag}}
\DeclareMathOperator{\eq}{\mathrm{eq}}
\DeclareMathOperator{\coeq}{\mathrm{coeq}}
\DeclareMathOperator{\coker}{\mathrm{coker}}
\DeclareMathOperator{\dom}{\mathrm{dom}}
\DeclareMathOperator{\cod}{\mathrm{codom}}
\DeclareMathOperator{\im}{\mathrm{im}}
\DeclareMathOperator{\Dim}{\mathrm{dim}}
\DeclareMathOperator{\codim}{\mathrm{codim}}
\DeclareMathOperator{\Sym}{\mathrm{Sym}}
\DeclareMathOperator{\lcm}{\mathrm{lcm}}
\DeclareMathOperator{\Inn}{\mathrm{Inn}}
\DeclareMathOperator{\sgn}{sgn}						% sgn operator
\DeclareMathOperator{\intr}{\text{int}}             % Interior
\DeclareMathOperator{\co}{\mathrm{co}}				% dual/convex Hull
\DeclareMathOperator{\Ann}{\mathrm{Ann}}
\DeclareMathOperator{\Tor}{\mathrm{Tor}}


% misc symbols
\newcommand{\divides}{\big\lvert}
\newcommand{\grad}{\nabla}
\newcommand{\veps}{\varepsilon}						% Preferred epsilon
\newcommand{\vphi}{\varphi}
\newcommand{\del}{\partial}							% Differential/Boundary
\renewcommand{\emptyset}{\text{\O}}					% Traditional emptyset symbol
\newcommand{\tril}{\triangleleft}					% Quandle operation
\newcommand{\nabt}{\widetilde{\nabla}}				% Contravariant derivative
\newcommand{\later}{$\textcolor{red}{\blacksquare}$}% Laziness indicator

% misc
\mathchardef\mhyphen="2D							% mathomode hyphen
\renewcommand{\mod}[1]{\ (\mathrm{mod}\ #1)}
\renewcommand{\bar}[1]{\overline{#1}}				% Closure/conjugate
\renewcommand\qedsymbol{$\blacksquare$} 			% Changes default qed in proof environment
%%%%% raised chi
\DeclareRobustCommand{\rchi}{{\mathpalette\irchi\relax}}
\newcommand{\irchi}[2]{\raisebox{\depth}{$#1\chi$}}
\newcommand\concat{+\kern-1.3ex+\kern0.8ex}

% Arrows
\newcommand{\weak}{\rightharpoonup}					% Weak convergence
\newcommand{\weakstar}{\overset{*}{\rightharpoonup}}% Weak-star convergence
\newcommand{\inclusion}{\hookrightarrow}			% Inclusion/injective map
\renewcommand{\natural}{\twoheadrightarrow}				% Natural map

% Environments
\theoremstyle{plain}
\newtheorem{thm}{Theorem}[section]
%\newtheorem{lem}[thm]{Lemma}
\newtheorem{lem}{Lemma}
\newtheorem*{lems}{Lemma}
\newtheorem{cor}[thm]{Corollary}
\newtheorem{prop}{Proposition}
\newtheorem*{claim}{Claim}
\newtheorem*{cors}{Corollary}
\newtheorem*{props}{Proposition}
\newtheorem*{conj}{Conjecture}

\theoremstyle{definition}
\newtheorem{defn}{Definition}[section]
\newtheorem*{defns}{Definition}
\newtheorem{exm}{Example}[section]
\newtheorem{exer}{Exercise}[section]

\theoremstyle{remark}
\newtheorem*{rem}{Remark}

\newtheorem*{solnx}{Solution}
\newenvironment{soln}
    {\pushQED{\qed}\renewcommand{\qedsymbol}{$\Diamond$}\solnx}
    {\popQED\endsolnx}%

% Macros
\newcommand{\restr}[1]{_{\mkern 1mu \vrule height 2ex\mkern2mu #1}}
\newcommand{\Upushout}[5]{
    \begin{tikzcd}[ampersand replacement = \&]
    \&#2\ar[rd,"\iota_{#2}"]\ar[rrd,bend left,"f"]\&\&\\
    #1\ar[ur,"#4"]\ar[dr,"#5"]\&\&#2\oplus_{#1} #3\ar[r,dashed,"\vphi"]\&Z\\
    \&#3\ar[ur,"\iota_{#3}"']\ar[rru,bend right,"g"']\&\&
    \end{tikzcd}
}
\newcommand{\exactshort}[5]{
		\begin{tikzcd}[ampersand replacement = \&]
			0\ar[r]\&#1\ar[r,"#2"]\& #3 \ar[r,"#4"]\& #5 \ar[r]\&0
		\end{tikzcd}
}
\newcommand{\product}[6]{
		\begin{tikzcd}[ampersand replacement = \&]
			#1 \& #2 \ar[l,"#4"'] \\
			#3 \ar[u,"#5"] \ar[ur,"#6"']
		\end{tikzcd}
}
\newcommand{\coproduct}[6]{
		\begin{tikzcd}[ampersand replacement = \&]
			#1 \ar[r,"#4"] \ar[d,"#5"'] \& #2 \ar[dl,"#6"] \\
			#3
		\end{tikzcd}
}
%%%%%%%%%%%% PAGE FORMATTING %%%%%%%%%

\usepackage{geometry}
    \geometry{
		left=15mm,
		right=15mm,
		top=15mm,
		bottom=15mm	
		}

\usepackage{color} % to do: change to xcolor
\usepackage{listings}
\lstset{
    basicstyle=\ttfamily,columns=fullflexible,keepspaces=true
}
\usepackage{setspace}
\usepackage{setspace}
\usepackage{mdframed}
\usepackage{booktabs}
\usepackage[document]{ragged2e}
\usepackage{amsmath}
\pagestyle{fancy}{
	\fancyhead[L]{Fall 2022}
	\fancyhead[C]{202A - Complex Analysis}
	\fancyhead[R]{John White}
  
  \fancyfoot[R]{\footnotesize Page \thepage \ of \pageref{LastPage}}
	\fancyfoot[C]{}
	}
\fancypagestyle{firststyle}{
     \fancyhead[L]{}
     \fancyhead[R]{}
     \fancyhead[C]{}
     \renewcommand{\headrulewidth}{0pt}
	\fancyfoot[R]{\footnotesize Page \thepage \ of \pageref{LastPage}}
}
\newcommand{\oo}{{\infty}}
\newcommand{\bigt}{\bigtriangleup}
\DeclareMathOperator{\ind}{ind}
\DeclareMathOperator{\res}{res}



\title{220A - Riemann Surfaces}
\author{John White}
\date{Fall 2022}

\begin{document}

\section*{Lecture 1}

We begin by trying to gain a deeper understanding of the Cauchy-Riemann equations. 

Let $f:X\to \C$, where $X \subset \C^n$. For now, let's say $X \subset \C$. In real analysis, we have a notion of differentiability for $f:\R^n\to\R^k$. We can say that $f$ is differentiable at a point $p \in X$ when
\[
f(p + h) = f(p) + (df_p)h + \rho(h)
\]
Where $(df)_p:\R^n\to\R^k$ is a linear map $\in \Hom_{\R}(\R^n,\R^k)$, and $\frac{|\rho(h)|}{|h|} \to 0$ as $h \to 0$. 

So we can think of the ``real differential" as a linear map in $\Hom_{\R}(\C^n,\C)$. 

\defn

Let $X \subset \C$, and $f:X\to\C$. Differentiability refers to the existence of a $(df)_p \in \Hom_{\R}(\C,\C)$. 

So, $f$ is complex differentiable at $p \in X$ means that 
\[
f(p + h) = f(p) + f'(p)h + \rho(h)
\]
Where $f'(p)$ is a complex number and $\frac{|\rho(h)|}{|h|}\to 0$. 

If $A \in \Hom_{\C}(\C,\C) \cong \C$, $A(z) = \alpha z, \alpha \in \C$. 

So $(df)_p \in \Hom_{\C}(\C,\C)$.

$\Hom_{\C}(\C,\C)$ is a $\C$-vector space of dimension 1. 

$\Hom_{\R}(\C, \C) \cong \Hom_{\R}(\R^2,\R^2)$ is a $\C$-vector space of dimension 2. 

So where did the extra dimension go? What happened? 

Consider an element of $\Hom_{\R}(\C,\C)$ given by $\begin{pmatrix} x \\ y \end{pmatrix} \mapsto x - iy$. 

We also have $\begin{pmatrix} x \\ y \end{pmatrix} \mapsto x + iy$. 

From the real analysis point of view, these two functions are equal to their differentials. The first is called $d\bar{z}$, and the second is called $dz$.

$dz = dx + idy$ and
$d\bar{z} = dx - idy$
\begin{align*}
dx\begin{pmatrix} h_1 \\ h_2 \end{pmatrix} & = h_1 \\
dy\begin{pmatrix} h_1 \\ h_2 \end{pmatrix} & = h_2 \\
\end{align*}

On a complex vector space, suppose $\phi\in\Hom_{\mathbb{K}}(\C^n,\C)$, we have $(\bar{\phi})(v) = \bar{\phi(v)}.$ So $\bar{dz} = d\bar{z}$. 

Now, $\C$-valued real differentiable functions are just pairs of $\R$-valued real differentiable functions. 

\exm

If $k, m \in \N$, then $z^k\bar{z}^m:\C\to\C$ is a \underline{real} smooth function (when viewed as an element of $\Hom_{\R}(\R^2, \R^2)$), with

\[
d(z^k\bar{z}^m) = kz^{k - 1}\bar{z}^m + m\bar{z}^{m - 1}z^kd\bar{z}
\]

We will study the differences between $\Hom_{\R}(\C^n, \C)$ versus $\Hom_{\C}(\C^n, \C)$, with complex dimensions $2$ and $1$, respectively. 

\defn Let $V$ be a real vector space. 

A \underline{complex structure} on $V$ is a $J \in \End_{\R}(V)$ which satisfies $J^2 = -\operatorname{Id}_V$

\prop Define $V_J = V$ as a set and group, with a $\C$-action $\C\times V_J \to V_J$ defined by $((\alpha + i\beta), x) \mapsto \alpha x + \beta Jx.$

\proof Check $z(wx) = (zw)x$ for all $z, w \in \C$ and $x \in V_J$. 

\prop If a vector space $V$ admits a complex structure $J$, then $\dim_\R V = 2n$. Further, $\dim_\R V = 2\dim_\C V_J$. 

\proof 

First, $\det(J^2) = \det(-\Id_V) = (-1)^{\dim_\R V}$, so the dimension must be even. Alternatively, if $e_1, \dots, e_n$ is a basis of $V_J$, then check $e_1, \dots, e_n, Je_1, \dots, Je_n$ is a basis of $V$ over $R$.

\exm For $\R^2$, let $J_0 = \begin{pmatrix} 0 & - 1\\ 1 & 0 \\ \end{pmatrix}$. We see that $J_0\begin{pmatrix}x \\ y\end{pmatrix} = \begin{pmatrix} -y \\ x \end{pmatrix}$.  This is like $i(x + iy) = ix - y$.

So $A: (\R^2)_{J_0} \to \C$ is an isomorphism of $\C$-vector spaces. 

Let $W$ be a vector space over $\C$. Consider $W_{\R}$, a real vector space. We see $\dim_{\R}W_{\R} = 2\dim_{\C}W$. Consider $J:W_\R\to W\R$ given by $x \mapsto ix$. Then $J^2 = -\Id_{W_\R}$. 

Let $V$ be a real vector space with complex structure $J$. Consider $\Hom_{\R}(V, \C) =\C\otimes_{\R} V^*$. 

$J^t: \Hom_{\R}(V, \C)\to\Hom_{\R}(V, \C)$, we can express $\Hom_{\R}(V,\C) \ni \phi = \phi_1 + \phi_2$, and by definition, 
\[
J^t \phi = \phi \circ J = \phi_1 \circ J + i\phi_2 \circ J
\]
So $(J^t)^2 = -1$. 

$J^t \in \End_{\C}(\Hom_{\R}(V, \C))$. 

Main observation: $\phi \in \Hom_\R(V, \C)$ is $\C$-linear in $V_j$, meaning $\phi(ix) = i\phi(x)$, which is equivalent to $\phi(Jx) = i\phi(x)$. 

In other words, such a $\phi$ is only $\C$-linear if $\phi$ is an eigenfunction of $J^t$ with eigenvalue $i$. 

\defn $\phi$ is $\C$-antilinear on $V_J$ means 
\[
\phi((\alpha + i\beta)x) = \bar{(\alpha + i\beta)}\phi(x)
\]
for all $x \in V$. 

We denote the space of $\C$-antilinear functionals by $\bar{\Hom}_{\C}(V_J,\C)$. 

In fact, there is an isomorphism between $\Hom_{\C}(V_J,\C)$ and $\bar{\Hom}_{\C}(V_J, \C)$ as \underline{real} vector spaces. 

\thm 

Let $V$ be a real vector space with complex structure $J$. Then
\begin{enumerate}
\item $\Hom_{\R}(V, \C) = \Hom_{\C}(V_J, \C)\oplus\bar{\Hom}_{\C}(V_J, \C)$. 
\item If $\Hom_{\C}(V_J, \C) := V^{1, 0}$, and $\bar{\Hom}_{\C}(V_J, \C) := V^{0, 1}$, then $\Hom_{\R}(V,\C) = V^{1, 0} \oplus_{\C}V^{0, 1}$. 
\item $\dim_{\C}V^{1, 0} = \dim_{\C}V^{0, 1} = \frac{\dim_{\C}(\Hom_{\R}(V, \C))}{2}$
\end{enumerate}

\proof

Observe that $\phi \in \Hom_{\R}(V, \C)$ can be written as 
\[
\phi = \frac{\phi -i\phi\circ J}{2} + \frac{\phi + i\phi\circ J}{2} = \frac{\phi(Jx) + i\phi(x)}{2} = i\frac{\phi - i\phi\circ J}{2}(x) = \phi
\]

Further, $V^{1, 0} \cap V^{0, 1} = 0$ by the definitions, so we are done. 

\qed

Thus, any differential can be split into a $\C$-linear and a $\C$-antilinear part. 

\defn

$\pi^{1, 0}$ is projection on the first factor, $\pi^{0, 1}$ is projection onto the second. We have
\[
\phi = \pi^{1, 0}\phi + \pi^{0, 1}\phi
\]

\cor

If $\phi \in \Hom_{\R}(V,\C)$, then $\phi$ being $\C$-linear (i.e $\phi \in V^{1, 0}$) if and only if $\pi^{0, 1}\phi = 0$. 

\defn

Applying to $(df)_p \in \Hom_{\R}(V, \C)$, then 
\[
(df)_p = \pi^{1, 0}df_p + \pi^{0, 1}df_p
\]
Say $\pi^{1, 0}df_p = \underbrace{\partial f_p}_{\text{ complex linear}}$ and $\pi^{0, 1}df_p = \underbrace{\bar{\partial}df_p}_{\text{ complex antilinear}}$

\thm A function $f:X\to\C$ is $\C$-differentiable at $p \in X$ if and only if $f$ is $\R$-differentiable at $p$ and $df_p = \partial f_p$, which happens if and only if $\bar{\partial}f_p = 0$. 

\proof

We have $\C \cong \R^2_{J_0}$, which has standard basis $\R^2 = \langle e_1, e_2\rangle_{\R^2}$. This has a dual basis in $\Hom_{\R}(\R^2, \R)$ given by $dx$ and $dy$. That is, $\Hom_{\R}(\R^2, \C) = \langle dx, dy \rangle_{\C}$. 

$J_0e_1 = e_2$ and $J_0e_2 = -e_1$, so $dx \circ J_0 = -dy$ and $dy \circ J_0 = dx$. 

\begin{align*}
\pi^{0, 1}dx & =\frac{1}{2}(dx - idx\circ J_0)\\
				 & = \frac{1}{2}(dx + idy) \eqdef dz \\
\end{align*}

Further, 
\begin{align*}
\pi^{0, 1}dx & = \frac{1}{2}d\bar{z} \\
\pi^{1, 0}dy & = \frac{1}{2}dz \\
\end{align*}

So 
\[
df = f_xdx + f_ydy = \frac{f_x - if_y}{2}dz + \frac{f_x + if_y}d\bar{z} = \partial f + \bar{\partial} f
\]

\defn

\begin{align*}
\frac{\partial}{\partial z} & = \frac{\partial_x - i\partial_y}{2} \\
\frac{\partial}{\partial \bar{z}} & = \frac{\partial_x + i\partial_y}{2} \\
\end{align*}

So 

\[
df = \frac{\partial f}{\partial z}dz + \frac{\partial f}{\partial \bar{z}}d\bar{z}
\]

So analyticity is equivalent to $df = \partial f$, meaning $\bar{\partial}f = 0$, which is equivalent to $\frac{\partial f}{\partial \bar{z}} = 0$, which means
\[
\frac{\partial (u + iv)}{\partial\bar{z}} = 0
\]
So $(\partial_x + i\partial_y)(u + iv) = 0$. Multiplying out, we get
\begin{align*}
u_x & = v_y \\
u_y & = - v_x \\
\end{align*}

\section*{Lecture 2} 

The focus for the first bit of this course will be the so-called (by Dennis) $\bar{\del}$-calculus. 

Suppose $f:X\to\C$ is differentiable for some $X \subseteq \R^{2n}$. It has a differential $df_p \in \Hom_\R(\R^{2n},\C)$.

$f$ is holomorphic if and only if $df_p\in\Hom_{\C}(\C^n, \C)$. Last time, we talked about how the second sits in the first, and how they interact. 

Question: How to make $\C^n$ out of $\R^{2n}$. The abstract algebra way to do it is with a complex structure $J$. If $V$ is a vector space over $\R$, then $\dim_\R V = 2n$
$J\in\End_{\R}(V)$ with $J^2 = -\Id_V$. 

For all $x \in V_J$, we define $ix = Jx$, so $V_J$ is a vector space over $\C$, and $\dim_\C V_j = \frac{\dim_\R V}{2}$

\exm Let $V = \R^2$, $J = J_0 = \begin{pmatrix} 0 & -1 \\ 1 & 0 \\ \end{pmatrix}$. $\R^2_{J_0}\cong \C$

Last time, we showed that for any $\phi\in\Hom_\R(V,\C) = H$, then 
\[
\phi = \underbrace{\frac{\phi - i\phi\circ J}{2}}_{\in \Hom_\C(V_J,\C)} + \underbrace{\frac{\phi + i\phi\circ J}{2}}_{\in\bar{\Hom_\C}(V_J,\C)}
\]

So, $H = \Hom_\C(V_J,\C)\oplus\bar{\Hom_\C(V_J,\C)} = V^{1, 0}\oplus V^{0, 1}$

Where $V^{1, 0}$ and $V^{0, 1}$ are what we call $\Hom_\C(V_J,\C)$ and $\bar{\Hom}_\C(V_J,\C)$ by tradition. 

Let $V = \R^2$, $J = J_0 = \begin{pmatrix} 0 & -1 \\ 1 & 0 \\ \end{pmatrix}$. 

Let $\phi = df_p = \frac{\del f(p)}{\del x}dx_p + \frac{\del f(p)}{\del y}dy_p$

After doing some computations, we get

\[
d_pf = \pi^{1, 0}df + \pi^{0, 1}df = Adz + Bd\bar{z}
\]

Where $dz = dx + idy$ and $d\bar{z} = dx - idy$. You can check that the former is in $V^{1, 0}$ and the latter in $V^{0, 1}$. 

The coefficient $A$ is denoted by tradition as $\frac{\del f}{\del z}(p)$, and $B$ as $\frac{\del f}{\del \bar{z}}(p)$. 

\underline{Here, the presence of $\del$ does not imply any limit taking or anything, they are just notation.}

Note $A = \frac{1}{2}(\del_xf - i\del_yf)|_p$ and $B = \frac{1}{2}(\del_xf + i\del_yf)|_p$.

\defn

We define 
\begin{align*}
\del f_p & \eqdef f_z(p)dz_p \\
\bar{\del}f_p & \eqdef f_{\bar{z}}(p)d\bar{z}_p \\
\end{align*}

The former is $\C$-linear, and the second $\C$-antilinear. 

\claim

$f$ is $\C$-differentiable at $p$ if and only if $f$ is $\R$-differentiable at $p$ and $\bar{\del}f_p = 0$

\proof

If $f = u + iv$, we get
\[
\bar{\del}f_p = \frac{\del f}{\del\bar{z}}(p) =0
\]
Which gives you the Cauchy-Riemann equations. 

So $f$ is analytic if $d_pf = f_zdz$. 

\exm

What are the following? 
\begin{enumerate}
\item $\frac{\del|z|}{\del z}$
\item $\frac{\del|z|}{\del\bar{z}}$
\end{enumerate}

How do we manage these problems?

\claim If $m, k \in \Z\backslash\{0\}$, we will consider $d(z^m\bar{z}^k)$. We have $f(z + h) = (z + h)^m(\bar{z} + \bar{h})^m = f(z) + (mz^{m-1}\bar{z}^k)h + (k\bar{z}^{k - 1}z^m)\bar{h} + \mc{O}(h^2)$

Where $\frac{|LHS - RHS|}{|z - 2|} \to 0$ as $z\to 2$

Then $\bar{\del}(z^m\bar{z}^k) = (k\bar{z}^{k - 1}z^m)d\bar{z}$

\proof



Let's do some examples. Consider $\frac{\bar{z} - 1}{z + 1}$. We have

\begin{align*}
\frac{\bar{z} - 1}{z + 1}  & = \frac{(\bar{z} - 2) + 2 - 1}{(z - 2) + 2 + 1}  \\ & = \frac{(\bar{z} - 2) + 1}{3}\frac{1}{1 + \frac{z - 2}{3}}\\ &  = \frac{\bar{z} - 1}{3}\left(1 - \frac{z - 2}{3} + \mc{O}(H^2)\right)\\ & = c + Az + B\bar{z} + \rho(z)
\end{align*}

There are two building blocks for doing problems: 
\begin{enumerate}
\item First, remember you are really doing real analysis. 
\item Use the formula $d_pf = \pi^{1, 0}\cdots $
\end{enumerate}

\exm

We will calculate $d|z| = d\sqrt{z\bar{z}}$.

\begin{align*}
d|z| & = d\sqrt{z\bar{z}}\\
& = \frac{1}{2\sqrt{z\bar{z}}}d(z\bar{z})\\
& = \frac{1}{2|z|}d(z\bar{z})\\ & = \frac{1}{2|z|}zd\bar{z} + \bar{z}dz \\
\end{align*}

So the answer would be $\frac{z}{2|z|}$. 

So, just express your function as a function of $z\bar{z}$ and proceed to do real analysis. 

We know $\del_zf, \del_{\bar{z}}f$, and want to find $\del_z(\bar{f}), \del_{\bar{z}}(\bar{f}) = ?$

We have

\begin{align*}
df = \del f & + \bar{\del}f \\
d(\bar{f})  &= \bar{df} = \bar{(\del f)} + \bar{(\bar{\del}f)} \\
\end{align*}

Now, $\bar{\del}(\bar{f}) = \bar{\del f}$. The bottom is equal to $\frac{\del \bar{f}}{\del\bar{z}}d\bar{z}$

So 

\[
\frac{\del \bar{f}}{\del \bar{z}} = \bar{\frac{\del f}{\del z}}
\]
The conjugate of something which is complex anti-linear is complex linear. 

What is $\frac{\del \bar{f}}{\del z}$? This is $\bar{\frac{\del f}{\del \bar{z}}}$. 

So, the general procedure is to decompose your function as something linear + something antilinear, and use the sentence I just wrote. 

How to compute $\frac{\del f\circ g}{\del \bar{z}}$ ?

Well $\underbrace{d(f\circ g) = df \circ df}_{\text{ The chain rule expresses the functoriality of the derivative}}$, which is equal to 

$(\del f + \bar{\del}f)\circ(\del g + \bar{\del}g) = \del f \circ \del g + \bar{\del f}\circ dg + \cdots$.

Now, $\bar{\del}f\circ\del g = f_{\bar{z}}d\bar{z}\circ(g_zdz) = f_{\bar{z}}\bar{g_z}$. 

\defn

Suppose $\frac{\del f}{\del \bar{z}}(p) = 0$. Then $df_p = \frac{\del f}{\del z}(p)dz_p$, and we write this as $f'(p)dz$. 

Homework problem: We know $f$ holomorphic, compute $\frac{\del}{\del}{\bar{z}}F(|f|)$, where $F:\R\to\R$ is smooth. 

So 

\[
dF(|f|) = F'(|f|)d|f| = F'(|f|)d\sqrt{f\bar{f}}
\]

And $d\sqrt{u} = \frac{1}{2u}du$, so the above is equal to 

\[
F'(|f|)\frac{1}{2|f|}d(f\bar{f})^{-1} = \frac{F'(|f|)}{2|f|}(\bar{f}df + fd\bar{z}) = \frac{F'(|f|)}{2|f|}(\bar{f}f_zdz + f\bar{f_z}d\bar{z})
\]

Our answer is thus whatever we get in front of $d\bar{z}$, so in this case the solution is 

\[
\frac{\del}{\del\bar{z}}F(|f|) = \frac{F'(|f|)}{2|f|}f\bar{f'}
\]

The $|z| = \sqrt{z\bar{z}}$ is a very useful trick. 

Complex analysis is kind of a local study of $f:X\to\C$ for some $X\subseteq\C$, where $f$ is differentiable, $\bar{\del}f = 0$, or equivalently $\frac{\del f}{\del\bar{z}} = 0$ in $X$. Really, we are studying solutions to a certain PDE (Cauchy-Riemann equation). 

Suppose you want $u_{xx} - u_{yy} = 0$. If this is the case (and it turns out exactly when this is the case), we can express $u = \phi(x - y) + \psi(x + y)$ for $\phi, \psi$ arbitrary of $1$ variable. 

What if we want to study $u_y = u_{xx}$ in, say, $y > -\varepsilon$? We actually have a formula:

\[
u(x, y) = \frac{1}{2\pi y}\int_{-\oo}^{\oo} e^{-\frac{(x - s)^2}{4y}}u(s, 0)ds, y > 0
\]

Suppose we want to study $u_{yy} + u_{xx} = 0$ in $y > -\varepsilon$, or maybe an open ball around the origin. We have a formula
\[
u(x, y) = \int_{-\oo}^\oo P_H(x, y - s)u(s, 0)ds
\]

Where $P_H$ is the Poisson Kernel. 

\subsection*{\underline{MAIN LOCAL THM}}
Let $f:B_{R + \varepsilon}\to \C$. Then the following statements are all equivalent. 
\begin{enumerate}
\item For any $p \in B_R$, $f$ is differentiable and $df_p = \del f_p$, so $\bar{\del}f_p = 0$.
\item $f(z) = \frac{1}{2\pi i}\int_{C_R}\frac{f(w)}{z - w}dw$
\item Like 8 other things
\end{enumerate}

What exactly is complex integration? 

Let $\gamma:[a, b]\to\C$ be a piecewise smooth map. We integrate functions over maps. If $\phi$ is a continuous function, the formula is 

\[
\int_\gamma\phi(z)dz = \int_a^b\phi(\gamma(t))\dot{\gamma}(t)dt
\]

where $\dot{\gamma}$ is the time-derivative of $\gamma$, so will be a complex number. 

Let $t \in [0, 2\pi]$, $C_R(t) = Re^{it}$, a circle going counterclockwise. Let $\gamma(t) = Re^{-it}$. Then $\gamma^{-1}:[a, b]\to\C$ is the same but in the opposite direction. We have

\[
\int_{\gamma^{-1}}\phi\,dz = \int_{\gamma}\phi\,dz
\]

\section*{Lecture 3}

We continue with the Local Theorem for analytic functions. Let $f:B_{R_0 + \varepsilon}\to \C$. Then the following statements are all equivalent. 
\begin{enumerate}[label=(\roman*)]
\item $\forall z \exists$ a finite $\lim_{\Delta z\to\oo}\frac{f(z + \Delta z) - f(z)}{\Delta z} \eqdef f'(z)$. 
\item $\forall z$, $f$ is $\R$-differentiable at $z$ and $df_z = \del f_z$, which is equivalent to $\bar{\del}f_z \equiv 0$, which gives the Cauchy-Riemann equations. 
\item $f$ is continuous, and for all $R < R_0$, for all $z$ such that $|z - a| < R$, then 
\[
f(z) = \frac{1}{2\pi i}\int_{C_R(a)}\frac{f(w)}{w - z}dw
\]
\item $f(z) = \sum_{n=0}^\oo a_n(z - a)^n$ in $B_R(a)$ with $a_n = \frac{1}{2\pi i}\int_{C_R(a)}\frac{f(w)}{(w - a)^{n + 1}}dw$ for all $R < R_0$. 
\item For some coefficient $c_n$, $f(z) = \sum_{n=0}^\oo c_n(z - a)^n$ for all $|z - a|< R_0$ for some $c_n$. 
\item For all $z$, $\exists f'(z), f''(z), \dots, f^{(n)}(z), \dots$. Moreover, $f(z) = \sum_{n=0}^\oo c_n(z - a)^n $ for all $|z - a| < R_0$ with 
\[
c_n = \frac{f^{(n)}(a)}{n!}
\]
and
\[
f'(z) = \sum_{n=1}^\oo nc_n(z - a)^{n - 1}
\]

for all $|z - a| < R_0$. 
\end{enumerate}

\proof

We already proved (i) and (ii) equivalent. $(ii)\implies(iii)$ will come later, and the steps $(iv)\implies(v)\implies(vi)$ are proven by undergrad power series techniques. Then of course $(vi)\implies(i)$ is a triviality. 

\claim $(iii)\implies(iv)$. 

\proof

Recall the setup: $(iii)$ holds in a disk of radius $R$, which is $\delta$ less than $R_0$. 

We have $f(z) = \frac{1}{2\pi i}\int_{C_R(a)}\frac{f(w)}{z - w}dw$. Now, 
\[
\frac{1}{w-z} = \frac{1}{(w - z) - (z - a)} = \frac{1}{w - a}\cdot\frac{1}{1 - \frac{z - a}{w - a}} = \sum_{n=0}^\oo\frac{(z - a)^n}{(w - a)^{n + 1}}
\]

This works because $|z - a| < |w - a|$. So 

\[
f(z) = \frac{1}{2\pi i}\int_{C_R(a)}\left(\sum_{n=0}^\oo(z - a)^n\frac{f(w)}{(w - a)^{n + 1}}\right)dw
\]

We want to swap the integral and the summation, and then we will be done. We can swap an integral and a sum provided the sum converges uniformly on $[a, b]$. We will show that this sum converges uniformly using the $M$ test. 

Let $M_n = \sup_{[a, b]}|f_n|$. Then if $\sum_{n\in\N}M_n$ converges, then $\sum_{n\in\N}f_n$ converges uniformly . 

We have to calculate $\sup_{|w - a| = R}|(z - a)^n\frac{f(w)}{(w - a)^{n + 1}}| = M_n$. We find
\[
M_n = \frac{(R - \delta)^n}{R^{n + 1}}\left(\frac{\sup_{\bar{B_R(a)}}|f|}{R}\right)
\]
so $\sum_{n\in\N}M_n<\oo$

\qed

We make use of the trick where $\frac{1}{A + B} = \frac{1}{A}\cdots\frac{1}{1 + \frac{B}{A}}$, and if $\frac{B}{A}$ is small we can do a series expansion. 

Now, we will prove the $\bigtriangleup$ inequality for complex integrals.

Let $\gamma:[a, b]\to\C$ be $C^1$ or piecewise smooth. , and let $\phi:\C\to\C$ be continuous. By definition, we have
\begin{align*}
\abs*{\int_\gamma f(z)\,dz} & = \int_a^b\phi(\gamma(t))\dot{\gamma}(t)\,dt \\
							 & \leq \int_a^b|\phi(\gamma(t))||\dot{\gamma}(t)|\,dt \\
							 & =: \int_\gamma |\phi(z)||dz| \\
\end{align*}

We have used the fact that for continuous real functions $f:\R\to\R$ we have 
\[
\abs*{\int_a^b f(t)\,dt} \leq \int_a^b|f(t)|\,dt
\]

We are moving towards proving $(ii) \implies (iii)$. The next result we need \underline{Goursat's Lemma}. 

\defn

Let $X \subset \C$ be open. If $f'(z)$ exists for all $z \in X$, we say $f \in A(X)$ or $f \in \ms{H}(X)$. 

\lem (Goursat's Lemma)

Let a solid $\bigtriangleup \subset \Omega\subset \C$, with $\Omega$ open ($\bigtriangleup$ is a $2$-simplex). 

Denote the boundary of $\bigtriangleup$ by $T$, with the counter-clockwise orientation. Then for any $f \in \ms{H}(\Omega)$, we have
\[
\int_Tf\,dz = 0
\]

\proof

Recall the Cauchy-Riemann system 
\begin{align*}
u_x & = v_y \\
u_y & = -v_x \\
\end{align*}

We have from Calc 3 the classic Green's theorem: 
\[
\int_{\del\Omega}Pdx + Qdy = \int\int_{\Omega}(\text{something})dxdy
\]

Using these together will give the proof. 

\subsection*{Step 1}

First, we will seek a contradiction. Suppose that there is some $\varepsilon_0>0$ such that 
\[
\abs*{\int_Tf\,dz}\geq\varepsilon_0
\]

We can perform barycentric subdivision to $\bigtriangleup$ (pictures forthcoming until I figure out how to insert them in TeXwriter), and we can express this integral as the sum of the integrals of four sub-simplices, $T_1,\dots,T_4$. We must have $|\int_{T_i}f\,dz|\geq \frac{\varepsilon_0}{4}$. Denote this sub-simplex by $T_1$, and it's interior by $\bigtriangleup_1$. 

So we have
\[
\abs*{\int_{T_1}f\,dz}\geq\frac{\varepsilon_0}{4}
\]
Now, $\operatorname{length}(T_1) = \frac{\operatorname{length}(T)}{2} = \frac{c_0}{2}$, and $\operatorname{diam}(\bigtriangleup_1) = \frac{\operatorname{diam}(\bigtriangleup)^2}{2} = \frac{c_1}{2}$.

We can keep proceeding by doing barycentric subdivision to $T_1$. 

So we have a sequence $\bigt_1 \supset \bigt_2 \supset \dots $, with $\del \bigt_i = T_i$, with $|\int_{T_i}f\,dz|\geq\frac{\varepsilon_0}{4^i}$. Further, the length of $T_j$ is $\frac{c_0}{2^j}$, and the diameter of $\bigt_j$ is $\frac{c_1}{2^j}$. 

We can see that $\cap_{j=1}^\oo\bigt_j = \{p\}$, and without loss of generality we can assume $p = 0 \in \Omega$. 

\subsection*{Step 2}

We have $f(z) = f(0) + f'(0)z + \rho(z)$, where $\frac{|\rho(z)|}{|z|}\to 0$. So
\[
\int_\gamma 1\,dz = \int_a^b1\cdot\dot{\gamma}(t)dt = \gamma(b) - \gamma(a)
\]
So
\[
\int_\gamma z\,dz = \int_a^b\gamma(t)\dot{\gamma}(t)\,dt = \frac{1}{2}\int_a^b(\dot{(\gamma(t)^2)})\,dt = \frac{\gamma(b)^2 - \gamma(a)^2}{2}
\]

So
\[
\int_{T_j}f\,dz = \int_{T_j}f(0)\,dz + \int_{T_j}f'(0)\,dz + \int_{T_j}\rho(z)\,dz = 0 + 0 + \abs*{\int_{T_j}\rho(z)\,dz}
\]

So 
\begin{align*}
\abs*{\int_{T_j}f\,dz} & \leq \int_{T_j}|\rho(z)||dz| = \int_{T_j}\frac{|\rho(z)|}{|z|}|z||dz| \\
						& \leq \sup_{\bigt_j}\frac{|\rho(z)|}{|z|}\operatorname{diam}(\bigt_j)\operatorname{length}(T_j) \\
\end{align*}
But $\frac{\varepsilon_0}{c_0c_1 4^i}\leq \sup_{bigt_j}\frac{|\rho(z)|}{|z|}\frac{c_0c_1}{4^j}$, which for large $d$ contradicts our assumption that $\frac{|\rho(z)|}{|z|}\to0$, proving Gousat's Lemma. 

Now that we have Goursat's Lemma, we can prove $(ii)\implies(iii)$. 

\prop $(ii)\implies(iii)$. 

\proof



\subsection*{Step 1}


By Goursat, for $F \in \ms{H}(B_{R + \varepsilon}(a))$, $\int_{C_R(a)}F\,dw = \int_{C_\varepsilon(z)}F\,dw$ when $z \in B_{R}(a)$. We will show this in step 2. 

Now, consider the map $w\mapsto \frac{f(w)}{w - z}$. We have

\[
\int_{C_R(a)}\frac{f(w)}{w - z}\,dz = \int_{C_\varepsilon(z)}\frac{f(w)}{w - z}\,dz = \int_0^{2\pi}\frac{f(z + \varepsilon e^{it})}{\varepsilon e^{it}}\,dz = i\int_0^{2\pi}f(z + \varepsilon e^{it})\,dt = if(z)2\pi i
\]

\section*{Lecture 4}

Suppose we have a function $F$, which is holomorphic in $B_a(R+ \varepsilon)\backslash\{z\}$. $F(w) = \frac{f(w)}{w - z}$, with $f \in \ms{H}(B_{R+\varepsilon}(a)$. We have

\[
\int_{C_R(a)}F(w)\,dw =_* \int_{C_\varepsilon(z)}F(w)\,dw
\]

Once we have established this, we can take the limit as $\varepsilon\to0$ to get the Cauchy formula $f(z) = \cdots$. 

We need some facts to prove $*$

\subsubsection*{\underline{Fact 1}} 

Suppose a circle has $N$ points distributed on it's circumference. Any two adjacent points forms an angle with the center. Let $\varphi_N$ be the max of these angles. Note $\varphi_N\to0$ as $N\to\oo$. Let $L_n$ be the polygonal curve formed by the secants between adjacent points. Then 
\[
\int_{L_n}\phi(w)\, dw \to \int_{C_r(a)}\phi(w)\,dw
\]
for $\phi$ continuous. The proof is easy, just break up the integrals and use uniform convergence. 

\subsubsection*{\underline{Fact 2}}

Let $A_1, A_2, A_3, A_4$ be points in a a ball, and consider the loop going from $A_1\to A_2$, then $A_2\to A_3$, then $A_3\to A_4$, and finally $A_4\to A_1$. 

Suppose that $z$ is not in the convex hull of $\{A_1, A_2, A_3, A_4\}$. 

Then $\int_{A_1\to\cdots\to A_1}F(w)\,dw = 0$. We calculate this by breaking this loop up into the sum of two simplices, and the integral over the simplices will be zero by Goursat's lemma. 

We can then approximate the region in between $B_a(R + \varepsilon)$ and $B_z(\varepsilon)$ by a collection of loops of the above form. 

This completes the proof

\qed

\thm (Inverse Function Theorem)

Let $f \in \ms{H}(\Omega)$, with $\Omega\subseteq \C$ open. Let $z_0 \in \Omega$. 

Suppose that $f'(z_0) \neq 0$. Then there exists a $\varepsilon>0$ such that $f|_{B_\varepsilon(z_0)}$ is a biholomorphism onto its image, which is open. 

Moreover, there is a $g:f(B_{\varepsilon}(z_0))\to B_{\varepsilon}(z_0))$ which is holomorphic, such that $g \circ f = \Id$. In other words, $g$ is a ``local inverse" to $f$ (think two branches of square root, etc.)


\proof

Let $z_0 = \begin{pmatrix} x_0 \\ y_0 \end{pmatrix}$

Just looking at $f$ purely as a smooth function from $\R^2\to\R^2$, we have

\[
\det\begin{pmatrix} u_x & u_y \\ v_x & v_y \\ \end{pmatrix} = u_xv_y - u_yv_x 
\]

By Cauchy-Riemann, we have $f' = u_x + iv_x$, so $u_x^2 + v_x^2 = \abs*{f'}^2|_{(x_0, y_0)} \neq 0$. This immediately gives bijectivity of the map $f$, and the fact that its image is open.

Now $f^{-1}:f(B_{\varepsilon}(z_0))\to B_{\varepsilon}(z_0)$ is a real $C^\oo$ function. We just need to prove it is analytic. 

We know $\Id_{B_{\varepsilon}(z_0)} = g \circ f$, so 
\begin{align*}
\Id & = dg \circ df \\
	 & =(\del g + \bar{\del}g)\circ(\del f + \bar{\del} f) \\
	 & = \del g \circ \del f + \bar{\del}g \circ \del f \\
\end{align*}

We must conclude that $\bar{\del}g \circ \del f$ is both $\C$-linear and $\C$-antilinear, meaning it is identically zero. Now, $\del f$ is linear and bijective, thus, $\bar{\del}g = 0$, so $dg = \del g$, so $g$ is holomorphic. 

\qed

If $\det df = 0$ for a real smooth function, then we can't really say anything about $f$. However, suppose $f'(z) = 0$, but $f'(z) \not\equiv 0$. Then we know a lot about the structure of $f$. 

A homework problem asks to show the following: let $K \subseteq \Omega \subseteq \C$, with $\Omega\neq K$, $K$ compact, $\Omega$ open.

 By compactness, $d(K, \C\backslash\Omega) = \delta > 0$. So 

\[
f(z) = \frac{1}{2\pi i}\int_{C_{\frac{\delta}{2}}(a)}\frac{f(w)}{w - z}\,dz
\]

Then 

\[
f'(a) = \frac{1}{2\pi i}\int_{C_{\frac{\delta}{2}}(a)}\frac{f(w)}{(w - a)^2}\,dw
\]

And now you can do some bounding shenanigans to get $\leq \frac{\sup |f|}{(\frac{\delta}{2})^2}$

By controlling the derivative, we may control the derivative. The reverse is not true: no matter how we restrict the variance of $f$, the derivative can do strange things. 

But for solutions to PDEs, the control goes the other way. That is, if you control $\sup|f|$, this controls the derivative, a fact which is not true in general. And every holomorphic function is a solution to a PDE.

So if a sequence of functions converges uniformly, so does the sequence of their derivatives, and so on. 

\thm (Liouville)

Suppose $f:\C\to\C$ is an entire function (meaning holomorphic on all of $\C$), such that $\sup_{\C}|f| < \oo$. Then $f(z)$ is constant. 

\rem This gives the easiest proof of the fundamental theorem of algebra:

Suppose we have a nice non-constant polynomial $p$ without a root. Then $\frac{1}{p}$ is entire and bounded, so $\frac{1}{p}$ must be constant, so $p$ is constant, a contradiction (unless $p$ is constant in the first place).

Thus $p$ must have a root. 


\proof

Write 

\[
f(z) = \frac{1}{2\pi i}\int_{C_R(z_0)}\frac{f(w)}{w - z_0}\,dw
\] Then 

\[
f' = \frac{1}{2\pi i}\int_{C_r(z_0)}\frac{f(w)}{(w - z_0)^2}\,dw
\]

The complex triangle inequality tells us

\[
|f'| \leq c\abs*{\int_{C_r(z_0)}\frac{f(w)}{(w - z_0)^2}}|dw| \leq c \frac{\sup_{\C}|f|}{R^2}2\pi R = c\frac{\sup_{\C}|f|}{R} 
\]

Taking a limit as $R\to\oo$, we see that $|f'(z_0)| = 0$. But $z_0$ was arbitrary, so $f$ is constant. 

\thm (Factorization)

Let $f:B_{\varepsilon}(a)\to\C$ be holomorphic, and suppose $f(a) = 0$ and $f' \not\equiv 0$. 

Then there exists an $N \in \N$ such that $f(z) = (z - a)^N\phi(z)$, with $\phi$ holomorphic in $B_{\varepsilon}(a)$, and $\phi(a)\neq 0$. 

\proof

Write 

Find the first $N$ such that the $N$th derivative of $f$ is nonzero. 

\[
f(z) = \sum_{n=1}^\oo c_n(z - a)^n = \sum_{n=N}^\oo c_n(z - a)^n = (z - a)^n(c_N + c_{N + 1}(z - a) + \cdots )
\]

\cor

The zeroes of a holomorphic function are isolated, unless the function is identically zero. 

\cor(analytic continuation principle)

Let $f, g:B_{\varepsilon}(a)\to\C$ be holomorphic, and let $f(z_j) = g(z_j)$ for some sequence $z_j \to a$, $z_j \neq a$ for all $j$. 

Then $f = g$ in $B_{\varepsilon}(a)$. 

This follows directly from the zeroes of non-zero holomorphic functions being isolated. 

\qed

\rem 

Let $\Omega_1, \Omega_2$ be open, $\Omega_1\subset\Omega_2\subset\C$, with $f, g:\Omega_2\to\C$ holomorphic. Suppose that $f|_{\Omega_1} = g|_{\Omega_1}$. 

Then $f = g$ in $\Omega_z$. 

\proof

Let $a \in \Omega_1$, let $\gamma:I\to\Omega_2$ be a path from $a$ to $p \in \Omega_2$. Let $G = \{t \in I \mid |f(\gamma(t)) - g(\gamma(t))| = 0$. This set is closed, and moreover $[0, \varepsilon) \subset G$ for any $0<\varepsilon<1$. We can then prove that $G$ is open. 

But for $G$ to be a clopen subset of $I$ means that $G = I$ (also because $G$ is nonempty). 

\section*{Lecture 4}

\rem

Dennis wants to comment on a question on the homework relating to expansions in an annulus. 

Consider $\frac{z^2}{z - 3i}$. Expand in powers of $(z - 1)$ in an annulus containing $i$. You don't need anything about Laurent series. The only thing you need to know is 
\[
\frac{1}{1 - q} = \sum_{n=0}^\oo q^n
\]
when $|q| < 1$. 

What if $|z| > 1$?

Note that 
\[
\frac{1}{1 - z} = \frac{1}{z}\frac{1}{1 - \frac{1}{z}} = \frac{1}{z}\sum_{n=0}^\oo\frac{1}{z^n}
\]

Let's do the problem. 

At $3i$ there is a singularity, and we want an annulus centered at $1$. 

Take 

\begin{align*}
\frac{z^2}{z - 3i} & = ((z - 1) + 1))^2\frac{1}{(z - 1) + (1 - 3i)} \\
						  & = \frac{(1 + (z - 1))^2}{1 - 3i}\frac{1}{1 - \frac{z - 1}{3i - 1}} \\
						  & = \frac{(1 + w^2)}{1 - 3i}\sum_{n=0}^\oo\frac{1}{(3i - 1)^n}w^n, w = z - 1, |w| < |3i - 1| = \sqrt{10}
\end{align*}

What if we want to do it in an annulus containing $4i$? 

The same calculations hold, up to the second line: 

\begin{align*}
\frac{z^2}{z - 3i} & = ((z - 1) + 1))^2\frac{1}{(z - 1) + (1 - 3i)} \\
						  & = \frac{(1 + (z - 1))^2}{1 - 3i}\frac{1}{1 - \frac{z - 1}{3i - 1}} \\
						  & = \frac{1 + (z - 1)^2}{(1 - 3i)\frac{z - 1}{3i - 1}}\left(\frac{1}{-1 + \frac{3i - 1}{z - 1}}\right) \\
						  & = \frac{1 + (z - 1)^2}{(z - 1)}\frac{1}{1 - \frac{3i - 1}{z - 1}} \\
							& = \frac{1 + \cdots}{\cdots}\sum_{n=0}^\oo (3i - 1)^n\frac{1}{(z + 1)^n} \\
\end{align*}

For the last inequality, $|z - 1| > \sqrt{10}$. 

What if we want an expansion in the maximal annulus containing $4i$?

Now, to elaborate on something in the main local theorem. 

\thm (Morera's)

Let $f:B_{R_0}(a) \to \C$. The following are equivalent:
\begin{itemize}
\item $f \in \ms{H}(B_{R_0}(a))$
\item $f$ is \underline{continuous} in $B_{R_0}(a)$ and for all solid 2-simplices $\bigtriangleup$ with boundary $T$, we have
\[
\int_Tf(z)\,dz = 0
\]
for some appropriate parameterization of $T$. 
\end{itemize}

\defn

Let $\phi$ be a function. The \underline{support} of $\phi$, $\operatorname{supp} \phi$, is defined as $S = \bar{\{z\mid\phi(z)\neq 0\}}$. 

\rem The support is a closed set, and $\phi|_{S^c} \equiv 0$. 

\defn

Let $X$ be an open subset of $C$. 

Then 
\[
C_0^\oo(X, \C)\eqdef C_0^\oo(x) = \{f:X\to \C \mid \text{ all}\del_x^\alpha,\del_y^\beta f\text{ exist in }X,\operatorname{supp}f \text{ is compact}\}
\]
This is often called $\ms{D}(X)$

Any element of $C_0^\oo$ is not holomorphic, as it is identically zero on an open set. 

An example: 
\[
f(x) = \begin{cases} e^{-\frac{1}{|z|^2 - 1}} & |z| < 1 \\ 0 & |z| \geq 1 \end{cases} 
\]
Then $f(x) \in \ms{D}(\C)$. 

\thm 

The following are equivalent: 
\begin{enumerate}[label=(\alph*)]
\item $f \in \ms{H}(B_{R_0}(a))$ 
\item $\forall R < R_0$, $f \in L^1(B_{R_0}(a))$ (in the Lebesgue sense) and $\frac{\del f}{\del \bar{z}} = 0$. This means that for any $\phi \in \ms{D}(B_{R_0}(a))$, $\int_{B_{R_0}(a)}f\frac{\del \phi}{\del \bar{z}}\,d\lambda^2 = 0 $. 
\end{enumerate}

\proof of Morera's theorem

Showing $(i) \implies (ii)$ is Goursat's lemma. So we will now prove $(ii) \implies (i)$. 

\subsubsection*{\underline{Step 1}}

Define $F:B_{R_0}(a)\to\C$ defined by $z\mapsto\int_{S(z)}f(w)\,dw$, where $S(z): I\to B_{R_0}(a)$ is given by $t \mapsto (1 - t)a + tz$.

\subsubsection*{\underline{Goal}}

Our goal is to show that $F$ is holomorphic in $B_{R_0}(a)$, and $F'(z) = f(z)$ for all $z$. 

\subsubsection*{\underline{Step 2}}

By definition, 
\[
F' = \lim_{h\to0}\frac{F(z + h) - F(z)}{h} = \frac{1}{h}(\int_{I_2} + \int_{I_1}) = \frac{1}{h}\int_{I_3}f\,dz
\]
Where $I_1$ is a line from $z$ to $a$, $I_2$ is from $a$ to a third point, and $I_3$ is from that third point to $z$. The last term goes to $0$ as $h$ goes to $0$. 


\thm (Removable singularity) 

Let $f:B_R\setminus\{0\}\to\C$ satisfy 
\begin{enumerate}[label=(\alph*)]
\item $f \in \ms{H}(B_R\setminus\{0\})$
\item $\frac{|f(z)|}{|z|}\to0$ as $z\to0$. 
\end{enumerate}

Then we can define $f(0)$ such that $f \in \ms{H}(B_R)$. 

\proof

Define $f(z)z = F(z)$ for $z \neq 0$. 

Define $F(0) = 0$.

Then $F$ is continuous in $B_R(0)$: we will prove $F \in \ms{H}(B_R(0))$

So
\begin{align*}
f(z)z & = F(z) = \sum_{n=1}^\oo a_nz^{n - 1} \\
		& = \sum_{n=0}^\oo b_nz^n \\
\end{align*}

And we are done. 

By Goursat, we have to prove that for all $\bigtriangleup$, $\int_TF\,dz = 0$. 

There are some cases:
\begin{enumerate}
\item In case $1$, $\int_TF\,dz = 0$, so we're done by Goursat. 
\item In case $2$, where $T$ has 0 as a vertex, we can chop off a tiny piece of $T$, and then the integral is the integral over the new trapezoid we've created, plus the integral over a smaller triangle. The magnitude of this second integral is bounded above by the max of $F$ on the triangle times the length of the triangle. This is bounded above by a constant times $\varepsilon$.  So $\abs*{\int_TF\,dz} = 0$.
\item In case $3$, $T$ has zero on the boundary, but not a vertex. In this case, we do more subdivision shenanigans, and it just works by case $2$. 
\item In case $4$, 0 is an interior point of $\bigtriangleup$. We can do more subdivision shenanigans, and use case $3$ and it just works. 
\end{enumerate}

We are now done

\qed

We also have:

Suppose $\Delta u = 0$ in $B_R\setminus\{0\}$, it suffices to have $\frac{|u(z)|}{\ln|z|}\to 0$. Then $u \in C^2(B_R(a))$ and $\Delta u = 0$. 

\thm $(i)\implies (ii)$

Let $f \in \ms{H}$. Then for all $k, m$, $\del_x^k\del_y^mf$ exists and is continuous, and $\frac{\del f}{\del \bar{z}} = \frac{1}{z}(\del_x + i\del_y)f = 0$. 

Then $\frac{1}{2}(\del_x + i\del_y)f\phi\,dxdy = 0$ for all $\phi \in \ms{D}(B_R(a))$. 

\proof

By Fubini, 
\begin{align*}
\frac{1}{2}\int_{\C}(\del_x f)\phi dxdy + \cdots & = \frac{1}{2}\int_{\R}\left(\int_{\R}\del_xf(x, y)\phi(x, y)dx\right)dy \\
			& = -\frac{1}{2}\int_{\R}\left(\int_{a_y}^{b_y}f\del_x\phi\,dx\right)\,dy + 0 \\
			& = -\frac{1}{2}\int_{\C}f\frac{\del\phi}{\del\bar{z}}\,d\lambda^2 \\
\end{align*}

\qed

We actually proved the following statement: 

Let $x \in \Omega\subseteq \R^N$, $u$ smooth, $f$ smooth.
Suppose we have a constant coefficient differential operator $f = \sum_{|\alpha|\leq N}C_\alpha\del_x^{\alpha}(u)$

Then $u \in L^1(\Omega)$ and 
\[
\int u \left(\sum_{|\alpha|\leq N}(-1)^{|\alpha|}C_\alpha \del_x^\alpha \phi\right)d\lambda^2 = \int_{\Omega}f\phi\,d\lambda^N
\]
for all $\phi \in \ms{D}(\Omega)$. 

So for a differential operator with constant coefficients, we can define a distribution of solutions to a PDE, such as $\bigtriangleup u = f$. 

\section*{Lecture 5}

Let $\alpha, \beta:I\to X$ be two loops in an open, connected subset of $\C$. When may we ``continuously deform" these loops into each other? 

What even is a ``continuous deformation"?

\defn

If $\alpha, \beta$ are two loops in $X$, we say that $\alpha, \beta$ are \underline{freely homotopic}, or $\alpha\overset{\circ}{\simeq}\beta$, if there exists a continuous $F:I \times I \to X$, such that

\begin{align*}
F(t, 0) & = \alpha(t), t \in I \\
F(t, 1) & = \beta(t), t \in I\\
F(0, s) & = F(1, s), s \in I \\
\end{align*}

\thm (Homotopy Cauchy Formula)

Let $X \subseteq\C$ be connected and open, and $f:X\to\C$ is holomorphic. Let $\gamma_0, \gamma_1:I\to X$ be two piecewise smooth loops. If $\gamma_0, \gamma_1$ are freely homotopic in $X$, then
\[
\int_{\gamma_0}f\,dz = \int_{\gamma_1}f\,dz
\]

\proof

\subsubsection*{\underline{Step 1}}

\claim

By assumption, there is a homotopy $H\in C(I\times I, X)$ between $\gamma_0, \gamma_1$. This function can be approximated for any $\varepsilon>0$ by $H_{\varepsilon}:I\times I \to X$ in the following sense: 

\begin{enumerate}
\item $H_\varepsilon$ is continuous. 
\item We can chop $I\times I$ into squares. Every $[t_j, t_{j + 1}]\times[s_k, s_{k + 1}]$ is mapped into some $B_{\rho}(a) \subseteq X$, where $a, \rho$ depend on $j, k$, where, $t_i, s_i$ are elements of two partitions of $I$. 
\item For all $s$, $H_{\varepsilon}(-, s_j)$ is linear on $[t_k, t_{k + 1}]$, and for all $t$, $H_\varepsilon(t_j, -)$ is linear on $[s_k, s_{k + 1}]$
\item $H_\varepsilon(0, s) = H_\varepsilon(1, s)$ for all $s \in I$. 
\item 
\begin{align*}
\int_{\gamma_0}f\,dz - \int_{H_\varepsilon(-, s)}f\,dt & \leq \varepsilon \\
\int_{\gamma_1}f\,dz - \int_{H_\varepsilon(t, -)}f\,ds & \leq \varepsilon
\end{align*}
\end{enumerate}

\proof of claim. 

The proof of the existence of $H_\varepsilon$ follows the same idea as the proof of: 

\claim

Let $\phi:I\to\R$ be continuous. Then, for every $\varepsilon>0$, there is an $N$ such that there exists a piecewise linear $\phi_\varepsilon$, with 
\[
\sup_I|\phi - \phi_\varepsilon|\leq\varepsilon
\]
And, if $\phi'$ exists and is continuous, then $\phi_\varepsilon'$ exists and is continuous, and further 
\[
\sup_I|\phi' - \phi'_\varepsilon|\leq \varepsilon
\]

\proof

On homework.

\subsubsection*{\underline{Step 2}}

Let $\tilde{\gamma_0} = H_\varepsilon(-, 0)$, and $\tilde{\gamma_1} = H_\varepsilon(-, 1)$ be piecewise linear loops. 

We want to show that $\int_{\tilde{\gamma_0}}f\,dz = \int_{\tilde{\gamma_1}}f\,dz$, and then by part 5 of step 1, we would be finished. Let's do that now. 

Dennis has drawn a picture of the unit square, with the bottom $\frac{1}{N}$ of it noted. This box, of length 1 and height $\frac{1}{N}$, he integrates over the boundary.

Goal: 
\[
\int_{\tilde{\gamma_0}}f\,dz = \int_{H_\varepsilon(-, \frac{1}{N})}f\,dz
\]

The integrals of the left and right side of the box are equal, but have opposite orientation, so they cancel. If we can show the integral over the entire boundary of the small box is 0, we would be done. 

We can chop up $[0, 1]\times[0,\frac{1}{N}]$ into a bunch of boxes. The integral over each of the small boxes is zero, because each small box is mapped into $B_\rho(a)$. The sum of these integrals, which is the integral over the box, is zero. 

\qed

\subsection*{\underline{Index of a curve}}

\defn

Let $\gamma:I\to\C\setminus\{0\}$ be a piecewise smooth loop. We define the \underline{index} as follows.
\[
\operatorname{ind}(\gamma, \rho) = \frac{1}{2\pi i}\int_\gamma\frac{dz}{z}
\]

If you expand this out, things will cancel and this will always be an integer. 

By the theorem we just proved, we can show that the index of two freely homotopic loops is the same. 

Ahlfor's computation: 

Let $h(t) = \int_0^t\frac{\gamma'(s)}{\gamma(s)}\,ds$. This function is piecewise smooth by the fundamental theorem of calculus. $h(0) = 0$, $h(1) = \int_{\gamma}\frac{dz}{z}$. Ahlfors observes that if we consider $g(t) = e^{-h(t)}\gamma(t)$ is a nice, continuous, piecewise smooth function, so it has derivative 

\begin{align*}
g' & = e^{-h}(-h')\gamma + e^{-h}\gamma'\\
& = e^{-h}\frac{\gamma'}{\gamma}\gamma + -e^{-h}\gamma' = 0
\end{align*}

The derivative being zero tells us $g(0) = g(1)$. So $\gamma(0) = e^{-h(1)}\gamma(1)$, so $h(1) = 2i \pi N$ for some integer $N$. 

\thm

Let $\lambda$ be a piecewise smooth loop in $\C$, such that $p \not\in\lambda(I)$. Then
\begin{enumerate}
\item The index $\operatorname{ind}(\gamma, p)$ is an integer (we already proved this)
\item $\operatorname{ind}(\gamma, -):\C\setminus\lambda(I)\to\Z$ is continuous. That is, it is locally constant (constant on every connected component).  
\item Suppose $\lambda\overset{\circ}{\simeq} \nu$ in $\C\setminus\{p\}$. Then $\operatorname{ind}(\lambda, p) = \operatorname{ind}(\nu, p)$.
\item If $\gamma(I) \subseteq B_\rho(a)$, with $p\not\in B_\rho(a)$, then $\ind(\lambda,p) = 0$.
\end{enumerate}

Let $X = \C\setminus\{0\}$, also called $\C^\times$. Let $\gamma$ be a loop around 0, and $\beta$ be a loop not around zero. How do we show $\alpha\not\overset{\circ}{\simeq}\beta$ ?

First, you show that $\C\setminus\{0\} \cong S^1$ is a homotopy equivalence. 

Then you life $\alpha,\beta$ into the cover of $S^1$, etc. 

Alternatively, if $\alpha\overset{\circ}{\simeq}\beta$, then $\ind(\alpha,0) = \ind(\beta,0)$. We can thus clearly see that $\alpha\not\overset{\circ}{\simeq}\beta$. 

\section*{Lecture 6}

\thm

Suppose $f \in \ms{H}(\{z \mid r_0 - \varepsilon<  |a - z| < r + \varepsilon\}$

Then
\begin{enumerate}[label=(\roman*)]
\item 
\[
f(z) = \frac{1}{2\pi i}\int_{C_{r_1}(a)}\frac{f(w)}{w - z}\,dw - \frac{1}{2\pi i}\int_{C_{r_0}(a)}\frac{f(w)}{w - z}\,dw
\] 
\item There exists $g_1(z) = \sum_{n=0}^\oo (z - a)^nc_n$ which converges in $|z - a| < r_1$, and $g_2(z) = \sum_{n=1}^\oo\frac{d_n}{(z - a)^n}$ which converges in $|z - a| > r_0$, such that $f(z) \equiv g_1(z) + g_2(z)$ for $r_0 < |z - a| < r_1$. 
\item $c_n, d_n$ are unique. 
\item For all piecewise smooth loops $\lambda:S^1\to\{r_0\leq|z - a|\leq r_1\}$, 
\begin{align*}
\ind(\lambda, a)c_n & = \frac{1}{2\pi i}\int_{\lambda}\frac{f(w)}{(w - a)^{n + 1}}\, dw,\, n \geq 0 \\
\ind(\lambda, a)d_n & = \frac{1}{2\pi i}\int_{\lambda}(w - a)^{n - 1}f(w)\,dw,\, n \geq 1 \\
\end{align*}
\end{enumerate}

\cor

Suppose $f:B_R(a)\setminus\{a\}\to\C$ is holomorphic, and $\lambda:S^1\to B_R(a)\setminus\{a\}$ is piecewise smooth. Then
\[
\int_{\lambda}f(w)\,dw = \ind(\lambda, a)d_1
\]
$f(z) = \frac{d_1}{z - a} + \sum_{n\neq -1}d_n(z - a)^n$. 

\defn
The coefficient $d_1$ is called the \underline{residue of $f$ at $a$}, denoted $\res_a f$. 

\proof

\begin{enumerate}[label=(\roman*)]
\item Fix $z$ with $r_0 < |z - a| < r_1$, and consider the function
\[
F(w) = \frac{f(w) - f(z)}{w - z}
\]
This is holomorphic in the annulus $\{r_0 \leq |w - a| \leq r_1 \}\setminus\{z\}$. Notice that $\lim_{w\to z}F(w) = f'(z)$, so $F$ is bounded in a neighborhood of the singularity at $z$, so this singularity is removable. 

Now, we have $r_0e^{it} \overset{\circ}{\simeq} r_1e^{it}$, with $H(s, t) = (1 - s)e^{ir_0t} + se^{ir_1t}$. So
\[
\int_{C_{r_0}(a)}F(w)\,dw = \int_{C_{r_1}(a)}F(w)\,dw
\]
So
\[
-f(z)\int_{C_{r_1}(a)}\int\frac{dw}{w - z} + \int_{C_{r_0}(a)}\frac{f(w)}{w - z}\,dw
\]
Which is equal to 
\[
-f(z)\int_{C_{r_1}(a)}\frac{dw}{w - z} + \int_{C_{r_1}(a)}\frac{f(w)}{w - z}\,dw
\]
\end{enumerate}

Details of the rest of the proof posted by Denis on The Gauchu. 

\rem In the proof of $(iv)$ you use the fact that if $f$ is holomorphic and $\gamma:I\to\C$ smooth, then $\frac{d}{dt}(f \circ \gamma(t)) = f'(\gamma(t))\dot{\gamma}(t)$. 

$d(f\circ \gamma) = df\circ d\gamma = \del f \circ (\dot{\gamma}dt) = f'dz\circ(\dot{\gamma}dt) = (f'\dot{\gamma})dt$

We now study isolated singularities. 

\defn

Let $f:B_{\delta}(a)\setminus\{a\}\to\C$ be holomorphic. The \underline{order of $f$ at $a$}, denoted $\omega(f, a)$, is the minimal power in Laurent's expansion of $f$ in $0 < |z - a| < \delta$. 
\[
\omega(f, a) = N <=> f(z) = \sum_{n=N}^\oo b_n(z - a)^n, b_N\neq0
\]

\[
\omega(f, a) = -\oo <=>f(z) = \sum_{n=-\oo}^\oo b_n(z - a)^n, b_n\neq0\text{ for infinitely many }n
\]

\thm

Let $f:B_{\delta}(a)\setminus\{a\}\to\C$ be holomorphic. 
\begin{enumerate}[label=(\roman*)]
\item $\omega(f, a) = -N$ for a finite $N > 0$ if and only if $f(z) = \frac{\phi(z)}{(z - a)^N}$, $N > 0$ finite, with $\phi$ holomorphic in $B_{\delta}(a)$, $\phi(a)\neq0$, which happens if and only if $|f(z)|\to\oo$ as $z\to a$. 
\item If $\omega(f, a) = -\oo$, then for all $\varepsilon>0$, $\bar{f(B_{\varepsilon}(a)\setminus\{a\})} = \C$.
\end{enumerate}

\proof of (ii)
Assume for the sake of contradiction that for some $\varepsilon>0$, $\bar{f(\cdots)}\neq\C$. That means there exists a point $w_0 \in \C$ and a real $\rho>0$, such that $\bar{B_{\rho}(w_0)}\cap f(\cdots) = \varnothing$. 

So for all $0 < |z - a| < \delta$, $|f(z) - w_0| \geq \rho$. 

Now, consider $F(z) =\frac{1}{f(z) - w_0}$. This is holomorphic in $B_{\varepsilon}(a)\setminus\{a\}$, and $|F(z)| \leq \frac{1}{\rho}<\oo$.

So we can remove the singularity at $a$ to make $F$ holomorphic in $B_{\varepsilon}(a)$. We can factorize it as 
\[
F(z) = (z - a)^N\phi(z),\,\phi\in\ms{H}(B_{\varepsilon}(a)),\,\phi(a)\neq0
\]
Then
\[
F = \frac{1}{f - w_0}
\]
which is equivalent to 
\[
f(z) = w_0 + \frac{1}{F} = w_0 + \frac{\frac{1}{\phi(z)}}{(z - a)^N} = w_0 + \frac{1}{(z - a)^N}\psi(z) = \frac{1}{(z - a)^N}\sum_{n=0}^\oo c_n(z - a)^n
\]
But then $\omega(f, a)\neq -\oo$, a contradiction. 

\proof of (i)

(i) is a string of 3 equivalent statements. You show $1 \implies 2\implies 3 \implies 1$. This is easy, and for the final implication, you use part (ii). 

Let $\hat{\C}$ denote the Riemann sphere, which is topologically the one point compactification of $\C$. 

\defn

Let $a \in X \subset \hat{\C}$, $X$ open in $\ms{T}_{\oo}$, and consider $f:X\to\hat{\C}$. 

$f$ is holomorphic in a neighborhood of $a$ if
\begin{enumerate}
\item $a \in \C$, $f(a) \in \C$, with $f$ holomorphic in the old sense. 
\item $a \in \C$, $f(a) = \oo$ if and only if $F(z) = \begin{cases} \frac{1}{f(z)} & z\neq a\\ 0 & z = a \\ \end{cases}$ is holomorphic in a neighborhood of $0$. 
\item $a = \oo$, $f(a) \in \C$ is holomorphic if and only if $F(z) = \begin{cases} f(\frac{1}{z}) & z \neq0\\ f(a) & z = 0 \\ \end{cases}$ is holomorphic
\item $a = \oo, f(a) = \oo$ is holomorphic if and only if $F(z) = \begin{cases} \frac{1}{f(\frac{1}{z})} & z\neq0 \\ 0 & z = 0 \\ \end{cases}$ is holomorphic near 0 in the old sense. 
\end{enumerate}

In case 2, $a$ is a pole of $f$, meaning $\omega(f, a) = N < 0$ is finite. 

\thm

Let $f:\hat{\C}\to\hat{\C}$ be holomorphic in a neighborhood of $\oo$. Then 
\begin{enumerate}[label=(\roman*)]
\item $f(\oo) = 0$ if and only if $f(z) = \frac{1}{z^N}\phi(z)$ for $|z| > R$, $0<N < \oo$, $\phi$ holomorphic on $|z|>R$, and $\phi\to A\neq0, \neq\oo$ as $|z|\to\oo$. 
\item $f(\oo)=\oo$ if and only if $f(z) = z^N\phi(z), |z| > R, 0 < N < \oo, \phi\to A\neq0,\neq\oo$ as $|z|\to\oo$. 
\end{enumerate}

\underline{Exercise:}

Suppose $f:\hat{\C}\to\hat{\C}$ is holomorphic on $\hat{\C}$. Show that this is equivalent to $f$ being a rational function. 

\defn

Let $X \subset \C$ be open, connected. A map $f:X\to\hat{\C}$ is \underline{meromorphic} if it is holomorphic in a neighborhood of every point of $X$ in the extended sense.   

The set of meromorphic functions is called $\ms{M}$. This is equivalent to a function $f:\C\to\C$ being holomorphic everywhere except an at most countable family of isolated poles. 

\prop

Let $X \subset \C$ be open. 
\begin{enumerate}[label=(\roman*)]
\item $\ms{M}(X)$ is a field
\item $f \in \ms{M}(X)$ if and only if $\frac{1}{f} \in \ms{M}(X)$. 
\item $S_{\frac{1}{f}} = Z_fZ_{\frac{1}{f}} = S_f$
\end{enumerate}

All of the above can be proven using the factorization theorem. 

































\end{document}