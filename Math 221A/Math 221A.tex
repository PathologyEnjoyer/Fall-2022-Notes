\documentclass[x11names,reqno,14pt]{extarticle}
% Choomno Moos
% Portland State University
% Choom@pdx.edu


%% stupid experiment %%
%%%%%%%%%%%%% PACKAGES %%%%%%%%%%%%%

%%%% SYMBOLS AND MATH %%%%
\let\oldvec\vec
\usepackage{authblk}	% author block customization
\usepackage{microtype}	% makes stuff look real nice
\usepackage{amssymb} 	% math symbols
\usepackage{siunitx} 	% for SI units, and the degree symbol
\usepackage{mathrsfs}	% provides script fonts like mathscr
\usepackage{mathtools}	% extension to amsmath, also loads amsmath
\usepackage{esint}		% extended set of integrals
\mathtoolsset{showonlyrefs} % equation numbers only shown when referenced
\usepackage{amsthm}		% theorem environments
\usepackage{relsize}	%font size commands
\usepackage{bm}			% provides bold math
\usepackage{bbm}		% for blackboard bold 1

%%%% FIGURES %%%%
\usepackage{graphicx} % for including pictures
\usepackage{float} % allows [H] option on figures, so that they appear where they are typed in code
\usepackage{caption}
\usepackage{hyperref}
%\usepackage{titling}
\usepackage{tikz} % for drawing
\usetikzlibrary{shapes,arrows,chains,positioning,cd,decorations.pathreplacing,decorations.markings,hobby,knots,braids}
\usepackage{subcaption}	% subfigure environment in figures

%%%% MISC %%%%
\usepackage{enumitem} % for lists and itemizations
\setlist[enumerate]{leftmargin=*,label=\bf \arabic*.}

\usepackage{multicol}
\usepackage{multirow}
\usepackage{url}
\usepackage[symbol]{footmisc}
\renewcommand{\thefootnote}{\fnsymbol{footnote}}
\usepackage{lastpage} % provides the total number of pages for the "X of LastPage" page numbering
\usepackage{fancyhdr}
\usepackage{manfnt}
\usepackage{nicefrac}
%\usepackage{fontspec}
%\usepackage{polyglossia}
%\setmainlanguage{english}
%\setotherlanguages{khmer}
%\newfontfamily\khmerfont[Script=Khmer]{Khmer Busra}

%%% Khmer script commands for math %%%
%\newcommand{\ka}{\text{\textkhmer{ក}}}
%\newcommand{\ko}{\text{\textkhmer{ត}}}
%\newcommand{\kha}{\text{\textkhmer{ខ}}}

%\usepackage[
%backend=biber,
% numeric
%style=numeric,
% APA
%bibstyle=apa,
%citestyle=authoryear,
%]{biblatex}

\usepackage[explicit]{titlesec}
%%%%%%%% SOME CODE FOR REDECLARING %%%%%%%%%%

\makeatletter
\newcommand\RedeclareMathOperator{%
	\@ifstar{\def\rmo@s{m}\rmo@redeclare}{\def\rmo@s{o}\rmo@redeclare}%
}
% this is taken from \renew@command
\newcommand\rmo@redeclare[2]{%
	\begingroup \escapechar\m@ne\xdef\@gtempa{{\string#1}}\endgroup
	\expandafter\@ifundefined\@gtempa
	{\@latex@error{\noexpand#1undefined}\@ehc}%
	\relax
	\expandafter\rmo@declmathop\rmo@s{#1}{#2}}
% This is just \@declmathop without \@ifdefinable
\newcommand\rmo@declmathop[3]{%
	\DeclareRobustCommand{#2}{\qopname\newmcodes@#1{#3}}%
}
\@onlypreamble\RedeclareMathOperator
\makeatother

\makeatletter
\newcommand*{\relrelbarsep}{.386ex}
\newcommand*{\relrelbar}{%
	\mathrel{%
		\mathpalette\@relrelbar\relrelbarsep
	}%
}
\newcommand*{\@relrelbar}[2]{%
	\raise#2\hbox to 0pt{$\m@th#1\relbar$\hss}%
	\lower#2\hbox{$\m@th#1\relbar$}%
}
\providecommand*{\rightrightarrowsfill@}{%
	\arrowfill@\relrelbar\relrelbar\rightrightarrows
}
\providecommand*{\leftleftarrowsfill@}{%
	\arrowfill@\leftleftarrows\relrelbar\relrelbar
}
\providecommand*{\xrightrightarrows}[2][]{%
	\ext@arrow 0359\rightrightarrowsfill@{#1}{#2}%
}
\providecommand*{\xleftleftarrows}[2][]{%
	\ext@arrow 3095\leftleftarrowsfill@{#1}{#2}%
}
\makeatother

%%%%%%%% NEW COMMANDS %%%%%%%%%%

% settings
\newcommand{\N}{\mathbb{N}}                     	% Natural numbers
\newcommand{\Z}{\mathbb{Z}}                     	% Integers
\newcommand{\Q}{\mathbb{Q}}                     	% Rationals
\newcommand{\R}{\mathbb{R}}                     	% Reals
\newcommand{\C}{\mathbb{C}}                     	% Complex numbers
\newcommand{\K}{\mathbb{K}}							% Scalars
\newcommand{\F}{\mathbb{F}}                     	% Arbitrary Field
\newcommand{\E}{\mathbb{E}}                     	% Euclidean topological space
\renewcommand{\H}{{\mathbb{H}}}                   	% Quaternions / Half space
\newcommand{\RP}{{\mathbb{RP}}}                       % Real projective space
\newcommand{\CP}{{\mathbb{CP}}}                       % Complex projective space
\newcommand{\Mat}{{\mathrm{Mat}}}						% Matrix ring
\newcommand{\M}{\mathcal{M}}
\newcommand{\GL}{{\mathrm{GL}}}
\newcommand{\SL}{{\mathrm{SL}}}

\newcommand{\tgl}{\mathfrak{gl}}
\newcommand{\tsl}{\mathfrak{sl}}                  % Lie algebras; i.e., tangent space of SO/SL/SU
\newcommand{\tso}{\mathfrak{so}}
\newcommand{\tsu}{\mathfrak{sl}}


% typography
\newcommand{\noi}{\noindent}						% Removes indent
\newcommand{\tbf}[1]{\textbf{#1}}					% Boldface
\newcommand{\mc}[1]{\mathcal{#1}}               	% Calligraphic
\newcommand{\ms}[1]{\mathscr{#1}}               	% Script
\newcommand{\mbb}[1]{\mathbb{#1}}               	% Blackboard bold


% (in)equalities
\newcommand{\eqdef}{\overset{\mathrm{def}}{=}}		% Definition equals
\newcommand{\sub}{\subseteq}						% Changes default symbol from proper to improper
\newcommand{\psub}{\subset}						% Preferred proper subset symbol

% Categories
\newcommand{\catname}[1]{{\text{\sffamily {#1}}}}

\newcommand{\Cat}{{\catname{C}}}
\newcommand{\cat}[1]{{\catname{\ifblank{#1}{C}{#1}}}}
\newcommand{\CAT}{{\catname{Cat}}}
\newcommand{\Set}{{\catname{Set}}}

\newcommand{\Top}{{\catname{Top}}}
\newcommand{\Met}{{\catname{Met}}}
\newcommand{\PL}{{\catname{PL}}}
\newcommand{\Man}{{\catname{Man}}}
\newcommand{\Diff}{{\catname{Diff}}}

\newcommand{\Grp}{{\catname{Grp}}}
\newcommand{\Grpd}{{\catname{Grpd}}}
\newcommand{\Ab}{{\catname{Ab}}}
\newcommand{\Ring}{{\catname{Ring}}}
\newcommand{\CRing}{{\catname{CRing}}}
\newcommand{\Mod}{{\mhyphen\catname{Mod}}}
\newcommand{\Alg}{{\mhyphen\catname{Alg}}}
\newcommand{\Field}{{\catname{Field}}}
\newcommand{\Vect}{{\catname{Vect}}}
\newcommand{\Hilb}{{\catname{Hilb}}}
\newcommand{\Ch}{{\catname{Ch}}}

\newcommand{\Hom}{{\mathrm{Hom}}}
\newcommand{\End}{{\mathrm{End}}}
\newcommand{\Aut}{{\mathrm{Aut}}}
\newcommand{\Obj}{{\mathrm{Obj}}}
\newcommand{\op}{{\mathrm{op}}}

% Norms, inner products
\delimitershortfall=-1sp
\newcommand{\widecdot}{\, \cdot \,}
\newcommand\emptyarg{{}\cdot{}}
\DeclarePairedDelimiterX{\norm}[1]{\Vert}{\Vert}{\ifblank{#1}{\emptyarg}{#1}}
\DeclarePairedDelimiterX{\abs}[1]\vert\vert{\ifblank{#1}{\emptyarg}{#1}}
\DeclarePairedDelimiterX\inn[1]\langle\rangle{\ifblank{#1}{\emptyarg,\emptyarg}{#1}}
\DeclarePairedDelimiterX\cur[1]\{\}{\ifblank{#1}{\emptyarg,\emptyarg}{#1}}
\DeclarePairedDelimiterX\pa[1](){\ifblank{#1}{\emptyarg}{#1}}
\DeclarePairedDelimiterX\brak[1][]{\ifblank{#1}{\emptyarg}{#1}}
\DeclarePairedDelimiterX{\an}[1]\langle\rangle{\ifblank{#1}{\emptyarg}{#1}}
\DeclarePairedDelimiterX{\bra}[1]\langle\vert{\ifblank{#1}{\emptyarg}{#1}}
\DeclarePairedDelimiterX{\ket}[1]\vert\rangle{\ifblank{#1}{\emptyarg}{#1}}

% mathmode text operators
\RedeclareMathOperator{\Re}{\operatorname{Re}}		% Real part
\RedeclareMathOperator{\Im}{\operatorname{Im}}		% Imaginary part
\DeclareMathOperator{\Stab}{\mathrm{Stab}}
\DeclareMathOperator{\Orb}{\mathrm{Orb}}
\DeclareMathOperator{\Id}{\mathrm{Id}}
\DeclareMathOperator{\vspan}{\mathrm{span}}			% Vector span
\DeclareMathOperator{\tr}{\mathrm{tr}}
\DeclareMathOperator{\adj}{\mathrm{adj}}
\DeclareMathOperator{\diag}{\mathrm{diag}}
\DeclareMathOperator{\eq}{\mathrm{eq}}
\DeclareMathOperator{\coeq}{\mathrm{coeq}}
\DeclareMathOperator{\coker}{\mathrm{coker}}
\DeclareMathOperator{\dom}{\mathrm{dom}}
\DeclareMathOperator{\cod}{\mathrm{codom}}
\DeclareMathOperator{\im}{\mathrm{im}}
\DeclareMathOperator{\Dim}{\mathrm{dim}}
\DeclareMathOperator{\codim}{\mathrm{codim}}
\DeclareMathOperator{\Sym}{\mathrm{Sym}}
\DeclareMathOperator{\lcm}{\mathrm{lcm}}
\DeclareMathOperator{\Inn}{\mathrm{Inn}}
\DeclareMathOperator{\sgn}{sgn}						% sgn operator
\DeclareMathOperator{\intr}{\text{int}}             % Interior
\DeclareMathOperator{\co}{\mathrm{co}}				% dual/convex Hull
\DeclareMathOperator{\Ann}{\mathrm{Ann}}
\DeclareMathOperator{\Tor}{\mathrm{Tor}}


% misc symbols
\newcommand{\divides}{\big\lvert}
\newcommand{\grad}{\nabla}
\newcommand{\veps}{\varepsilon}						% Preferred epsilon
\newcommand{\vphi}{\varphi}
\newcommand{\del}{\partial}							% Differential/Boundary
\renewcommand{\emptyset}{\text{\O}}					% Traditional emptyset symbol
\newcommand{\tril}{\triangleleft}					% Quandle operation
\newcommand{\nabt}{\widetilde{\nabla}}				% Contravariant derivative
\newcommand{\later}{$\textcolor{red}{\blacksquare}$}% Laziness indicator

% misc
\mathchardef\mhyphen="2D							% mathomode hyphen
\renewcommand{\mod}[1]{\ (\mathrm{mod}\ #1)}
\renewcommand{\bar}[1]{\overline{#1}}				% Closure/conjugate
\renewcommand\qedsymbol{$\blacksquare$} 			% Changes default qed in proof environment
%%%%% raised chi
\DeclareRobustCommand{\rchi}{{\mathpalette\irchi\relax}}
\newcommand{\irchi}[2]{\raisebox{\depth}{$#1\chi$}}
\newcommand\concat{+\kern-1.3ex+\kern0.8ex}

% Arrows
\newcommand{\weak}{\rightharpoonup}					% Weak convergence
\newcommand{\weakstar}{\overset{*}{\rightharpoonup}}% Weak-star convergence
\newcommand{\inclusion}{\hookrightarrow}			% Inclusion/injective map
\renewcommand{\natural}{\twoheadrightarrow}				% Natural map

% Environments
\theoremstyle{plain}
\newtheorem{thm}{Theorem}[section]
%\newtheorem{lem}[thm]{Lemma}
\newtheorem{lem}{Lemma}
\newtheorem*{lems}{Lemma}
\newtheorem{cor}[thm]{Corollary}
\newtheorem{prop}{Proposition}
\newtheorem*{claim}{Claim}
\newtheorem*{cors}{Corollary}
\newtheorem*{props}{Proposition}
\newtheorem*{conj}{Conjecture}

\theoremstyle{definition}
\newtheorem{defn}{Definition}[section]
\newtheorem*{defns}{Definition}
\newtheorem{exm}{Example}[section]
\newtheorem{exer}{Exercise}[section]

\theoremstyle{remark}
\newtheorem*{rem}{Remark}

\newtheorem*{solnx}{Solution}
\newenvironment{soln}
    {\pushQED{\qed}\renewcommand{\qedsymbol}{$\Diamond$}\solnx}
    {\popQED\endsolnx}%

% Macros
\newcommand{\restr}[1]{_{\mkern 1mu \vrule height 2ex\mkern2mu #1}}
\newcommand{\Upushout}[5]{
    \begin{tikzcd}[ampersand replacement = \&]
    \&#2\ar[rd,"\iota_{#2}"]\ar[rrd,bend left,"f"]\&\&\\
    #1\ar[ur,"#4"]\ar[dr,"#5"]\&\&#2\oplus_{#1} #3\ar[r,dashed,"\vphi"]\&Z\\
    \&#3\ar[ur,"\iota_{#3}"']\ar[rru,bend right,"g"']\&\&
    \end{tikzcd}
}
\newcommand{\exactshort}[5]{
		\begin{tikzcd}[ampersand replacement = \&]
			0\ar[r]\&#1\ar[r,"#2"]\& #3 \ar[r,"#4"]\& #5 \ar[r]\&0
		\end{tikzcd}
}
\newcommand{\product}[6]{
		\begin{tikzcd}[ampersand replacement = \&]
			#1 \& #2 \ar[l,"#4"'] \\
			#3 \ar[u,"#5"] \ar[ur,"#6"']
		\end{tikzcd}
}
\newcommand{\coproduct}[6]{
		\begin{tikzcd}[ampersand replacement = \&]
			#1 \ar[r,"#4"] \ar[d,"#5"'] \& #2 \ar[dl,"#6"] \\
			#3
		\end{tikzcd}
}
%%%%%%%%%%%% PAGE FORMATTING %%%%%%%%%

\usepackage{geometry}
    \geometry{
		left=15mm,
		right=15mm,
		top=15mm,
		bottom=15mm	
		}

\usepackage{color} % to do: change to xcolor
\usepackage{listings}
\lstset{
    basicstyle=\ttfamily,columns=fullflexible,keepspaces=true
}
\usepackage{setspace}
\usepackage{setspace}
\usepackage{mdframed}
\usepackage{booktabs}
\newcommand*{\oo}{{\infty}}
\DeclareMathOperator{\cl}{cl}
\usepackage[document]{ragged2e}
\usepackage{amsmath}
\pagestyle{fancy}{
	\fancyhead[L]{Fall 2022}
	\fancyhead[C]{221A - General Topology}
	\fancyhead[R]{John White}
  
  \fancyfoot[R]{\footnotesize Page \thepage \ of \pageref{LastPage}}
	\fancyfoot[C]{}
	}
\fancypagestyle{firststyle}{
     \fancyhead[L]{}
     \fancyhead[R]{}
     \fancyhead[C]{}
     \renewcommand{\headrulewidth}{0pt}
	\fancyfoot[R]{\footnotesize Page \thepage \ of \pageref{LastPage}}
}

\title{220A - Groups}
\author{John White}
\date{Fall 2022}




\begin{document}

\section*{Lecture 1}

\defn

A \underline{metric space} is a set $X$ equipped with a function $d:X\times X \to \R$, which satisfies the following axioms:
\begin{enumerate}
	\item For any $x, y \in X$, $d(x, y) = d(y, x)$
	\item For any $x, y, z \in X$, we have $d(x, y) \leq d(x, z) + d(z, y)$. This is called the ``triangle inequality"
	\item For any $x, y \in X$, $d(x, y) = 0$ exactly when $x = y$

\end{enumerate}

\exm

For $x, y \in \R^n$, 
\[
d(x, y) \eqdef \left(\sum_{i=1}^n(x_i - y_i)^2\right)^{\frac{1}{2}}
\]

This is called the Euclidean distance. 2 can be replaced with any real $r \geq 1$, and it will still be a metric. 

\exm 

In this example, $C[0, 1]$ is the set of all continuous functions $f:[0, 1]\to\R$. Here, 
\[
d(f, g) \eqdef \sup_{x\in[0,1]}|f(x)- g(x)|
\]

\exm

Let $X = \N$, the natural numbers, including $0$. Let $p$ be a fixed prime number. The $p$-adic metric on $\N$ is defined by 
\[
d_p(a, b) \eqdef \frac{1}{p^\alpha} 
\]
Where $p^\alpha$ is the largest power of $p$ which divides $|a - b|$. So two naturals are ``close" if their difference is divisible by a high power of $p$. 

\claim This is a metric

\proof
The 1st and 3rd axioms are clear. So we must prove the triangle inequality. We will consider the three quantities $d_p(a, b), d_p(a, t)$, and $d_p(b, t)$, where $a, b, t \in \N$. 

Suppose $p^\beta$ divides both $a - t$ and $t - b$. Then $p^\beta$ divides $(a - t) + (t - b) = a - b$. Therefore, 
\[
d_p(a, b) \leq \frac{1}{p^\beta} \leq \max(d_p(a, t), d_p(t, b)) \leq d_p(a, t) + d_p(t, b)
\]
\qed

\defn

Let $(X, d_X), (Y, d_Y)$ be two metric spaces. For a function $f:X\to Y$, we say that $f$ is \underline{continuous} at $x_0 \in X$ if, for all $\epsilon > 0$, there is a $\delta > 0$ such that 
\[
0 < |d_X(x_0, x)| < \delta \implies |d_Y(f(x_0), f(x))| < \epsilon
\]

A function $f:X\to Y$ is said to be continuous if it is continuous at $x$ for all $x \in X$. 

\exm

Consider a map $(\N, d_5) \to (\N, d_5)$ defined by 
\[
x\mapsto x^2
\]

Is this continuous? 

At 0, to be continuous, then for any $x$, if we want to get within a small distance of 0, then $x$ has to be divisible by large powers of $5$. 

What about at $11$?

This is continuous. 

\exm

What about $(\N,d_5) \to (\N, d_{17})$. 

\section*{Lecture 2}

\thm

If $f:(X, d_X)\to (Y, d_Y)$ and $g:(Y, d_Y)\to(Z, d_Z)$ are both continuous, then $g \circ f:(X, d_X)\to(Z, d_Z)$

\proof

Fix $x \in X$ and $\varepsilon>0$. Choose $\delta_1 > 0$ so that if $d_Y(f(x), y) < \delta_1$, then $d_Z(gf(x), g(y)) < \varepsilon$. 

By continuity of $f$, we may then choose a $\delta_0 > 0$ such that if $d_X(x, x') < \delta_2$, then $d_Y(f(x), f(x')) < \delta_1$. 

\qed

\defn For a metric space $(X, d_X)$, and a real $r > 0$, the open $r$-ball around a point $x$ is defined as
\[
B_r(x) = \{x' \in X \mid d(x, x') < r\}
\]

Exercise: State and prove some theorem about the existence of a function from $X \times X' \to Y \times Y'$, given a function $f:X\to Y$ and $g:X'\to Y'$. 

\exm{Balls}

\begin{enumerate}
\item In $\R^2$, consider 
\[
d_r\left(\begin{pmatrix} x_1 \\ y_1 \end{pmatrix}, \begin{pmatrix} x_2 \\ y_2 \end{pmatrix} \right) = \left(\sum_{i=1}^2 (x_i - y_i)^r\right)^{\frac{1}{r}}
\]

For $r = 2$, this is the usual euclidean distance. For $r = 1$, the balls look like diamonds. In the limit, as $r\to \infty$, the metric will approach what is known as the ``box metric," in which the distance between any point and 0 is it's largest coordinate. 
\item On $C[0, 1]$, the set of continuous functions from $[0, 1]$ to $\R$, we have the sup metric: 
\[
d(f, g) = \sup_{x\in[0, 1]}|f(x) - g(x)|
\]

\item We also have
\[
d_1(f, g) = \int_0^1d|f(x) - g(x)|dx
\]



\end{enumerate}

\defn

For a metric space $(X, d_X)$, suppose that $U \subseteq X$ is said to be ``open" if, for any $x \in U$, there is a $\varepsilon>0$ such that $B_\varepsilon(x) \subseteq U$. 

\lem 

$B_\varepsilon(x)$ is always open. 

\proof

Let $y \in B_\varepsilon(x)$. Let $t = d(x, y)$. By construction, $t <\varepsilon$. Let $\delta = \varepsilon - t$. Consider $B_\delta(y)$. For any $z \in B_\delta(y)$, we have by the triangle inequality
\[
d(x, z) \leq d(x, y) + d(y, z) < \varepsilon - t + t = \varepsilon
\]
and so $z \in B_\varepsilon(x)$. $z$ was arbitrary, so we are done.

\qed

\section*{Lecture 3}

\defn Let $(X, d)$ be a metric space. A set $U\subseteq X$ is said to be \underline{open} if for all $x \in U$, there is an $\varepsilon > 0$ such that $B(x, \varepsilon) \subseteq U$. 

\thm Let $\{U_\alpha\}_{\alpha\in A}$ be a collection of open sets. Then

\begin{enumerate}
\item $\cup_{\alpha\in A}U_\alpha$ is open. 
\item Let $U_1, \dots, U_n$ be a finite subcollection. Then $\cap_{i=1}^\oo U_i$ is open. 
\end{enumerate}

\proof




\section*{Lecture 4}

\defn 

Two metrics $d_1, d_2$ are said to be \underline{equivalent} on a space $X$ if any set which is open under the $d_1$-induced topology is also open under the $d_2$-induced topology, and vice versa. 

\exm 

In $\R^2$, 
\begin{align*}
d_2(0, (x, y)) & = (x^2 + y^2)^{\frac{1}{2}}\\
d_1(0, (x, y)) & = |x| + |y| \\
d_{\infty}(0, (x, y)) & = \max\{|x|, |y|\} \\
\end{align*}

How do these metric's unit balls compare? In fact, $d_1$'s is within $d_2$'s, which is within $d_{\oo}$'s.

But all of these balls contain a ball of radius $\frac{1}{2}$ in any of the three metric. Thus, these are equivalent. 

\defn

Two metrics $d_1, d_2$ are called \underline{Lipschitz equivalent} if there exists some $k \in \R$ such that, for all $x, y \in X$, we have

\[
\frac{1}{k}d_2(x, y) 
< d_1(x, y) < kd_2(x, y)
\]

\exm

This is a non-example. The 5-adic and the 17-adics are not equivalent. 

\exm

Let 
\begin{align*}
d_1(f, g) & = \int_0^1|f(x) - g(x)|\,dx \\
d_\oo(f, g) & = \sup_{x\in[0, 1]}|f(x) - g(x)| \\
\end{align*}

One controls for area, one controls for the maximum value of $f$. We have
\[
B^{d_\oo}_\varepsilon (0) \subset B^{d_1}_\varepsilon(0)
\] 
This is because if we control for the maximum size of $f$, we can surely control for the area under it. However, no matter how much we limit the area under $f$, there is some $f$ which has that much area which has a sup greater than some $\varepsilon$ which is fixed. 

\thm

Let $d_1, d_2$ be equivalent metrics on $X$. Then the following are equivalent
\begin{enumerate}
\item $f:X\to Y$ is $d_1, d_Y$ continuous if and only if it is $d_2, d_Y$ continuous. 
\item $g:Z\to X$ is $d_Z, d_1$ continuous if and only if if is $d_Z, d_2$ continuous. 
\end{enumerate}

\proof

\begin{enumerate}
\item Let $U \subset Y$ be open. We know that the preimage $f^{-1}(U)$ is open under the $d_1$ metric. But by the equivalence of $d_1, d_2$, $f^{-1}(U)$ must also be open under the $d_2$ metric. The reverse argument also holds. 
\item Let $U \subset X$ be an open set under the $d_1$ metric. By continuity of $g$, we know $g^{-1}(U)$ is open. However, $U$ must also be open under the $d_2$ metric, meaning that $g$ must be continuous with respect to both metrics. 
\end{enumerate}

\qed

Recall: If $\{U_\alpha\}_{\alpha\in A}$ is a collection of open sets, then $\cup_{\alpha\in A}U_\alpha$ is open. 

We can associate to any set $R\subset X$ an open set, called the interior of $R$. 

\defn

For any $\R\subset X$, we define it's \underline{interior} by 
\[
\operatorname{int}(R) = \bigcup_{U\text{ open}, U \subseteq R}U
\]

We can say several things about $\operatorname{int}(R)$. 
\begin{enumerate}
\item $\operatorname{int}(R)$ is open. 
\item If $U$ is open, then $\operatorname{int}(U) = U$, and vice versa. 
\item Suppose $A \subseteq B$. Then $\operatorname{int}(A) \subseteq \intr(B)$. 
\end{enumerate}

Recall: If $\{C_\alpha\}_{\alpha\in A}$ are all closed, then $\cap_{\alpha\in A}C_\alpha$ is closed. 

\defn 

If $R\subseteq X$, then the \underline{closure} of $R$, denoted by $\bar{R}$, or sometimes $\cl(R)$, is defined as 
\[
\bar{R}\eqdef \bigcap_{C\text{ closed}, C\supseteq R}C
\]

Analagously, 
\begin{enumerate}
\item $\bar{R}$ is closed for any $R$. 
\item $R$ is closed if and only if $R = \bar{R}$. 
\item If $A \subset B$, then $\cl(A) \subset \cl(B)$. 
\end{enumerate}

\prop

Let $x \in \cl(R)$. Then, for all $\varepsilon>0$, $B_\varepsilon(x)\cap R \neq \varnothing$, and vice-versa. 

We will prove this next lecture. 

\section*{Lecture 4}

\proof 

First, suppose that $x\not\in \bar{A}$. The complement of $\bar{A}$ is open, so there exists a $\varepsilon>0$ such that $B_{\varepsilon}(x)\subseteq (\bar{A})^c$, and so $B_{\varepsilon}(x) \cap \bar{A} = \varnothing$. 

Now, suppose that there exists some $\varepsilon> 0$ such that $B_{\varepsilon}(x) \cap A = \varnothing$. Then $(B_{\varepsilon}(x))^c$ is a closed set containing $A$ which does not contain $x$, thus $\bar{A}$ cannot contain $x$. 

\qed

Now, it is time for the main event. 

\subsection*{\underline{Topological Spaces}}

Let $X$ be a set. Let $\ms{P}(X)$ denote the power set of $X$, which is the set of all subsets of $X$. 

\defn

Let $\ms{T} \subset\ms{P}(X)$. $\ms{T}$ is a \underline{topology on $X$} if it has the following properties: 
\begin{enumerate}
\item $\varnothing, X \in \ms{T}$. 
\item If $\{U_\alpha\}_{\alpha\in A}$ with each $U_\alpha \in \ms{T}$, then $\bigcup_{\alpha\in A}U_\alpha \in \ms{T}$. 
\item If $A, B \in \ms{T}$, then $A \cap B \in \ms{T}$. Of course, we can ``strengthen" this to the equivalent statement: if $U_1, \dots, U_k \in \ms{T}$, then $\cap_{i=1}^kU_i \in \ms{T}$
\end{enumerate}

Elements of $\ms{T}$ are called ``open sets." 

\exm Here are some simple examples. 


\begin{enumerate}
\item Open sets in $(X, d)$ form a topology (hence the definition)
\item $\ms{T} = \ms{P}(X)$ forms the discrete topology. 
\item $\ms{T} = \{\varnothing, X\}$ forms the indescrete topolgy.
\item If $X = \{0, 1\}$, then $\ms{T} = \{\varnothing, X, \{0\}\}$ forms a topology. 
\item We can form the \underline{Zariski topology} on $\R$ by specifying the closed sets, which satisfy a similar but slightly different set of axioms. We define the closed sets to be $\varnothing, \R$, and any finite collection of points. 

More generally, the Zariski topology on some ring $R$ is specified by its closed sets, which are the solution locii of some set of polynomials in $R$. 
\item Let $X = \R$, $\ms{T} = \{(r, \oo)\mid r\in\R\} \cup \{\varnothing, \R\}$. 
\end{enumerate}

We have now defined a set of objects (topological spaces). Now, we want to define morphisms, the maps between spaces. 

\defn Let $(X, \ms{T}_X), (Y, \ms{T}_Y)$ be topological spaces. We say a function $f:X\to Y$ is \underline{continuous} if, for all $V\in\ms{T}_Y$, $f^{-1}(V) \in \ms{T}_X$. In other words, the preimage of any open subset of $Y$ is an open subset of $X$. 

\exm

\begin{enumerate} Some baby examples
\item Any function $f:(X, \text{discrete})\to (Y, \ms{T}_Y)$ is continuous as long as $X$ has the discrete metric, as the preimage of any open set will be a subset of $X$, all of which are open under the discrete topology. 
\item Any function $f:(X,\ms{T}_X)\to(Y,\text{ discrete})$ is continous for a similar reason. 
\item $\Id_X:(X, \ms{T}_X)\to(X,\ms{T}_X)$ is continuous. 
\end{enumerate}

\thm 

Let $f:(X,\ms{T}_X)\to(Y,\ms{T}_Y)$ and $g:(Y,\ms{T}_Y)\to(Z,\ms{T}_Z)$ be continuous functions. Then $g \circ f:(X,\ms{T}_X)\to(Z,\ms{T}_Z)$ is continuous. 

In other words, the composition of continuous functions is continuous. 

\proof

Pick $W \in \ms{T}_Z$. Then $g^{-1}(W) \in \ms{T}_Y$ since $g$ is continuous. So, because $f$ is continuous, 
\[
(g\circ f)^{-1}(W) = f^{-1}(g^{-1}(W)) \in\ms{T}_X
\]
Hence, $g \circ f$ is continuous. 

\defn 

For $\ms{T}$ a topology on $X$, a \underline{basis} for $\ms{T}$ is a subset $\ms{B}\subseteq\ms{T}$ such that every set in $\ms{T}$ is the union of sets in $\ms{B}$. 

\section*{Lecture 5}

\prop

Suppose $f:(X, \ms{T}_X) \to (Y, \ms{T}_Y)$ is a function. Then $f$ is continuous if and only if $f^{-1}(B) \in \ms{T}_X$ for all $B \in \ms{B}$, with $\ms{B}$ a basis for $\ms{T}_Y$.

\proof

The forward direction is trivial, as a basis consists of sets which are all open. 

Now, suppose that $f^{-1}(B) \in \ms{T}_X$ for all $B \in \ms{B}$. Let $V\in \ms{T}_Y$. Write it as $\cup_{\alpha\in A}B_\alpha = V$ for $B_{\alpha}\in \ms{B}$. 

Then $f^{-1}(V) = f^{-1}(\cup_{\alpha\in A}B_\alpha) = \cup_{\alpha\in A}f^{-1}(B_\alpha)$. This is the union of open subsets of $X$, so $f^{-1}(V)$ is open. 

\qed

\exm

Consider $B_{\frac{p}{q}}(\frac{r}{s})$, with $p, q, r, s \in \Z$, $q, s \neq 0$. In other words, the balls of rational radius and rational center. This collection of sets forms a basis for $\R$ using the standard (or, as Darren calls it, the ``Mother's Knee") topology. 

Exercise: Write down the definition of interiors and closures, and check the following lemma is true: 

\lem 

$x \in \cl(A)$ if and only if, for all open $U \ni x$, $U\cap A \neq \varnothing$. 

\exm

Consider $(\N,$ Zariski). What is the closure of the collection of prime numbers under this topology? 

Recall that the Zariski topology on $\N$ defines closed sets to be either empty, $\N$, or finite. So, for example, the closure of the integers from $1$ to $10$ is itself. 

But the only closed set that can contain an infinite set is $\N$, so the closure of the primes is $\N$. More generally, for any infinite subset of $\N$, the closure is $\N$. 

\subsection*{\underline{Subobjects and product objects}}

\subsubsection*{\underline{Subobjects}}

We have now defined our objects and morphisms, so let's talk about subobjects and product objects. 

Let $(X, \ms{T}_X)$ be a topological space, and $A\subseteq X$ a nonempty subset. 

Denote the inclusion map $\iota:A\to X$. We want to topologize $A$ so that $\iota$ is continuous, and is as small as possible, in the sense that any topology for which $\iota$ is continuous includes this topology on $A$. 

For any open $U\subseteq X$, we want $\iota^{-1}(U)$ to be open. But $\iota^{-1}(U) = U \cap A$. 

\defn

For $(X, \ms{T}_X)$ a topological space and $A\subset X$ a subset, then the \underline{subspace topology} on $A$, $\ms{T}_A$, consists of 
\[
\ms{T}_A \eqdef \{U \cap A \mid U \in \ms{T}_X\}
\]

\prop

Suppose we have a commutative diagram of the form 
\[
\begin{tikzcd}
(Z, \ms{T}_Z) \arrow[rd, "\iota \circ g"'] \arrow[r, "g"] & A \arrow[d, "\iota"] \\
& X \\
\end{tikzcd}
\]

Then $\iota$ is continuous if and only if $\iota \circ g$ is continuous. 

\proof

One direction is trivial, as we know the composition of continuous functions is continuous. Now, suppose that $\iota \circ g$ is continuous. 

Let $W \in \ms{T}_A$. We know $W = W^*\cap A$ for some $W^* \in \ms{T}_X$. $\iota \circ g$ is continuous, so $(\iota \circ g)^{-1}W^*\in\ms{T}_Z$. 

So, $\ms{T}_Z\ni(\iota\circ g)^{-1} = g^{-1}(\iota^{-1}(W^*)) = g^{-1}(W)$. 

\qed

So we can see that if we want the above to hold, we are forced into our definition of $\ms{T}_A$. 

\prop

$\ms{T}_A$ is the only topology so that the previous proposition is true for all spaces $(Z, \ms{T}_Z)$ and functions $g$. 

\proof

Suppose that $\ms{T}_{\mu}$ ($\mu$ for ``mystery" topology) such that the previous proposition holds for all choices of $(Z, \ms{T}_Z)$ and $g:(Z,\ms{T}_Z)\to A$.  

Consider 
\[
\begin{tikzcd}
(A,\ms{T}_A)\arrow[rd, "\iota \circ \Id"'] \arrow[r, "\Id"] & (A, \ms{T}_\mu) \arrow[d, "\iota"] \\
& (X, \ms{T}_X) \\
\end{tikzcd}
\]

\section*{Lecture 6}

We are trying to prove that $\ms{T}_A$ is the unique topology on $A$ such that
\[
\begin{tikzcd}
(A, \ms{T}_A) \arrow[rd,"\iota\circ\Id"']\arrow[r, "\Id"] & (A, \ms{T}_\mu)\arrow[d, "\iota"] \\
& (X, \ms{T}_X) \\
\end{tikzcd}
\]

\proof

Suppose $\ms{T}_\mu$ is such a topology on $A$. Then 
\[
\begin{tikzcd}
(A, \ms{T}_A)\arrow[rd, "\iota\circ\Id"']\arrow[r,"\Id"] & (A, \ms{T}_\mu) \arrow[d, "\iota"] \\ & (X, \ms{T}_X) 
\end{tikzcd}
\]
We have $\iota\circ\Id$ is continuous, so $\begin{tikzcd}( A,\ms{T}_A)\arrow[r, "\Id"] & (A,\ms{T}_\mu) \end{tikzcd}$ is continuous. So if $W\in\ms{T}_\mu$, then $W \in \ms{T}_A$. Therefore $\ms{T}_\mu\subseteq\ms{T}_A$. 

We know that $\iota\circ\Id$ is continuous. Let $W \in \ms{T}_X$. By continuity, $(\iota\circ\Id)^{-1}W\in\ms{T}_\mu \implies(\Id)^{-1}\circ\iota^{-1} W \in \ms{T}_\mu$. 

So then $(\Id)^{-1}(W\cap A)\in\ms{T}_\mu$, and $W\cap A \in \ms{T}_\mu$. Therefore for any $W \in \ms{T}_\mu$, $W \cap A \in \ms{T}_\mu$, so $\ms{T}_A\subset\ms{T}_\mu$. 

\subsubsection*{\underline{Product Objects}}

Let $(X, \ms{T}_X), (Y,\ms{T}_Y)$ be topological spaces. We want to topologize $X\times Y$. 

Well, if that holds a topology, then the projections damn well better be continuous. In other words, we want to arrange such that
\[
\begin{tikzcd}
X\times Y \arrow[d, "\rho_Y"'] \arrow[r, "\rho_X"] & X \\
Y & \\
\end{tikzcd}
\]
both $\rho_X, \rho_Y$ are continuous. 

Let $U\in\ms{T}_X$. Then $\rho_X^{-1}U = U \times Y$. Similarly, for $V\in\ms{T}_Y$, $\rho_Y^{-1}V = X\times V$. 

Thus, for the projections to be continuous, we need that the intersection of $U\times Y$ and $X \times V$ are in $\ms{T}_{X\times Y}$ for all $U\in\ms{T}_X, V\in\ms{T}_Y$. The sets of this form, $X\times V \cap U\times Y$, form a basis for a topology. 

In other words, the product topology has basis $\{U\times V\mid U\in\ms{T}_X, V\in\ms{T}_Y\}$. 

\thm
Consider the commutative diagram

\begin{center}
\begin{tikzcd}
	& Z \ar[ld,"\rho_X\circ g"'] \ar[rd,"\rho_Y\circ g"] \ar[d,"g"]& \\
X & X\times Y \ar[l,"\pi_j"] \ar[r,"\pi_Y"']& Y
\end{tikzcd}
\end{center}

Then $g$ is continuous if and only if $\rho_X\circ g$ and $\rho_Y\circ g$ are continuous. 

\proof

Next time

\section*{Lecture 7}

\subsection*{\underline{Relevant digression}}

Let $(X_i, \ms{T}_i)_{i\geq1}$ be a family of topological spaces. A basis for a topology on $\prod_{i\geq1}X_i$ is $\{\prod_{i\geq1}u_i \mid u_i \in \ms{T}_i\}$

We can ask that the projection $\prod_{i\geq1}X_i\to X_j$ is continuous for each $j$, but then 
\begin{align*}
p_j^{-1}(V_j) & = X_1\times\cdots\times X_{j - 1}\times V_j \times X_{j + 1} \times \cdots \\
\end{align*}

\defn

Let $(X_i, \ms{T}_i)_{i\geq1}$ be a family of topological spaces. Then the 

\underline{Tychonoff's product topology} has a basis consisting of sets of the form $\prod_{i\geq1}V_i$, where $V_j \subseteq X_j$, and $V_j = X_j$ for all but finitely many $j$. 

\exm

Let $p$ be prime. Then $\Z/p^k$ will denote $\Z/p^k\Z$. We may equip it with the discrete topology. Consider $\prod_{k\geq1}\Z/p^k$. Let's put a metric on this. We will define
\[
d(x, y) = \sum_{k\geq1}\frac{1}{2^k}d_k(x_k, y_k)
\]
It is easy to convince yourself this is a metric. 

Note that there exists a map $\Z\to\prod_{k\geq1}\Z/p^k$, given by $x \mapsto (x\mod p, x\mod {p^2}, \dots)$

An integer $m$ is close to zero in this metric if $m$ is divisible by large powers of $p$. 

This is (basically) the $p$-adic metric. 

\underline{Exercise}: What is the closure of $\Z$ in this metric space? 

\subsection*{\underline{Hausdorff Spaces}}

\defn

Here is what Darren calls a ``reasonable definition" of convergence of a sequence in a general topological space. 

Let $x_n$ be a seequence in $(X,\ms{T})$. We say that $x_n$  \underline{converges to} $x$ if, for any $x \in U \in \ms{T}$, there exists an $N$ such that $x_k \in U$ for all $k \geq N$. 

\exm

Let $\mc{P}$ be the set of prime numbers, and let $\N$ have the Zariski topology. What does $\mc{P}$ converge to? 

Consider $193$. An open neighborhood $U$ of this would be a set which contains $193$, and all but finitely many primes. So, for any such neighborhood, there will be an $N$ such that $p_k \in U$ for all $k \geq N$. So, the sequence $x_k = p_k$ converges to $193$. But $193$ was arbitrary, so the sequence converges to any natural number. 

The problem here is that the open sets are too big. The Hausdorff condition will get us around this. 

\defn

A topological space $(X, \ms{T})$, is said to be \underline{Hausdorff} if, for any $x, y \in X$, when $x \neq y$, there exists $U_x, U_y \in \ms{T}$, such that $x \in U_x$, $y \in U_y$, and $U_x \cap U_y = \varnothing$. 

\lem


\begin{enumerate}[label=(\alph*)]
\item Metric spaces are Hausdorff
\item If $(x_n)$ has a limit in a Hausdorff space $(X, \ms{T})$, then it's unique 
\end{enumerate}

\proof
\begin{enumerate}[label=(\alph*)]
\item Pick $x \neq y$ in $(X, d)$ with $d(x, y) = \varepsilon>0$. Consider $B_{\frac{\varepsilon}{3}}(x)$ and $B_{\frac{\varepsilon}{3}}(y)$. These obviously are disjoint by the triangle inequality.
\item Let $x_n$ converge to $x$, and let $y \neq x$ be some other point besides $x$. Then, there is some neighborhood of $x$ which is disjoint from a neighborhood of $y$. Eventually, every $x_k$ is in this neighborhood, meaning none are in the neighborhood of $y$. Thus, $x_n$ cannot converge to $y$.  
\end{enumerate}

\defn

Let $(X, \ms{T}_X), (Y, \ms{T}_Y)$ be topological spaces. $X$ and $Y$ are said to be \underline{homeomorphic} if there exist continuous maps $f:X\to Y$, $g:Y\to X$, such that $g \circ f = \Id_X$ and $f \circ g = \Id_Y$.

\section*{Lecture 8}

\thm

\begin{enumerate}[label = (\alph*)]
\item Suppose $X, Y$ are Hausdorff spaces. Then $X\times Y$ is also Hausdorff. 
\item If $A \subseteq X$, and $X$ is Hausdorff, then $A$ is also Hausdorff under the subspace topology. 
\item If $(X, \ms{T}_X)$ is Hausdorff, and $X$ is homeomorphic to $(Y, \ms{T}_Y)$, then $Y$ is also Hausdorff. 
\end{enumerate}

\proof

\begin{enumerate}[label=(\alph*)]
\item Pick distinct points in $X \times Y$, $(x_1, y_1)$ and $(x_2, y_2)$. Either $x_1 \neq x_2$, or $y_1 \neq y_2$, or both. Without loss of generality, suppose $x_1 \neq x_2$. Because $X$ is Hausdorff, there are open sets $U_1, U_2 \subseteq X$, such that $x_1 \in U_1, x_2 \in U_2$, and $U_1 \cap U_2 = \varnothing$. Let $O_i = U_i \cap Y$. $(x_i, y_i) \in O_i$, and $O_1 \cap O_2 = \varnothing$, as no $x$-coordinate in $O_1$ is the same as any $x$-coordinate in $O_2$. 
\item Exercise
\item Pick $y_1 \neq y_2 \in Y$. Then there is a unique $x_1, x_2$ such that $f(x_i) = y_i$. Because $X$ is Hausdorff, there are disjoint open neighborhoods $x_1 \in U_1, x_2 \in U_2$. $f(U_1)$ and $f(U_2)$ are both open, as $f$ is a homeomorphic; each $f(U_i)$ contains $y_i$; finally, $f(U_1) \cap f(U_2) = \varnothing$, as $f$ was bijective. Thus, $Y$ is Hausdorff. 
\end{enumerate}

\qed

\subsection*{\underline{Compact Spaces}}

Compactness is a very good property that a space can have. For metric spaces, we can think of compactness as a generalization of being a finite collection of points. 

\exm (Exercise)

\begin{enumerate}[label=(\alph*)]
\item Let $(X, d)$ be a metric space, $|X|<\oo$. Given $\begin{tikzcd} X \ar[r, "f"] & X \end{tikzcd}$ such that $d(f(x), f(y)) \geq d(x, y)$ for all $x, y \in X$. Then $f$ is an isometry. Give yourself one line to prove this. 
\item Do the same thing for $(X, d)$ a compact metric space. 
\end{enumerate}

\defn

A topological space $(X, \ms{T})$ is called \underline{compact} if, for any open cover $X = \cup_{\alpha \in A}U_\alpha$, $U_\alpha \in \ms{T}$, there exists $\alpha_1, \dots, \alpha_k \in \alpha$ such that $\cup_{i\leq j\leq k}U_{\alpha_j} = X$. In other words, any open cover admits a finite subcover. 

\exm

\begin{enumerate}
\item Let $X$ be any set, and consider $\ms{T} = \{\varnothing, X\}$. This is trivially compact. 
\item $(\R, $Zariski) is compact easily. Given an open covering, pick a set, and this covers every point of $\R$ except finitely many points. For each of those points, it is covered by another element of the covering. Thus, we can produce a finite subcover. 
\item $(\R, \ms{T}_{MK})$. This is not compact, as $\R = \cup_{N\geq0}(-n, n)$. This is an open cover which clearly does not have a finite subcover. 
\end{enumerate}

\defn

A subset $C \subseteq (X, d)$ is \underline{bounded} if for any $\xi \in C$,, there is a $K_\xi$ such that $B_{K_\xi}(\xi)\supseteq C$. 

\lem

If $C \subseteq (X, d)$ is compact, then $C$ is bounded. 

\proof

Clearly, $C = \cup_{i=1}^\oo B_i(x)$ for any particular $x \in C$. This must have a finite subcover, so in particular there is a largest $n$ such that $C \subseteq B_n(x)$. 

\qed

\lem

Suppose $A \subseteq(X, \ms{T})$ is compact in the subspace topology with $X$ Hausdorff. Then $A$ is closed. 

\proof 

Take $\xi\not\in A$. We want to construct an open neighborhood of $\xi$ which is disjoint from $A$. 

For any point $a \in A$, there are open neighborhoods $U_a, V_a$, with $a \in U_a, \xi \in V_a$, with $U_a \cap V_a = \varnothing$. 
Do this for every $a \in A$. Then $\{U_a \cap A \mid a \in A \}$ is an open covering of $A$ in $\ms{T}_A$. $A$ is compact, so there exists a finite subcover $(U_{a_1} \cap A) \cup \cdots \cup (U_{a_k} \cap A) \supseteq A$. Each $U_{a_i}$ has a corrseponding $V_{a_i}$. Let $V^* = \cap_{1\leq i \leq k}V_{a_i}$. this is an open set, because the intersection of finitely many sets is open, and is disjoint from $A$, as it is disjoint from $U_{a_i}$ for each $i$. 

So, $\xi$ is in the interior of $A^c$. However, $\xi$ was arbitrary, so $A^c$ is open, so $A$ is closed. 

\qed

\section*{Lecture 9}

\thm (Heine-Borel)

$[a, b]$ is compact.

\section*{Lecture 10}

\thm 

Suppose $A \subseteq (X, \ms{T})$ is a closed subset of a compact space. Then $A$ is compact. 

%%%%EVIL%%%
\proof

Let $\{U_\alpha \cap A\}_{\alpha \in A}$ be an open (in the subspace topology) cover of $A$. Of course, $U_\alpha \in \ms{T}_X$ for all $\alpha$. 

Then $\{X \setminus A, U_\alpha\}_{\alpha \in A}$ is an open covering of $X$. By hypothesis, $X$ is compact, so this admits a finite subcover $\{X\setminus A, U_{\alpha_1} \cup \cdots \cup U_{\alpha_k}$ of $X$. 

Then $\{U_{\alpha_i} \cap A\}_{i=1}^k$ is an open cover of $A$. So, $\{U_\alpha \cap A\}_{\alpha \in A}$ admits a finite subcover. But this cover was arbitrary, so we may conclude that any such open cover of $A$ admits a finite subcover. 

Thus, $A$ is compact. 

\qed

\cor

Let $C \subseteq (\R, \ms{T}_{MK})$. Then $C$ is compact if and only if it is closed and bounded. 

\proof

Compact implies closed and bounded is easy. For the other direction, if $C$ is bounded, then it is contained in some interval $[a, b]$. By Heine-Borel, such an interval is compact. So then $C$ is a closed subset of $[a, b]$, which is compact, so $C$ is a closed subset of a compact space, hence compact. 

\rem

Darren has several remarks:

\begin{enumerate}
\item Closed and bounded does \underline{not} equate to compactness in a general metric space. 
\item This is a digression, but the above tells us that the Cantor set is compact.
\end{enumerate}

\defn

Let $(X, d)$ be a metric space. We say that $X$ is \underline{Sequentially Compact} if, for any sequence $(x_n) \in X^\N$, there is a convergent subsequence $(x_{n_k})$ of $(x_n)$. 

\thm 

Let $(X, d)$ be metric space. Then $X$ is compact if and only if it is sequentially compact. 

\proof of easy direction

Let $(x_n) \in X^\N$ be a sequence in $X$. If $(x_n)$ takes on some value $x \in X$ infinitely many times, we're done. 

So we can assume that doesn't happen. 

Suppose for contradiction that there is no convergent subsequence. For each $x \in X$, there is an $\varepsilon = \varepsilon(x) > 0$ such that $B_{\varepsilon(x)}(x)$ contains only finitely many points of $(x_n)$. 

Then $\bigcup_{x\in X}B_{\varepsilon(x)}(x)$ is an open cover, so there exists a finite subcover $B_{\varepsilon(x_1)}(x_1) \cup \cdots \cup B_{\varepsilon(x_n)}(x_n)$. But this implies there are only finitely many elements of our sequence, a contradiction. 

\qed

\defn

Given $\varepsilon>0$, an $\varepsilon$-net is a set of points $x_1, \dots, x_k \in X$ so that $B_{\varepsilon}(x_1) \cup \cdots \cup B_{\varepsilon}(x_k) = X$. 

\thm 

Suppose a metric space $(X, d)$ is sequentially compact. Then $X$ has a (finite) $\varepsilon$-net for every $\varepsilon > 0$. 

\proof

We proceed by contraposition. Suppose that there exists a bad $\varepsilon>0$, such that there is no finite $\varepsilon$-net. 

Pick a point $x_1$. By hypothesis, $B_{\varepsilon}(x_1)$ is \underline{not} all of $X$. So we can pick a point $x_2 \in B_\varepsilon(x_1)^c$. Then, by hypothesis, $B_\varepsilon(x_1) \cup B_\varepsilon(x_2)$ is \underline{not} all of $X$. We can keep doing this indefinitely, to get a sequence $(x_n) \in X^\N$, such that $d(x_i, x_j) \geq \varepsilon > 0$ for all $i \neq j$. 

Thus, $(x_n) \in X^\N$ is a sequence which admits no cauchy subsequence, meaning it admits no convergent subsequence, meaning that $X$ cannot be compact.

\qed

\section*{Lecture 11}

\lem

Suppose $(X, d)$ is sequentially compact. Then, for all $\varepsilon>0$, $(X, d)$ has a finite $\varepsilon$-net. 

We proved this last time. 

\defn

Let $(X, d)$ be a metric space, and $\ms{O} = \{U_\alpha\}_{\alpha\in A}$ an open covering. 

Then a \underline{Lebesgue Number} for $\ms{O}$ is a $\delta>0$ such that for all $x \in X$, $B_{\delta(X)}\subset U_\beta$ for at least one $\beta \in A, U_\beta \in \ms{O}$. 

\lem

If $(X, d)$ is sequentially compact, then every open covering has a Lebesgue Number.

\proof

Let $\ms{O} = \{U_\alpha\}_{\alpha\in A}$ be an open cover of $X$.  Suppose $\ms{O}$ has no Lebesgue number. $\frac{1}{2}$ is not a Lebesgue number, so there exists a point $x_2$ such that $B_{\frac{1}{2}}(x_2)$ is not inside any single $U_\alpha$. Next, $\frac{1}{3}$ is not a Lebesgue number, so $B_{\frac{1}{3}}(x_3)$ is not in any individual $U_\alpha$. We can continue, and generate a sequence $\{x_2, x_3, \dots\}$, such that $B_{\frac{1}{i}}(x_i)$ is not contained in any $U_\alpha$. 

By hypothesis, this sequence has a convergent subsequence $x_{n_k}$, which converges to $x \in X$. For any $\varepsilon>0$, $B_{\varepsilon}(x)$ contains infinitely many $x_i$. In particular, for a $U_\alpha$ containing $x$, there is a $\varepsilon>0$ such that $B_{\varepsilon}(x)$ contains infinitely many of the $x_i$. But then eventually the ball of radius $\frac{1}{i}$ around $x_i$ will be entirely in this ball, contradicting the hypothesis. 

Explicitly, pick $n_k >>0$ such that $\frac{1}{n_k} < \frac{\varepsilon}{10}$, then $B_{\frac{1}{n_k}}(x_{n_k}) \subset B_\varepsilon(x) \subset U_\beta$. 

\qed

Now, we prove the converse of the theorem from the end of last lecture. 

\proof that sequential compactness implies compactness. 

Given $\ms{O} = \{U_\alpha\}_{\alpha\in A}$ an open covering. By the previous lemma, there is a Lebesgue number $\varepsilon^* > 0$ for $\ms{O}$. By another previous lemma, there is a finite $\varepsilon^*$-net for $\ms{O}$. 

That is, there are points $\{x_1, \dots, x_k\}$, such that $B_{\varepsilon^*}(x_1) \cup \cdots \cup B_{\varepsilon^*}(x_k)$. But each of these lies inside one of the $U_\alpha$, so the corresponding $U_\alpha$ yields a finite subset of $\ms{O}$. 

\qed

\exm

Consider $(C[0, 1], \sup)$. Let $f_0$ be the zero function, and consider $\bar{B_1(f_0)}$. This is closed, and bounded. However, it is not compact. 

Consider a function which is zero, except for a spike at some point. Each such function is distance 1 from $f_0$, but is also distance 1 from each other such function. A sequence of such functions can't have a convergent subsequence. 

So, this ball is not sequentially compact, and thus not compact. 

\exm

Consider the infinite product
\[
X = \prod_{k\geq1}\Z/p^k
\]
where $p$ is a prime fixed beforehand. Earlier, we defined a metric by
\[
d(x, y) = \sum_{k\geq1}\frac{1}{2^k}d_k(x_k, y_k)
\]
This is sequentially compact. Can you see why?

\underline{Exercise}: Write this down. 

\underline{Exercise}: Suppose $f:(X, d_X) \to (Y,d_Y)$ is continuous, and $X$ is compact. Prove $f$ is uniformly continuous. i.e. given $\varepsilon>0$, there is a $\delta>0$ such that
\[
d_X(x, x') < \delta \implies d_Y(f(x), f(x')) < \varepsilon
\]
for \underline{any} $x, x' \in X$. 

\section*{Lecture 12}

\thm

Let $(X, \ms{T}_X)$, $(Y, \ms{T}_Y)$ be compact. Then $(X\times Y, \ms{T}_{X\times Y})$ is compact. 

\rem

Suppose $(X, d_X), (Y, d_Y)$ are metric spaces, both compact. Then $(X\times Y, d_X + d_Y)$ is compact. This is easy, as $X\times Y$ in this case is sequentially compact. 

\proof

Suppose $\{U_\alpha\}_{\alpha \in A}$ is an open covering of $X\times Y$.

Fix $y \in Y$. Then $X\times\{y\}$ (with the subspace topology that $X\times Y$ induces) is homeomorphic to $X$. $X$ is compact, so $X\times\{y\}$ is compact. 

For each $x \in X$, $(x, y) \in U_\beta$ for some $\beta \in A$. $U_\beta\in\ms{T}_{X\times Y}$, so $U_\beta = \bigcup($open in $X) \times$(open in $Y)$ . 

So there exists $A_{(x, y)}\times B_{(x, y)} \subset U_\beta$, $A_{(x, y)} \in \ms{T}_X, B_{(x, y)}\in\ms{T}_Y$. 

$\{A_{(x, y)}\}_{x\in X}$ is an open covering of $X$, and thus admits a finite subcover, $A_{(x_1, y)}\cup\cdots\cup A_{(x_k, y)} = X$. For each $A_{(x_i, y)}$, there is a $B_{(x_1, y)}$. 

Set $B_y^* = \bigcap_{1\leq i \leq k}B_{(x_i, y)}$. Then the entire strip $X\times B_y^*$ is covered by a finite number of $U_\beta$.

$\{B_y^*\}_{y\in Y}$ is an open covering of $Y$, so by compactness of $Y$ admits a finite subcover $B_{y_1}^*\cup\cdots\cup B_{y_\ell}^*$. Each $B_{y_i}^*$ has the property that $X\times B_{y_i}^*$ is covered by finitely many $U_\beta$. So $X \times Y$ is covered by a finite number of $U_{\beta}$'s. 

The cover $\{U_\alpha\}_{\alpha \in A}$ was arbitrary, so we are done. 

\qed

Before we started this proof, Daren said ``First I'm gonna take a sip of the real stuff," and took a drink from his can of diet Coke. What did he mean by this? 

\subsection*{\underline{Connected Spaces}}

\defn 

A topological space $(X, \ms{T})$ is \underline{not connected} if there exist nonempty, disjoint, open sets $A, B$ such that $X = A\coprod B$. 

\defn

A topological space $(X, \ms{T})$ is \underline{not connected} if there is a continuous surjection $f:(X,\ms{T})\to\{0,1\}$.

\prop

The above two definitions are equivalent. 

\proof

First, suppose that there is a continuous surjection $f:(X,\ms{T})\to\{0, 1\}$. Then $f^{-1}(0), f^{-1}(1)$ are nonempty disjoint open subsets of $X$, whose union is $X$. 

Now, suppose that we can express $X$ as the union of two disjoint nonempty open sets $A, B$. Then $f(x) = \begin{cases} 1 & x \in A \\ 0 & x \in B \\ \end{cases}$ is a continuous surjection from $X$ to $\{0,1\}$. 

\lem

Suppose $X$ is connected, and $f:X\to Y$ is continuous. Then $f(X)$ is connected.

\proof

If there exists $\pi:f(X)\to\{0, 1\}$ which is continuous and surjective, then $\pi \circ f :X\to\{0, 1\}$ is continuous and surjective. 

\qed

\cor

Connectedness is a topological property. 

\thm

$X \times Y$ is a connected topological space if and only if $X$ and $Y$ are connected. 

\proof

If $X \times Y$ is connected, then $p_X:X\times Y \to X$ is continuous, so by the above lemma, $X$ is connected. A similar argument shows $Y$ is connected. 

Now, suppose $X, Y$ are connected. Let $f:X\times Y\to\{0,1\}$ be continuous. $X\times\{y\}$ is connected for any $y \in Y$, $f(X\times\{y\}) = \{0\}$ without loss of generality. 

\section*{Lecture 13}

\lem

$S \subseteq (\R, \ms{T}_{MK})$ is an interval if and only if it satisfies the inbetweenness (sp?) property: if $x, y \in S$, and $x < u < y$, then $u \in S$. 

\proof

One direction is easy.

Now, suppose that $S$ has the inbetweenness property. We will abuse notation somewhat by using square brackets for any interval, even if one of the bounds is $\pm\oo$. 

Let $\lambda = \inf S$, and $\nu = \sup S$. 

\claim

$[\lambda, \nu] \subseteq S \subseteq [\lambda, \nu]$. 

\proof

This is immediate. $S$ is certainly contained inside $[\lambda, \nu]$. For containment the other way, pick $u \in [\lambda, \nu]$. Then there exists an $x \in S$ such that $x < u$. Similarly, there is a $y \in S$ with $u < y$. So $u \in S$. 

To correctly interpret this proof, simply replace the brackets with either $($ or $[$. Make sure to pick the ones which make the statements correct. 

\qed

\thm

The connected subsets of $\R$ are precisely the intervals. 

\proof

Let $S \subset \R$. If $S$ is not an interval, then there is some $x, y, z \in \R$, with $x < y < z$, with $x, z \in S$,  $y \not\in S$. Then $(-\oo, y) \coprod (y, \oo)$ is a disconnection of $S$. So if $S$ is connected, it is an interval. 

Now, suppose that $S$ is an interval that is disconnected. That is, $S = A \coprod B$, $A, B$ open, nonempty, disjoint, etc. There is some $x \in A$ and $y \in B$, as by assumption they are nonempty. 

It will suffice to reduce to the case of $[x, y]$. So $(A \cap [x, y]) \coprod (B \cap [x, y])$ is a disconnection of $[x, y]$. Consider $d:(A\cap[x, y])\times(B\cap[x, y])\to \R$ which gives the distance between a point in $A \cap [x, y]$ and a point in $B \cap [x, y]$. This is the product of compact spaces, and thus compact. Therefore, the infimum is achieved, and must be greater than $0$. 

So there exist $\alpha \in A\cap[x, y], \beta \in B\cap[x, y]$, such that $d(\alpha,\beta) \leq d(a, b)$ for any $a \in A\cap[x, y], b \in B\cap[x,y]$. But $d(l, r) = |l - r|$ for any $l, r$. 

Take $\xi = \frac{1}{2}(\alpha + \beta)$. Then
\[
|\beta - \frac{1}{2}(\alpha + \beta)| = \frac{1}{2}|\beta - \alpha| < |\alpha - \beta|
\]
So $\frac{1}{2}(\alpha + \beta)\not\in B$. Similarly, $\frac{1}{2}(\alpha + \beta) \not\in A$, a contradiction. 

\qed

\thm

Suppose $A$ is connected, with $A \subseteq K \subseteq\bar{A}$. Then $K$ is connected. 

\proof

Suppose $K$ is not connected. Then there is a continuous, surjective function $f:K\to\{0, 1\}$. Because $A$ is connected, $f|_A:A\to\{0, 1\}$ is not surjective. Suppose there is a $k \in K$ such that $f(k) = 1$. Then $f^{-1}(1)$ is an open neighborhood of $k$. But, by the definition of the closure, $k$ $f^{-1}(1) \cap A \neq\varnothing$. But $f$ is identically $0$ on $A$, a contradiction. Thus, there is no such $f$. 

\qed

\exm

Time for a classic example. Consider $g:(0, \oo)\to\R^2$ given by $t \mapsto (t, \sin(1/t))$. This is the so-called ``topologists sine curve". This is continuous, so its image is connected as a subset of $\R^2$. Then, by the theorem, $\bar{\im g}$ is connected. Look at a picture. 

\section*{Lecture 14}

\defn

Let $(X, \ms{T})$ be a topological space. Define the equivalence relation $x \sim y$ if, for all $f:X\to\{0, 1\}$ continuous and onto, $f(x) = f(y)$. 

The equivalence classes are called \underline{connected components} of $X$. 

\exm

\begin{enumerate}
\item $[0, 1] \cup [2, 3]$ has two connected components. 
\item In $\Q$ the only connected components are singletons. A space with this property is said to be \underline{totally disconnected}.
\item By the lemma from last time (closure of connected set is connected), components are closed. 
\end{enumerate}

\prop

Let $\ms{C}(X, \ms{T})$ denote the components of $f$. Suppose that $f:X\to Y$ is a homeomorphism. Then there is a bijection $f_*:\ms{C}(X)\to\ms{C}(Y)$. 

\proof

For a point $x \in X$, define $C_x\in\ms{C}(X)$ as the component containing $x$. Then we define $f_*(C_x) = C_{f(x)}$. 

We need to show this is well defined. Suppose $x' \in C_x$. Then $f(x') \in f(C_x)$, which is a connected set that contains $f(x)$ and $f(x')$. So $f_*(C_x) = f_*(C_{x'})$. So this is well defined. 

\[
\begin{tikzcd}
\ms{C}(X) \ar[r, "f_*"] \ar[rd, "(g \circ f)_*"' ] & \ms{C}(Y) \ar[d, "g_*"] \\ & \ms{C}(Z) \\
\end{tikzcd} 
\]

I claim this commutes. 
\[
g_*f_*C_x = g_*C_{f(x)} = C_{g(f(x))} = (gf)_*(C_x)
\]

Further, $I_*$ is the identity, so we are done. 

\defn

Let $(X, \ms{T})$ be a topological space. We say it is \underline{path connected} if, for any $x, y \in X$, there exists a continuous $\sigma:[0, 1]\to X$ such that $\sigma(0) = x, \sigma(1) = y$. 

\lem

Suppose $X$ is a path connected topological space, and $f:X\to Y$ is a continuous function. Then $f(X) \subseteq Y$ is path connected.

\proof

Let $\xi, \eta \in f(X)$. Then there exist $x_\xi, x_\eta \in X$ such that $f(x_\xi) = \xi, f(x_\eta) = \eta$. By hypothesis, there is a continuous $\sigma:[0, 1]\to X$ such that $\sigma(0) = x_\xi, \sigma(1) = x_\eta$. Then $f\circ \sigma:[0, 1]\to f(X)$ is continuous, and $(f\circ \sigma)(0) = \xi, (f\circ \sigma)(1) = \eta$. 

\qed

\cor

Path connectedness is invariant under homeomorphism. 

\prop

Suppose $X$ is path connected. Then $X$ is connected. 

\proof

If $X$ is not connected, then $X = A\coprod B$, for $A, B \subseteq X$ open, disjoint, nonempty. Let $\alpha \in A, \beta \in B$. 

If there exists a path $\sigma:[0, 1]\to X$ with $\sigma(0) = a, \sigma(1) = \beta$, then $\sigma^{-1}(A) \coprod \sigma^{-1}(B)$ is a partition of $[0, 1]$. But we know this space is connected. 

\qed

The converse of the above proposition is false. 

Let $g:(0, \oo)\to\R^2$ be defined by 
\[
t\mapsto (t, \sin(\frac{1}{t}))
\]
%% This is a graph of the curve. It makes it take too long to compile, so I've commented it out, but here it is. 
%\begin{center}
%\begin{tikzpicture}[x=7cm]
%    \draw[xstep=.2,ystep=.5,lightgray,ultra thin] (-0.1,-1.5) grid %(1.1,1.5);
%    \draw[->] (0,0) -- (1.1,0) node[right] {$x$};
%    \draw[->] (0,-1) -- (0,1.1);
%  \draw[blue,domain=0.01:1,samples=5000] plot (\x, %. {sin((1/\x)r)});
% \end{tikzpicture}
% \end{center}

\claim

$\bar{\im(g)}$ is not path connected. 

\proof

Let $A$ be a point on the graph, and $(0, 0)$ the origin (which lies not in $\im(g)$, but in its closure). 

Suppose there is a continuous $\sigma[0, 1]\to \bar{\im(g)}$ with $\sigma(0) = A, \sigma(1) = (0, 0)$, and consider the projection onto the $y$-axis, $\rho_y$. $\rho_y \circ \sigma$ is uniformly continuous. 

Given $\delta = \frac{1}{3}$, there is a $\varepsilon = \frac{1}{n}$ such that $|x - x'| < \frac{1}{n}$, and $|\rho_y\sigma(x) - \rho_y\sigma(x')|$. 

To traverse between to zeroes of this function, you need to travel $\frac{6}{n}$. 

Alternatively, consider the sequence $x_n = \frac{1}{\frac{\pi}{2} + 2\pi n}$. We have
\[
\lim_{n\to\oo}\sigma(x_n) = (1, 1) \neq(0, 0) = \sigma(0) = \sigma(\lim_{n\to\oo}x_n)
\]

\thm

Let $U \subseteq (\R^n, \ms{T}_{mk})$ be open and connected. Then $U$ is path connected. 

\proof

Pick $\xi \in U$. Consider $A = \{\alpha\in U \mid \alpha$ can be connected to $\beta$ by a path$\}$, and let $B = A^c$.

\claim

$A, B$ are both open.

\proof

 Suppose $\alpha \in A$. Then there is an open ball around $\alpha$ lying entirely inside $U$, and these balls are path connected. So we must conclude that one of $A, B$ is empty. $A$ is not, so $A = U$. 

\section*{Lecture 15}

\subsection*{\underline{Completeness}}

Let $(X, d)$ be a metric space. Suppose $x_n\to x$. This means that given $\varepsilon>0$, there is an $N\in\N$ such that, for all $k \geq N$, $d(x_k, x) < \varepsilon$. 

\defn

A sequence $(x_n)_{n=1}^\oo$ in a metric space is said to be \underline{Cauchy} if, given $\varepsilon>0$, there is an $N \in \N$ such that for $k, \ell \geq N$, $d(x_k, x_\ell) < \varepsilon$.

It is an easy consequence of definitions that if $(x_n)$ converges, then it is Cauchy. The converse is not true in general. 

\exm

In $(\Q, \abs*{\cdot}_{MK})$, we can find a sequence of rationals which is Cauchy, and converges to $\sqrt{2}$, which is not rational.

\defn

A metric space $(X, d)$ is said to be \underline{complete} if every Cauchy sequence $(x_n)$ converges to some $x \in X$. 

\thm

$(\R,\abs*{\cdot}_{MK})$ is complete. 

\proof

Let $(x_n)$ be a Cauchy sequence in $\R$. 

Then $(x_n) \subset \R$ is bounded. So, there exists an $A \in \N$ such that $[-A, A] \supseteq \{x_n\}$. But $[-A, A]$ is compact, and therefore sequentially compact. So, $(x_n)$ has a convergent subsequence $\{x_{n_k}\}_{k=1}^\oo$. 

Because $(x_n)$ is Cauchy, it can be shown (as an exercise!) that if a Cauchy sequence has a convergent subsequence, then the whole sequence converges. 

\qed

\rem

Completeness is \underline{not} a topological invariant. For example, $(\R, \abs*{\cdot}_{MK})$ and $((0, 1), \abs*{\cdot}_{MK})$ are homeomorphic, but one is complete and the other is not. 

\prop

Let $(X, d_X), (Y, d_Y)$ be complete metric spaces. Then $(X \times Y,d_{X\times Y} = \max(d_X, d_Y)$. That is, $d_{X\times Y}((x_1, y_1), (x_2, y_2)) = \max(d_X(x_1, x_2), d_Y(y_1, y_2))$.

\proof

Think

\qed

\cor

$(\R, \abs*{\cdot}_{MK})$ is complete. 

\thm

Let $(X, d)$ be complete. Then if $Y \subseteq X$ is closed, then $(Y, d)$ is complete. 

\proof

Let $(y_n)$ be a Cauchy sequence in $Y$. By hypothesis, this converges to some $\lambda \in X$. So, for any open $U \ni \lambda$, $U$ contains points in $Y$. Thus, $\lambda \in \bar{Y}$. However, $Y$ is closed, so $Y = \bar{Y}$, so $\lambda \in Y$. 

\qed

\thm

Let $(X, d)$ be a metric space, and $Y$ is a complete subset of $X$. Then $Y$ is closed. 

\proof

Pick $y^\star \in \bar{Y}$. We will show $y^\star \in Y$. For any open ball containing $y^\star$, this ball intersects $Y$ nontrivially by the definition of $\bar{Y}$. So for each $n \in \N, B_{\frac{1}{n}}(y^\star)$ contains a point in $Y$. Let $y_n \in Y \cap B_{\frac{1}{n}}(y^\star)$. 

The sequence $y_n \to y^\star$, so it is Cauchy. But $Y$ is complete, so $\lim_{n\to\oo}y_n = y^* \in Y$. $y^\star$ was arbitrary, so $Y = \bar{Y}$, so $Y$ is closed. 

\qed

\thm

Let $(X, d)$ be compact. Then $(X, d)$ is complete. 

\proof

Let $(x_n)$ be a Cauchy sequence in $X$. In metric spaces, compactness is equivalent to sequential compactness, so this has a convergent subsequence. But the limit of this subsequence must be the limit of $(x_n)$, because $(x_n)$ is Cauchy. Thus, $(x_n)$ converges. $(x_n)$ was an arbitrary Cauchy sequence, so every Cauchy sequence in $X$ converges. Thus, $X$ is complete. 

\qed

\defn (Old definition)

A metric space $(X, d)$ is \underline{totally bounded} if given $\varepsilon>0$, there exist $x_1, \dot, x_k \in X$ such that 
\[
X = \cup_{1\leq i\leq k}B_\varepsilon(x_i)
\]

Recall that in a metric space, compactness implies total boundedness. 

\thm

If $(X, d)$ is complete and totally bounded, then it is compact. 

Proof will be next time.

\section*{Lecture 16}

\proof

We will show that $X$ is sequentially compact. Let $(x_n)$ be a sequence in $X$. Fix $\varepsilon=\frac{1}{2}$. By hypothesis, there exists a finite $\frac{1}{2}$-net $B_{\frac{1}{2}}(c_1) \cup \cdots \cup B_{\frac{1}{2}}(c_k) = X$. 

Wew can extract a subsequence $x^{(1)}_1, x^{(1)}_2, \dots$, such that one of these $\frac{1}{2}$-balls contains $x^{(1)}_i$ for all $i$.

Now, let $\varepsilon = \frac{1}{4}$. By hypothesis, there exists a subsequence of $(x^{(1)}_n)$, $x^{(2)}_1, x^{(2)}_2, \cdots$, with $x^{(2)}_i$ contained in the same $\frac{1}{4}$-ball. 

We continue this construction inductively, with $x^{(i)}_j$ being the $j$th element of a subsequence of $x$ lying entirely in the intersection of balls. 

Then $x^{(i)}_i$ is a Cauchy subsequence: $x^{(i)}_i, x^{(j)}_j$ lie in the same ball of radius $\frac{1}{2^{\max(i, j)}}$. That is, $d(x^{(i)}_i, x^{(j)}_j) \leq \frac{1}{2^{\min(i, j)}}$. So this sequence is a Cauchy subsequence of $(x_n)$. But $(x_n)$ was arbitrary, so we are done with the proof. 

\qed

\thm

Any metric space $(X, d)$ has a canonical completion $(\tilde{X}, \tilde{d})$. In other words, there exists an isometry $i:X\to\tilde{X}$ such that $\bar{i(X)} = \tilde{X}$ (i.e. $i(X)$ is \underline{dense} in $\tilde{X}$). We also stipulate that $\tilde{X}$ is complete. 

That is, if $Y \supset X$ is a complete metric space, then there is an injective $f:\tilde{X}\to Y$ which makes the following diagram commutative: 

\[
\begin{tikzcd}
& Y \\
X \ar[ur, "\iota"] \ar[r, "\iota_X"] & \ar[u, "f"] \tilde{X} \\
\end{tikzcd}
\]

\proof

\prop

Let $\ms{B}(X, \R) \eqdef \{f:X\to\R \mid f\text{ is bounded}\}$. This is a metric space, with metric $d(f, g) = \sup_{x\in X}|f(x) - g(x)|$. 

Then $\ms{B}(X, \R)$ is complete. 

\rem 

In this definition, we can replace $\R$ with any metric space, and $\sup_{x\in X}|f(x) - g(x)|$ with $\sup_{x\in X}d(f(x), g(x))$, and this remains true.

\proof of proposition.

Let $(f_k)$ be a Cauchy sequence of functions in $\ms{B}(X, \R)$. Fix $x \in X$. Then $\{f_k(x)\}$ is a Cauchy sequence in $\R$, and so converges to a point we will call $f(x)$. We need to show that $f(x)$ is a bounded function, and we need to show that $f_k \to f$ (we know this happens pointwise). 

To show that $f$ is bounded, take $\varepsilon = 1$. Then there is an $N \in \N$ such that $d(f_k, f_\ell)< 1$ for $k, \ell \geq N$. Therefore, $|f_k(x) - f_\ell(x)| < 1$ for all $x \in X$ by definition. 

In particular, $\sup_{x\in X}|f_N(x) - f_\ell(x)| < 1$ for all $\ell$. For a fixed $x\in X$, we can see $|f_N(x), f_\ell(x)| \leq 1$. Letting $\ell\to\oo$, we have $|f(x)| \leq 1 + $the bound for $f_N(x)$. So, $f$ is bounded. 

Now for the second part. We will show $f_n \to f$ in the sup metric. Darren will leave this as an exercise, but it is easy if you use the fact $d(f, g) = 0$ if and only if $f = g$ everywhere. 

\qed

Now we have proven the proposition, we can prove the theorem. 

We will isometrically embed $(X, d)\to(\ms{B}(X, \R), \sup$ metric). Our first guess is to map $x\mapsto f_x$, where $f_x(\xi) = d(x, \xi)$. The problem is that this is not necessarily a bounded function. 

Fix a base point $x^\star \in X$. Then we map $x\mapsto f_x$, where $f_x(\xi) = d(x, \xi) - d(\xi, x^\star)$. This is a bounded function (by triangle inequality, this is bounded above in magnitude by $d(x, x^\star)$). 

We will now show that $x \mapsto f_x$ is isometric. Let $x, y \in X$. What is $d(f_x, f_y)$ under the sup metric? By definition, it is given by 
\[
\sup_{\xi\in X}|f_x(\xi) - f_y(\xi)| = \sup_{\xi\in X}|d(x, \xi) - d(y, \xi)| 
\]
This last term is bounded above by $d(x, y)$. This supremum can be achieved by letting $\xi = x$ or $\xi = y$. So, $d(f_x, f_y) = d(x, y)$. So $x\mapsto f_x$ is an isometry, so we are done. 

\qed

\section*{Lecture 17}

\thm (Banach Fixed Point Theorem)

Let $(X, d)$ be a complete metric space, and let $f:X\to X$ be complete, with $d(f(x), f(y)) \leq kd(x, y)$ for all $x, y$, where $0 \leq k < 1$ (such an $f$ is called a $k$-contraction). Then $f$ has a unique fixed point, i.e. a point $p\in X$ such that $f(p) = p$. 

\proof

Let $x_0 \in X$, and let $x_n = f^{(n)}(x_0)$. 

\claim $\{x_n\}$ is Cauchy

\proof

Indeed, for $n \leq m$, 
\begin{align*}
d(x_n, x_m) & = d(x_n, x_{n + 1}) + d(x_{n + 1}, x_{n + 2}) + \cdots + d(x_{m - 1}, x_m) \\
			   & \leq kd(x_{n - 1}, x_n) + \cdots  \\
			   & \leq k^nd(x_0, x_1) + k^{n + 1}d(x_0, x_1) + \cdots + k^{m - 1}d(x_0, x) \\
				& = k^nd(x_0, x_1)(1 + k + \cdots + k^{m - 1 - n}) < \frac{k^n}{1 - k}d(x_0, x_1) \\
\end{align*}
$k < 1$, so as $n\to\oo$, this gets arbitrarily small, so $\{x_n\}$ is Cauchy, proving the claim. 

Because $(X, d)$ is complete, $x_n\to x$. Because of $d(f(x), f(y)) \leq kd(x, y)$ for all $x, y$, $f$ is uniformly continuous. So, $\lim_{n\to\oo}f(x_n) = f(\lim_{n\to\oo}x_n) = f(x)$. So, $f(x) = x$. 

The uniqueness comes from the fact that if $x, y$ are both fixed points, then the distance between them is multiplied by $k$. This can only happen if $k = 0$, or $d(x, y) = 0$, implying $x = y$. 

\qed

\cor

If $f:X\to X$ is a continuous function such that $f^{(t)}$ is a $k$-contraction for some $t$, then $f$ has a unique fixed point. 

\proof

By Banach, $f^{(t)}$ has a fixed point $x^\star$. But then $f^{(t)}\circ f(x^\star) = f^{(t + 1)}(x^\star) = f \circ f^{(t)}(x^\star) = f(x^\star)$. So $f(x^\star)$ is a fixed point of $f^{(t)}$. But the above theorem gives us that the fixed point of $f^{(t)}$ is unique, so $x^\star = f(x^\star)$, so $x^\star$ is a fixed point of $f$. 

\thm (Cantor)

Let $(X, d)$ be a metric space. Then $X$ is complete if for every sequence of nonempty closed sets $V_1\supseteq V_2 \supseteq V_3 \supseteq \cdots$, such that $\operatorname{diam}(V_j)\to 0$, there exists a point $x^\star \in \cap_{k\geq1}V_k$. 

\rem

The point $x^\star$ (if it exists) is unique, which is immediate from the condition that the diameter of $V_j$ goes to 0.

\proof

\underline{$<=$}

Suppose $(X, d)$ is not complete. Let $(x_n)$ be a Cauchy sequence which does not converge. Because it does not converge, it has no limit points, and hence the image of the sequence is closed. So, $V_k \eqdef \{x_k, x_{k + 1}, \dots \}$ is closed for any $k$. 

We can see that $V_k \supseteq V_{k + 1}$ for all $k$ easily, and the diameter goes to zero because it is Cauchy. 

However, the intersection of all these sets is empty. So, $X$ does not satisfy the given condition. 

\underline{$=>$}

Suppose $(X, d)$ is complete. Pick $x_k \in V_k$. Then $(x_k)$ is a Cauchy sequence, because the diameter of $V_j$ goes to zero. $X$ is complete, so $x_k$ converges to some $x$. 

\claim 

$x^\star \in \cap_{k\geq 1}V_k$. 

\proof

Each $V_k$ is closed, so, because $(x_k)$ eventually lies completely in any $V_j$, we must conclude that each $V_k$ contains $x^\star$. 

This completes the proof. 

\qed

\defn

Let $(X, d)$ be a metric space (although this definition doesn't require the metric). A subset $H\subseteq X$ is called \underline{nowhere dense} if $\operatorname{int}(\cl(H)) = \varnothing$. 

\thm (Baire)

Let $(X, d)$ be a complete metric space, and suppose $\{H_n\}_{n\geq1}$ is a countable collection of nowhere dense sets. Then $X\setminus\left(\cup_{k\geq1}V_k\right)$ is dense. 

We will prove this next time. 

\section*{Lecture 18}

\lem

Let $(X, d)$ be a metric space. Then a subset $H$ is nowhere dense if and only if for all nonempty open $U \subset X$, $U$ contains a ball disjoint from $H$. 

\proof

\subsubsection*{\underline{$<=$}}

We want to show that $\operatorname{int}(\cl(H)) = \varnothing$.

Let $y \in U$, $\delta>0$. By hypothesis, there is some $z \in B_{\delta}(y)$, and $\varepsilon>0$, such that $B_{\varepsilon}(z) \subset B_{\delta}(y)$, and $B_{\varepsilon}(z) \cap H = \varnothing$. Thus, $z \not\in \cl(H)$. But $z$ was chosen to be an interior point of $B_{\varepsilon}(y)$, so the interior of $\cl(H)$ does not contain any open set. 

\subsubsection*{\underline{$=>$}}

Pick $x \in U$ and $\varepsilon > 0$ so that $B_{\varepsilon}(x) \subset U$. By hypothesis, $\operatorname{int}(\cl(H)) = \varnothing$. So $B_\varepsilon(x)$ is not contained in $\cl(H)$, and so $z \in \underbrace{(X \setminus \cl(H))}_{\text{open}} \cap \overbrace{B_{\varepsilon}(x)}^{\text{open}}$. So this set is open, and so $z$ admits a ball which is entirely contained in that set. 

\qed

We are now ready to prove Baire's Theorem. 

\proof of Baire's Theorem

We will make use of Cantor's Theorem.

Pick a nonempty open $U \subseteq X$. By the lemma, there exists a ball $B_{r_1}(x_1) \subseteq U$, such that $B_{r_1}(x) \cap H_1 = \varnothing$. 

Let $V_1 \eqdef \{x \mid d(x, x_1) \leq \frac{r_1}{2}\}$. This is a closed subset of $B_{r_1}(x)$. Now, $B_{\frac{r_1}{3}}(x_1)$ is open. So, by the lemma, there exists a $B_{r_2}(x_2) \subset B_{\frac{r_1}{3}}(x_1)$ which is disjoint from $H_2$. Let $V_2 = \{x \mid d(x, x_2) \leq \frac{r_2}{2^2}\}$. 

Continuing in this manner, we have a decreasing collection $V_1 \supset V_2 \supset \cdots $, with $\operatorname{diam}V_i \to 0$ as $n\to\oo$, so their intersection contains a point. But by construction $V_k \cap H_k = \varnothing$ for all $k$, so this point does not intersect any of the $H_k$. This means the point is in $X \setminus \cup_{k\geq1}H_k$. $U$ was arbitrary, so we have shown that any open set intersects $X \setminus \cup_{k\geq1}H_k$, and so this set is dense. 

\qed

\section*{Lecture 19}

\subsection*{\underline{Fundamental Group}}

Suppose $X$ is a topological space. We want a way to associate with $X$ some algebraic object, $G(X)$. We want to do this so that if $f:X\to Y$ is continuous, it ``descends" to some homomorphism $f_\star:G(X)\to G(Y)$. Further, in analogy with the fact that the composition of continuous maps is continuous, we want the commutative diagram 
\[
\begin{tikzcd}
X \ar[rd, "g \circ f"'] \ar[r, "f"] & Y \ar[d, "g"] \\
& Z \\
\end{tikzcd}
\]

to become

\[
\begin{tikzcd}
G(X) \ar[rd, "(g \circ f)_\star"'] \ar[r, "f_\star"] & G(Y) \ar[d, "g_\star"] \\
& G(Z)
\end{tikzcd}
\]

We also want the identity homeomorphism on $X$ to be sent to the identity isomorphism on $G(X)$. 

\thm 

Suppose we have some way of doing the above. Then if $X$ is homeomorphic to $Y$, then $G(X)$ is isomorphic to $G(Y)$. 

\proof

Suppose $f:X\to Y$, $g:Y\to X$ are inverse homeomorphisms. Then the commutative diagram
\[
\begin{tikzcd}
X \ar[rd, "\Id"'] \ar[r, "f"] & Y \ar[d, "g"] \\
& X \\
\end{tikzcd}
\]

becomes
\[
\begin{tikzcd}
G(X) \ar[rd, "\Id"'] \ar[r, "f_\star"] & G(Y) \ar[d, "g_\star"] \\
& G(X)
\end{tikzcd}
\]
But this means $g_\star f_\star = \Id_{G(X)}$. A similar argument gives $f_\star g_\star = \Id_{G(Y)}$. 

\qed

Why might we want to do this? The above theorem shows that we can use such a group as a way of distinguishing spaces. If we can show $G(X)\not\simeq G(Y)$, then we may assume $X\not\cong Y$. 

We will begin this construction by defining a weaker equivalence relation than homeomorphism, which will preserve the fundamental group (which we will later construct). 

For the rest of time, $X, Y$ will be topological spaces, and $f, g$ continuous. 

\defn

Consider $f, g:X\to Y$. We say that \underline{$f$ is homotopic to $g$}, which we write as $f \simeq g$, if there exists a continuous $F:X\times\underbrace{[0, 1]}_{``I"}\to Y$, such that for all $x \in X$
\[
F(x, t) = \begin{cases} f(x) & t = 0 \\ g(x) & t = 1 \\ \end{cases}
\]
We sometimes write $F(\cdot, t)$ as $f_t$ or $F_t$, depending on convience. 

\lem

Homotopy, as defined above, is an equivalence relation between elements of $\Hom(X, Y)$. 

\proof

It is trivial to verify that $f \simeq f$. 

Next, suppose $f \simeq g$. If $F:X\times [0,1]\to Y$ is a homotopy between $f, g$, then $G:X\times[0, 1]\to Y$ given by $G(x, t) = F(x, 1 - t)$ is a homotopy which shows $g \simeq f$. 

Finally, suppose $f \simeq g$ and $g \simeq h$. Then there are homotopies $F_t, G_t$. We can define $H_t$ by 
\[
H_t(x) = \begin{cases} F_{2t}(x) & t \in [0, \frac{1}{2}] \\ G_{2t - 1}(x) & t \in [\frac{1}{2}, 1] \end{cases}
\]
We have to check this is continuous at $t = \frac{1}{2}$. Indeed, $F_{2\frac{1}{2}}(x) = G_{2\frac{1}{2} - 1}(x)$, so $H$ is continuous. 

\qed

\exm

\begin{enumerate}
\item $f, g:X\to Y\subseteq \R^n$ with $Y$ convex, then $F(x, t) = (1 - t)f(x) + tg(x)$ is a homotopy between $f, g$, so $f \simeq g$. We are using the convexity hypothesis when we assert that $(1 - t)f(x) + tg(x) \in Y$. 
\item $f, g:X\to S^{n - 1}\subset \R^n$. If $f(x) \neq -g(x)$ for all $x, \in x$, then $f\simeq g$. How can we see this? We will prove it in the special case of $S^2 \subset \R^3$. Consider $f(x), g(x) \in S^2$. There is a line going between these two points, and this line misses the origin. We can normalize every point on this line, and we get a path between $f(x)$ and $g(x)$. Explicitly, 
\[
F(x, t) = \frac{(1 - t)f(x) + tg(x)}{\norm{(1 - t)f(x) + tg(x)}}
\]
Note this is a sufficient but not necessary condition. 
\end{enumerate}





















\end{document}