\documentclass[x11names,reqno,14pt]{extarticle}
% Choomno Moos
% Portland State University
% Choom@pdx.edu


%% stupid experiment %%
%%%%%%%%%%%%% PACKAGES %%%%%%%%%%%%%

%%%% SYMBOLS AND MATH %%%%
\let\oldvec\vec
\usepackage{authblk}	% author block customization
\usepackage{microtype}	% makes stuff look real nice
\usepackage{amssymb} 	% math symbols
\usepackage{siunitx} 	% for SI units, and the degree symbol
\usepackage{mathrsfs}	% provides script fonts like mathscr
\usepackage{mathtools}	% extension to amsmath, also loads amsmath
\usepackage{esint}		% extended set of integrals
\mathtoolsset{showonlyrefs} % equation numbers only shown when referenced
\usepackage{amsthm}		% theorem environments
\usepackage{relsize}	%font size commands
\usepackage{bm}			% provides bold math
\usepackage{bbm}		% for blackboard bold 1

%%%% FIGURES %%%%
\usepackage{graphicx} % for including pictures
\usepackage{float} % allows [H] option on figures, so that they appear where they are typed in code
\usepackage{caption}
\usepackage{hyperref}
%\usepackage{titling}
\usepackage{tikz} % for drawing
\usetikzlibrary{shapes,arrows,chains,positioning,cd,decorations.pathreplacing,decorations.markings,hobby,knots,braids}
\usepackage{subcaption}	% subfigure environment in figures

%%%% MISC %%%%
\usepackage{enumitem} % for lists and itemizations
\setlist[enumerate]{leftmargin=*,label=\bf \arabic*.}

\usepackage{multicol}
\usepackage{multirow}
\usepackage{url}
\usepackage[symbol]{footmisc}
\renewcommand{\thefootnote}{\fnsymbol{footnote}}
\usepackage{lastpage} % provides the total number of pages for the "X of LastPage" page numbering
\usepackage{fancyhdr}
\usepackage{manfnt}
\usepackage{nicefrac}
%\usepackage{fontspec}
%\usepackage{polyglossia}
%\setmainlanguage{english}
%\setotherlanguages{khmer}
%\newfontfamily\khmerfont[Script=Khmer]{Khmer Busra}

%%% Khmer script commands for math %%%
%\newcommand{\ka}{\text{\textkhmer{ក}}}
%\newcommand{\ko}{\text{\textkhmer{ត}}}
%\newcommand{\kha}{\text{\textkhmer{ខ}}}

%\usepackage[
%backend=biber,
% numeric
%style=numeric,
% APA
%bibstyle=apa,
%citestyle=authoryear,
%]{biblatex}

\usepackage[explicit]{titlesec}
%%%%%%%% SOME CODE FOR REDECLARING %%%%%%%%%%

\makeatletter
\newcommand\RedeclareMathOperator{%
	\@ifstar{\def\rmo@s{m}\rmo@redeclare}{\def\rmo@s{o}\rmo@redeclare}%
}
% this is taken from \renew@command
\newcommand\rmo@redeclare[2]{%
	\begingroup \escapechar\m@ne\xdef\@gtempa{{\string#1}}\endgroup
	\expandafter\@ifundefined\@gtempa
	{\@latex@error{\noexpand#1undefined}\@ehc}%
	\relax
	\expandafter\rmo@declmathop\rmo@s{#1}{#2}}
% This is just \@declmathop without \@ifdefinable
\newcommand\rmo@declmathop[3]{%
	\DeclareRobustCommand{#2}{\qopname\newmcodes@#1{#3}}%
}
\@onlypreamble\RedeclareMathOperator
\makeatother

\makeatletter
\newcommand*{\relrelbarsep}{.386ex}
\newcommand*{\relrelbar}{%
	\mathrel{%
		\mathpalette\@relrelbar\relrelbarsep
	}%
}
\newcommand*{\@relrelbar}[2]{%
	\raise#2\hbox to 0pt{$\m@th#1\relbar$\hss}%
	\lower#2\hbox{$\m@th#1\relbar$}%
}
\providecommand*{\rightrightarrowsfill@}{%
	\arrowfill@\relrelbar\relrelbar\rightrightarrows
}
\providecommand*{\leftleftarrowsfill@}{%
	\arrowfill@\leftleftarrows\relrelbar\relrelbar
}
\providecommand*{\xrightrightarrows}[2][]{%
	\ext@arrow 0359\rightrightarrowsfill@{#1}{#2}%
}
\providecommand*{\xleftleftarrows}[2][]{%
	\ext@arrow 3095\leftleftarrowsfill@{#1}{#2}%
}
\makeatother

%%%%%%%% NEW COMMANDS %%%%%%%%%%

% settings
\newcommand{\N}{\mathbb{N}}                     	% Natural numbers
\newcommand{\Z}{\mathbb{Z}}                     	% Integers
\newcommand{\Q}{\mathbb{Q}}                     	% Rationals
\newcommand{\R}{\mathbb{R}}                     	% Reals
\newcommand{\C}{\mathbb{C}}                     	% Complex numbers
\newcommand{\K}{\mathbb{K}}							% Scalars
\newcommand{\F}{\mathbb{F}}                     	% Arbitrary Field
\newcommand{\E}{\mathbb{E}}                     	% Euclidean topological space
\renewcommand{\H}{{\mathbb{H}}}                   	% Quaternions / Half space
\newcommand{\RP}{{\mathbb{RP}}}                       % Real projective space
\newcommand{\CP}{{\mathbb{CP}}}                       % Complex projective space
\newcommand{\Mat}{{\mathrm{Mat}}}						% Matrix ring
\newcommand{\M}{\mathcal{M}}
\newcommand{\GL}{{\mathrm{GL}}}
\newcommand{\SL}{{\mathrm{SL}}}

\newcommand{\tgl}{\mathfrak{gl}}
\newcommand{\tsl}{\mathfrak{sl}}                  % Lie algebras; i.e., tangent space of SO/SL/SU
\newcommand{\tso}{\mathfrak{so}}
\newcommand{\tsu}{\mathfrak{sl}}


% typography
\newcommand{\noi}{\noindent}						% Removes indent
\newcommand{\tbf}[1]{\textbf{#1}}					% Boldface
\newcommand{\mc}[1]{\mathcal{#1}}               	% Calligraphic
\newcommand{\ms}[1]{\mathscr{#1}}               	% Script
\newcommand{\mbb}[1]{\mathbb{#1}}               	% Blackboard bold


% (in)equalities
\newcommand{\eqdef}{\overset{\mathrm{def}}{=}}		% Definition equals
\newcommand{\sub}{\subseteq}						% Changes default symbol from proper to improper
\newcommand{\psub}{\subset}						% Preferred proper subset symbol

% Categories
\newcommand{\catname}[1]{{\text{\sffamily {#1}}}}

\newcommand{\Cat}{{\catname{C}}}
\newcommand{\cat}[1]{{\catname{\ifblank{#1}{C}{#1}}}}
\newcommand{\CAT}{{\catname{Cat}}}
\newcommand{\Set}{{\catname{Set}}}

\newcommand{\Top}{{\catname{Top}}}
\newcommand{\Met}{{\catname{Met}}}
\newcommand{\PL}{{\catname{PL}}}
\newcommand{\Man}{{\catname{Man}}}
\newcommand{\Diff}{{\catname{Diff}}}

\newcommand{\Grp}{{\catname{Grp}}}
\newcommand{\Grpd}{{\catname{Grpd}}}
\newcommand{\Ab}{{\catname{Ab}}}
\newcommand{\Ring}{{\catname{Ring}}}
\newcommand{\CRing}{{\catname{CRing}}}
\newcommand{\Mod}{{\mhyphen\catname{Mod}}}
\newcommand{\Alg}{{\mhyphen\catname{Alg}}}
\newcommand{\Field}{{\catname{Field}}}
\newcommand{\Vect}{{\catname{Vect}}}
\newcommand{\Hilb}{{\catname{Hilb}}}
\newcommand{\Ch}{{\catname{Ch}}}

\newcommand{\Hom}{{\mathrm{Hom}}}
\newcommand{\End}{{\mathrm{End}}}
\newcommand{\Aut}{{\mathrm{Aut}}}
\newcommand{\Obj}{{\mathrm{Obj}}}
\newcommand{\op}{{\mathrm{op}}}

% Norms, inner products
\delimitershortfall=-1sp
\newcommand{\widecdot}{\, \cdot \,}
\newcommand\emptyarg{{}\cdot{}}
\DeclarePairedDelimiterX{\norm}[1]{\Vert}{\Vert}{\ifblank{#1}{\emptyarg}{#1}}
\DeclarePairedDelimiterX{\abs}[1]\vert\vert{\ifblank{#1}{\emptyarg}{#1}}
\DeclarePairedDelimiterX\inn[1]\langle\rangle{\ifblank{#1}{\emptyarg,\emptyarg}{#1}}
\DeclarePairedDelimiterX\cur[1]\{\}{\ifblank{#1}{\emptyarg,\emptyarg}{#1}}
\DeclarePairedDelimiterX\pa[1](){\ifblank{#1}{\emptyarg}{#1}}
\DeclarePairedDelimiterX\brak[1][]{\ifblank{#1}{\emptyarg}{#1}}
\DeclarePairedDelimiterX{\an}[1]\langle\rangle{\ifblank{#1}{\emptyarg}{#1}}
\DeclarePairedDelimiterX{\bra}[1]\langle\vert{\ifblank{#1}{\emptyarg}{#1}}
\DeclarePairedDelimiterX{\ket}[1]\vert\rangle{\ifblank{#1}{\emptyarg}{#1}}

% mathmode text operators
\RedeclareMathOperator{\Re}{\operatorname{Re}}		% Real part
\RedeclareMathOperator{\Im}{\operatorname{Im}}		% Imaginary part
\DeclareMathOperator{\Stab}{\mathrm{Stab}}
\DeclareMathOperator{\Orb}{\mathrm{Orb}}
\DeclareMathOperator{\Id}{\mathrm{Id}}
\DeclareMathOperator{\vspan}{\mathrm{span}}			% Vector span
\DeclareMathOperator{\tr}{\mathrm{tr}}
\DeclareMathOperator{\adj}{\mathrm{adj}}
\DeclareMathOperator{\diag}{\mathrm{diag}}
\DeclareMathOperator{\eq}{\mathrm{eq}}
\DeclareMathOperator{\coeq}{\mathrm{coeq}}
\DeclareMathOperator{\coker}{\mathrm{coker}}
\DeclareMathOperator{\dom}{\mathrm{dom}}
\DeclareMathOperator{\cod}{\mathrm{codom}}
\DeclareMathOperator{\im}{\mathrm{im}}
\DeclareMathOperator{\Dim}{\mathrm{dim}}
\DeclareMathOperator{\codim}{\mathrm{codim}}
\DeclareMathOperator{\Sym}{\mathrm{Sym}}
\DeclareMathOperator{\lcm}{\mathrm{lcm}}
\DeclareMathOperator{\Inn}{\mathrm{Inn}}
\DeclareMathOperator{\sgn}{sgn}						% sgn operator
\DeclareMathOperator{\intr}{\text{int}}             % Interior
\DeclareMathOperator{\co}{\mathrm{co}}				% dual/convex Hull
\DeclareMathOperator{\Ann}{\mathrm{Ann}}
\DeclareMathOperator{\Tor}{\mathrm{Tor}}


% misc symbols
\newcommand{\divides}{\big\lvert}
\newcommand{\grad}{\nabla}
\newcommand{\veps}{\varepsilon}						% Preferred epsilon
\newcommand{\vphi}{\varphi}
\newcommand{\del}{\partial}							% Differential/Boundary
\renewcommand{\emptyset}{\text{\O}}					% Traditional emptyset symbol
\newcommand{\tril}{\triangleleft}					% Quandle operation
\newcommand{\nabt}{\widetilde{\nabla}}				% Contravariant derivative
\newcommand{\later}{$\textcolor{red}{\blacksquare}$}% Laziness indicator

% misc
\mathchardef\mhyphen="2D							% mathomode hyphen
\renewcommand{\mod}[1]{\ (\mathrm{mod}\ #1)}
\renewcommand{\bar}[1]{\overline{#1}}				% Closure/conjugate
\renewcommand\qedsymbol{$\blacksquare$} 			% Changes default qed in proof environment
%%%%% raised chi
\DeclareRobustCommand{\rchi}{{\mathpalette\irchi\relax}}
\newcommand{\irchi}[2]{\raisebox{\depth}{$#1\chi$}}
\newcommand\concat{+\kern-1.3ex+\kern0.8ex}

% Arrows
\newcommand{\weak}{\rightharpoonup}					% Weak convergence
\newcommand{\weakstar}{\overset{*}{\rightharpoonup}}% Weak-star convergence
\newcommand{\inclusion}{\hookrightarrow}			% Inclusion/injective map
\renewcommand{\natural}{\twoheadrightarrow}				% Natural map

% Environments
\theoremstyle{plain}
\newtheorem{thm}{Theorem}[section]
%\newtheorem{lem}[thm]{Lemma}
\newtheorem{lem}{Lemma}
\newtheorem*{lems}{Lemma}
\newtheorem{cor}[thm]{Corollary}
\newtheorem{prop}{Proposition}
\newtheorem*{claim}{Claim}
\newtheorem*{cors}{Corollary}
\newtheorem*{props}{Proposition}
\newtheorem*{conj}{Conjecture}

\theoremstyle{definition}
\newtheorem{defn}{Definition}[section]
\newtheorem*{defns}{Definition}
\newtheorem{exm}{Example}[section]
\newtheorem{exer}{Exercise}[section]

\theoremstyle{remark}
\newtheorem*{rem}{Remark}

\newtheorem*{solnx}{Solution}
\newenvironment{soln}
    {\pushQED{\qed}\renewcommand{\qedsymbol}{$\Diamond$}\solnx}
    {\popQED\endsolnx}%

% Macros
\newcommand{\restr}[1]{_{\mkern 1mu \vrule height 2ex\mkern2mu #1}}
\newcommand{\Upushout}[5]{
    \begin{tikzcd}[ampersand replacement = \&]
    \&#2\ar[rd,"\iota_{#2}"]\ar[rrd,bend left,"f"]\&\&\\
    #1\ar[ur,"#4"]\ar[dr,"#5"]\&\&#2\oplus_{#1} #3\ar[r,dashed,"\vphi"]\&Z\\
    \&#3\ar[ur,"\iota_{#3}"']\ar[rru,bend right,"g"']\&\&
    \end{tikzcd}
}
\newcommand{\exactshort}[5]{
		\begin{tikzcd}[ampersand replacement = \&]
			0\ar[r]\&#1\ar[r,"#2"]\& #3 \ar[r,"#4"]\& #5 \ar[r]\&0
		\end{tikzcd}
}
\newcommand{\product}[6]{
		\begin{tikzcd}[ampersand replacement = \&]
			#1 \& #2 \ar[l,"#4"'] \\
			#3 \ar[u,"#5"] \ar[ur,"#6"']
		\end{tikzcd}
}
\newcommand{\coproduct}[6]{
		\begin{tikzcd}[ampersand replacement = \&]
			#1 \ar[r,"#4"] \ar[d,"#5"'] \& #2 \ar[dl,"#6"] \\
			#3
		\end{tikzcd}
}
%%%%%%%%%%%% PAGE FORMATTING %%%%%%%%%

\usepackage{geometry}
    \geometry{
		left=15mm,
		right=15mm,
		top=15mm,
		bottom=15mm	
		}

\usepackage{color} % to do: change to xcolor
\usepackage{listings}
\lstset{
    basicstyle=\ttfamily,columns=fullflexible,keepspaces=true
}
\usepackage{setspace}
\usepackage{setspace}
\usepackage{mdframed}
\usepackage{booktabs}
\usepackage[document]{ragged2e}
\usepackage{amsmath}
\pagestyle{fancy}{
	\fancyhead[L]{Fall 2022}
	\fancyhead[C]{201A - Real Analysis}
	\fancyhead[R]{John White}
  
  \fancyfoot[R]{\footnotesize Page \thepage \ of \pageref{LastPage}}
	\fancyfoot[C]{}
	}
\fancypagestyle{firststyle}{
     \fancyhead[L]{}
     \fancyhead[R]{}
     \fancyhead[C]{}
     \renewcommand{\headrulewidth}{0pt}
	\fancyfoot[R]{\footnotesize Page \thepage \ of \pageref{LastPage}}
}
\newcommand*{\oo}{\infty}
\title{220A - Groups}
\author{John White}
\date{Fall 2022}




\begin{document}

\section*{Lecture 1}

\subsection*{Why measure theory?}

We want to answer questions like the following: what is the ``total length" of an arbitrary $E \subseteq \R$? What about the ``total area" of an arbitrary $E \subseteq \R^2$? 

In other words, can we define a function $\mu:2^{\R^d}\to[0, + \infty)$ so that $\mu(E)$ is sufficiently ``nice?"

What properties would we like a function $\mu$ (called a measure) to have? Let's stick to $\R$ for now.

\begin{enumerate}
\item For $E = [a, b]$ (or $(a, b)$), we'd like $\mu(E) = b - a$.
\item For a sequence of disjoint intervals $I_i\subseteq \R$, 
\[
\mu\left(\bigcup_{i=1}^nI_i\right) = \sum_{i=1}^n\mu(I_i)
\]
\end{enumerate}

What about $\Q \cap [0, 1]$? What about the area under a curve?

What if $E$ is any arbitrary set????

\subsection*{Pre measure theory}

In the mid 1800s, Riemann first defined the Riemann integral in terms of upper and lower sums. 

Fortunately, it's good enough for most ``ordinary" functions a student might encounter when doing calculus for the first time. 

Unfortunately, it's not good for taking limits. 

For example, given $f_1, f_2, f_3, \cdots :[a, b]\to\R$ such that $\lim_{n\to\infty}f_n(x) =: f(x)$ exists for all $x$, when can we conclude that
\[
\lim_{n\to\infty}\int_a^bf_n(x)\,dx = \int_a^b\lim_{n\to\infty}f_n(x)]\,dx = \int_a^bf(x)\,dx
\]
We learn in undergraduate real analysis that we may only conclude the above if the $f_n$ converge uniformly to $f$. 

\subsection*{Measure theory}

Measure theory allows us to define a much more powerful theory of integration, giving us 
\begin{itemize}
\item More integrable functions
\item An integral which behaves better with limits
\item An integral ideally suited for probability theory. 
\end{itemize}

Our first goal will be to define a function $\mu:2^\R\to[0,\infty)$ satisfying the following:

\begin{enumerate}
\item If $E_1, E_2, \dots$ is a countable sequence of disjoint sets, then 
\[
\mu\left(\bigcup E_i \right) = \sum\mu(E_i)
\]
If $\mu$ satisfies this, we say it is ``countably additive."

\item $\mu([a, b]) = b - a$ for all such intervals. 

\item $\mu$ is translation invariant, i.e. for any $t \in \R$, 
\[
\mu(E + t) = \mu(E)
\]
Where ``$E + t$" $:=\{x + t \mid x \in E\}$
\end{enumerate}

\thm{(Vitali)} There is no such $\mu$.



\proof

Suppose that such a $\mu$ exists. 

\claim

If $A \subseteq B$, then $\mu(A) \leq \mu(B)$.

\proof

Note $B = A \coprod (B \backslash A)$, so 
\[
\mu(B) = \mu(A) + \mu(B \backslash A)
\]

And because $\mu$ is always nonnegative, we may conclude that $\mu(B) \geq \mu(A)$. 

\qed

Now, define an equivalence relation on $\R$ as follows:
\begin{align*}
x \sim y <=> x - y \in \Q \\
[x] := \{y \in \R \mid x \sim y\}
\end{align*}

\claim

Every equivalence class contains a point in $[0, 1]$. 

\proof

Homework exercise. 

\qed 

Now, for each equivalence class, choose an element in $[0, 1]$ belonging to that class. For this step, we are using choice and, it turns out, there is no way not to in this proof. 

Call the resulting set $A$. So $A \subseteq [0, 1]$, and for any $x$, $[x] \cap A$ is a singleton. 

Let $B = \bigcup_{q\in\Q\cap[-1,1]}A + q$

Note that this is a disjoint union: indeed, if $A + q$ intersects nontrivially with $A + q'$ for $q \neq q',$ then there are $x, x'$ in $A$ such that $x = x' + q$, and so $x \sim x'$, which by construction is impossible. 

\claim 
$[0, 1]\subseteq B \subseteq[-1, 2]$. 

\proof

First, if $x \in [0, 1]$, then $x = a + q$ for some $a \in A\subseteq[0, 1], q = x - a \in [-1, 1]$. Thus, $x \in B$. 

Next, if $b \in B$, then $b = a + q$, for $q \in A = [0, 1]$ and $q \in [-1, 1]$, so $b \in [-1, 2]$. 

So we must conclude by the lemma that 
\[
1 = \mu([1, 0]) \leq \mu(B) \leq \mu([-1, 2]) = 3
\] 

But by the properties of $\mu$, we also have
\[
\mu(B) = \sum_{q\in\Q\cap[0, 1]}\mu(A + q) = \sum_{q\in\Q\cap[0,1]}\mu(A)
\]
The sum on the right hand side is either $0$ or $\infty$. But we just showed that it is between $1$ and $3$, a contradiction. Therefore, $\mu(A)$ cannot be defined. 

\qed

So, if this is impossible, which criterion should we weaken to make it possible? 

If we weaken the first to get finite additivity, we run into problems for $d \geq 3$, for example the Banach-Tarski paradox. 

If we weaken the other two, then $\mu$ is no longer compactible with the usual notion of ``length." 

\subsection*{Two good choices}
\begin{itemize}
\item Given a measure on a family of sets, it extends to an outer measure on all sets. 
\item Similarly, given an outer measure, you can single out ``nice sets" on which it is a measure. 
\end{itemize}

What kind of family of subsets should we restrict to? 

Let $X$ be a set. 

\defn $\mathcal{A}$ is an \underline{algebra of subsets of $X$} if $\mathcal{A} \neq \varnothing$, and 
\begin{itemize}
\item 
\[
E_1, \dots, E_n \in \mathcal{U} \implies \bigcup_{i=1}^nE_i \in \mathcal{U}
\]
In other words, it is ``closed under finite unions." 
\item
\[
E \in \mathcal{U} \implies X \backslash E \in \mathcal{U}
\]
In other words it is ``closed under compliments."
\item 
\[
\varnothing, X \in \mathcal{A}
\]
\end{itemize}

\lem 

If $\mathcal{A}$ is an algebra of subsets, then $\mathcal{A}$ is closed under finite intersections.

\proof 

Homework 2

\exm

\begin{itemize}
\item[(i)] $\mathcal{A} = 2^X$
\item[(ii)] $\mathcal{A} = \{\varnothing, X\}$
\item[(iii)] $\mathcal{A} =$ all finite OR cofinite subsets of $X$ (cofinite means the complement is finite). 
\end{itemize}

\defn A \underline{$\sigma$-algebra} $\mathcal{A}$ is an algebra that is closed under countable unions. 

\rem

$\sigma$-algebras are closed under countable intersections. 

\exm

Above, $(i)$ and $(ii)$ are $\sigma$-algebras, but $(iii)$ is not. 

\prop

Given any family $\mathcal{E}$ of subsets of $X$, there is a smallest $\sigma$-algebra $\mu(\mathcal{E})$ containing $\mathcal{E}$, meaning that if $\mathcal{F}$ is a $\sigma$-algebra containing $\mathcal{E}$, then $\mu(\mathcal{E}) \subseteq \mathcal{F}$. 


\section*{Lecture 2}


\defn
Given a nonempty set $X$ and $\mathcal{M}$ a $\sigma$-algebra of subsets of $X$, we call $(X, \mathcal{M})$ a \underline{measurable space}.

Recall: 

\prop

Given any family $\mathcal{E}$ of subsets of $X$, there is a smallest $\sigma$-algebra $\mathcal{M}(\mathcal{E})$ containing $\mathcal{E}$, meaning that if $\mathcal{F}$ is a $\sigma$-algebra containing $\mathcal{E}$, then $\mathcal{M}(\mathcal{E}) \subseteq \mathcal{F}$. 

\proof

We begin with a claim. 

\claim Given any nonempty collection $\mathcal{C}$ of $\sigma$-algebras on $X$, then 
\[
\cap \mathcal{C} := \{E \subseteq X \mid E \in \mathcal{A} \forall \mathcal{A} \in \mathcal{C}\}
\]
is a $\sigma$-algebra. 

\proof Homework 2

\qed

Let $\mathcal{C} = \{\mathcal{A} \mid \mathcal{A}$ is a $\sigma$-algebra on $X$ and $\mathcal{E} \subseteq \mathcal{A}\}$. $\mathcal{C}$ is nonempty, because $2^X \in \mathcal{C}$. By the claim, $\cap\mathcal{C}$ is a $\sigma$-algebra. By the definition of $\mathcal{C}$, $\mathcal{E}\subseteq\mathcal{C}$ and for any $\sigma$-algebra $\mathcal{A}$ such that $\mathcal{E} \subseteq \mathcal{A},$ $\cap\mathcal{C} \subseteq \mathcal{A}$.

Thus $\mathcal{M}(\mathcal{E}) = \cap\mathcal{C}$ is the smallest $\sigma$-algebra containing $\mathcal{E}$.

\qed 

\rem Intuitively, $\mathcal{M}(\mathcal{E})$ is a $\sigma$-algebra containing the sets in $\mathcal{E}$ by ``going from the outside in," starting with $\sigma$-algebras that are ``too big" and taking intersections. 

Recall: a topology $\tau$ is a collection of subsets of a set $X$ (called open sets), which is closed under arbitrary unions and finite intersections, and $X, \varnothing \in \tau$

Let $(X, \tau)$ be a topoological space. 

\defn \underline{The Borel $\sigma$-algebra} of $X$, denoted $\mathcal{B}X$, is the $\sigma$-algebra generated by the open subsets of $X$. Its members are known as \underline{Borel sets}. 

What do the Borel sets look like? Let's go from the ``inside out." 

Let $\mathcal{F} = $open sets in $X$, $\mathcal{F}^\sigma$ all countable unions of sets in $\mathcal{F}$, $\mathcal{F}^\delta$ all countable intersections, and $\bar{\mathcal{F}}$ complements of sets in $\mathcal{F}$. 

To build Borel sets: 

\[
\mathcal{F}\to\mathcal{F}^\delta \cup \overline{\mathcal{F}^\delta} \to\cdots \to \mathcal{B}X
\]

To learn more, look up the ``Borel hierarchy." 

\prop The \underline{Borel $\sigma$-algebra} on $\R$, which we denote $\mathcal{B}_\R$, is generated by each of the following. 
\begin{itemize}
\item[(i)] Open intervals $\mathcal{E}_1 = \{(a, b) \mid a < b, a, b \in \R\}$
\item[(ii)] Closed intervals $\mathcal{E}_2 = \{[a, b] \mid a \leq b, a,b , \in \R\}$
\item[(iii)] Half-open intervals $\mathcal{E}_3 = \{[a, b) \mid a < b, a, b, \in \R\}$
\item[(iv)] Open rays $\mathcal{E}_4 = \{(a, \infty) \mid a \in \R\}$
\item[(v)] Closed rays $\mathcal{E}_5 = \{[a, \infty) \mid a \in \R\}$
\end{itemize}

That is, $\mathcal{M}(\mathcal{E}_i) = \mathcal{B}_\R$ for any $i \in \{1, \dots, 5\}$.

\proof 

Homework 2 

\qed

Let 
\begin{itemize}
\item $\{(X_1, \mathcal{M}_i)\}_{i=1}^\infty$ be a collection of measurable spaces. 
\[
X := \prod_{i=1}^\infty X_i
\]
\item $\pi_i$ be the projection $X \to X_i$
\end{itemize}

\exm If $(X_i, \mathcal{M}_i) = (\R, \mathcal{B}_\R)$, for $i \in \{1, \dots, n\}$. Then $X = \R^n$.


\defn The \underline{product $\sigma$-algebra}
\[
\bigotimes_{i\in\N} \mathcal{M}_i := \mathcal{M}\left(\{\prod_{i\in\N} E_i \mid E_i \in \mathcal{M}_i\}\right)
\]

Our goal is to show that $\mathcal{B}_{\R^n} = \otimes_{i=1}^n\mathcal{B}_\R$. 

\prop Given $\mathcal{E}_i \subseteq 2^{X_i}$ such that $X_i \in \mathcal{E}_i$, let $\mathcal{M}_i = \mathcal{M}(\mathcal{E}_i)$. Then
\[
\bigotimes_{i\in\N} \mathcal{M}_i = \mathcal{M}\left(\{\prod_{i\in\N}E_i \mid E_i \in \mathcal{E}_i\}\right)
\] 

Note: If $\mathcal{E}\subseteq \mathcal{F},$ then $\mathcal{M}(\mathcal{E}) \subseteq \mathcal{M}(\mathcal{F})$. 
If $\mathcal{E} \subseteq \mathcal{M}(\mathcal{F}),$ then $\mathcal{M}(\mathcal{E}) \subseteq \mathcal{M}\mathcal{F}$.

Recall: Given a function $f:X\to Y$ between arbitrary nonempty sets, then 
\begin{itemize}
\item[(i)] $f^{-1}\left(\cup_{i\in\N}E_i\right) = \cup_{i\in\N}f^{-1}(E_i)$ for all $E_i \subseteq X$. 
\item[(ii)] $f^{-1}(E^c) = (f^{-1}(C))^c$, for all $E \subseteq X$. 
\end{itemize}

\proof By the first statement of the preceding note, 
\[
\mathcal{M}\left(\{\prod_{i\in\N}E_i \mid E_i \in \mathcal{E}_i\}\right) \subseteq \mathcal{M}\left(\{\prod_{i\in\N}E_i \mid E_i \in \mathcal{M}_i\}\right) = \bigotimes_{i\in\N}\mathcal{M}_i
\]
For equality, it suffices to show
\[
\mathcal{M}\left(\{\prod_{i\in\N}E_i \mid E_i \in \mathcal{M}_i\}\right) \subseteq \mathcal{M}\left(\{\prod_{i\in\N}E_i \mid E_i \in \mathcal{E}_i\}\right)
\]

Let $\mathcal{M}\left(\{\prod_{i\in\N}E_i \mid E_i \in \mathcal{E}_i\}\right) = \mathcal{A}$.

Note that 
\begin{align*}
\prod_{i\in\N}E_i  &= \{x \in X \mid \pi_i(x) \in E_i\forall i\} \\ & = \bigcap_{i\in\N}\{x \in X: \pi_i(x) \in E_i\} \\
 & = \bigcap_{i\in\N}\pi_i^{-1}(E_i) \\
\end{align*}

Because $\mathcal{A}$ is a $\sigma$-algebra, it suffices to show that $\pi_i^{-1}(E_i) \in \mathcal{A}$ for all $E_i \in \mathcal{M}_i$. 

\claim Let $\mathcal{F}_i := \{E_i \subseteq X_i \mid \pi_i^{-1}(E_i) \in \mathcal{A}\}$.

This is a $\sigma$-algebra. 

\proof

\[
\bigcup_{i=1}^\infty \pi_i^{-1}(E_i) = \pi_i^{-1}(\bigcup_{i=1}^\infty E_i)
\]

So $\{E_i\}_{i=1}^\infty \subseteq \mathcal{F}_i$. 

Similarly, 

\[
\pi_i^{-1}(E^c) = (\pi_i^{-1}(E))^c
\]
so $E \in \mathcal{F}_i$ implies $E^c \in \mathcal{F}_i$.

\section*{Lecture 3}

Because $X_i \in \mathcal{E}_i$and $\pi_i^{-1}(E_i) = X_1\times X_2\times \cdots \times E_i \times \cdots $, we know $\pi_i^{-1}(E_i)\in\mathcal{A}$ for all $E_i\in\mathcal{E}_i$. 

In other words, $\mathcal{E}_i\subseteq\mathcal{F}_i$. Since $\mathcal{F}_i$ is a $\sigma$-algebra, $\mathcal{M}(\mathcal{E}_i) = \mathcal{M}_i\subseteq\mathcal{F}_i$. 

Thus, $\pi_i^{-1}(\mathcal{E}_i) \in \mathcal{A}$ for all $E_i \in \mathcal{M}_i$ and all $i$. 

\qed

In order to characterize the Borel product $\sigma$-algebra, it will be convenient to assume our undderlying spaces have a metric that induces the topology. 



Let $(X_i, d_i), i = 1, \dots, n$ be metric spaces. 
Let
\[
X = \prod_{i=1}^nX_i
\]
Endow the product space with the metric
\[
d_{\max}((x_1, x_2, \dots, x_n), (y_1, y_2, \dots, y_n)) = \max_{i=1, \dots, n}(d_1(x_1, y_1), d_2(x_2, y_2), \dots, d_n(x_n, y_n))
\]

\thm

Given metric spaces $X_1, X_2, \dots, X_d$ and their product
\[
X = \prod_{i=1}^dX_i
\]
endowed with the metric $d_{\max}$, then $\otimes_{i=1}^n\mathcal{B}_{X_i} \subseteq \mathcal{B}_X$

If the $X_i$ are all seperable, then $\otimes_{i=1}^n\mathcal{B}_{X_i} = \mathcal{B}_X$

\rem Since the definition of $\mathcal{B}_X$ only depends on the topology of $X$, then this statement holds even if $d_{\max}$ is replaced by an equivalent metric, where ``equivalent" means ``generates the same topology."

\rem $d_{\max}$ is convenient because:
\begin{align*}
B_r(x_1, \dots, x_m) & = \{(y_1, \dots, y_n) | d_{\max}(\vec{x}, \vec{y}) < r\} \\
							 & = \{(y_2, \dots, y_n) \mid d_i(x_i, y_i) < r \forall i\} \\
							 & = \prod_{i=1}^nB_r(X_i) \\
\end{align*}

Recall:

Fact 1: If $X_1, \dots, X_m$ are seperable, so is $\prod_{i=1}^mX_i$. 

Fact 2: In a seperable metric space, every open set can be written as a countable union of balls, $\mathcal{U} = \cup_{i=1}^\infty B_i$

Fact 3: $\{\prod_{i=1}^nE_i \mid E_i \subseteq X_i, \text{open}\} \subseteq \{\text{open subsets of }X\}$

\proof: By the previous proposition, $\otimes_{i=1}^n\mathcal{B}_{X_i}$ is generated by 
\[
\{\prod_{i=1}^nE_i \mid E_i \subseteq X_{\text{open}}\} \subseteq\{\text{open subsets of }X\}
\]

Thus $\otimes_{i=1}^n\mathcal{B}_{X_i} \subseteq \mathcal{B}_X$.

Now, suppose $X_1, \dots, X_n$ are seperable. By facts 1 and 2, every open subset of $X$ can be written as a countable union of balls. 

To prove $\otimes_{i=1}^n\mathcal{B}_{X_i} = \mathcal{B}_X$, it suffices to show that 

\begin{align*}
\{\text{open subsets of }X\} & \subseteq \otimes_{i=1}^n\mathcal{B}_{X_i}\\
\end{align*}

The left hand side is equal to $\{\cup_{j=1}^\infty B_j \mid B_j \subseteq X_{\text{open}}\text{ball}\}$, and the right hand side is equal to $\mathcal{M}\left(\{\prod_{i=1}^nE_i \mid E_i \text{ open} \}\right)$

This will hold, as long as we can show $B_j \in \mathcal{M}\left(\{\prod_{i=1}^nE_i\mid E_i\text{ open}\}\right)$

Since $X$ is endowed with $d_{\max}$, we know that any open ball in $X$ can be expressed as $B = \prod_{i=1}^nB_i$, where $B_i \subseteq X_i$ is a ball. This gives the result. 

\qed

Now, it is finally time to talk about measures. 

\subsection*{\underline{Measures}}

Call $(X, \mathcal{M})$ a measurable space when $X$ is a set and $\mathcal{M}$ is a $\sigma$-algebra on $X$. 

\defn A \underline{measure} on a measurable space $(X, \mathcal{M})$ is a function $\mu:\mathcal{M}\to[0, +\infty]$ such that

\begin{enumerate}
\item[(i)] $\mu(\varnothing) = 0$
\item[(ii)] If $\{E_i\}$ is a countable disjoint collection of sets, then
\[
\mu(\bigcup E_i) = \sum \mu(E_i)
\]
This is called ``countable (disjoint) additivity"
\end{enumerate}

\exm (Dirac mass/Dirac measure)

Let $(X, \mathcal{M}) = (X, 2^X)$. 

Fix $x_0 \in X$ and define 
\[
\mu(A) = \begin{cases} 1& \text{ if }x_0 \in A \\ 0 & \text{ otherwise} \\ \end{cases}
\]

\exm (Counting measure)

Let $(X, \mathcal{M}) = (X, 2^X)$. Define
\[
\mu(A) = |A| = \text{ the number of elements in }A
\]

Given a measurable space $(X, \mathcal{M})$ and a measure $\mu$, we call $(X, \mathcal{M}, \mu)$ a \underline{measure space} and $E \in \mathcal{M}$ a \underline{measurable set} 

\thm For any measure space $(X, \mathcal{M}, \mu)$ and measurable sets $A, B, A_1, A_2, \dots \in \mathcal{M}$, 

\begin{enumerate}
\item[(i)] $A \subseteq B \implies \mu(A)\leq\mu(B)$. This is called ``monotonicity"
\item[(ii)] $\mu(\bigcup_i A_i) \leq \sum_i\mu(A_i)$. This is called ``(countable) sub additivity)."
\item[(iii)] If $A_i\subseteq A_{i + 1}$, then $\mu(\bigcup_i A_i) = \lim_{n\to\infty}\mu(A_i)$. This is called ``continuity from below." 
\item[(iv)] If $A_{i + 1} \subseteq A_i$ for all $i$, and $\mu(A_i) < \infty$, then $\mu(\bigcap A_i) = \lim_{i\to\infty}\mu(A_i)$. This is called ``continuity from above."
\end{enumerate}

\rem For $(iv)$, why do we need the additional hypothesis $\mu(A_1) < \infty$?. 

Consider the counting measure on $(\N, 2^{\N})$, and $A_i = \{n \in \N \mid n \geq i\}$, which satisfies $A_{i + 1} \subseteq A_i$, but it fails $\mu(A_1) < \infty$:
\[
0 = \mu(\varnothing) = \mu(\cap_{i=1}^\infty A_i) \neq \lim_{i\to\infty}\mu(A_i) = + \infty
\]

\proof
\begin{enumerate}
\item[(i)] Since $A \subseteq B$, $B = A \cup (B\backslash A)$, so $\mu(B) = \mu(A) + \mu(B\backslash A)$ by countable additivity. $\mu(B\backslash A) \geq 0$, so $(i)$ follows. 
\item[(ii)] Define $B_1 = A_1$, $B_2 = A_2 \backslash A_1$, $B_3 = A_3 \backslash (A_1 \cup A_2), \dots, B_n = A_n\backslash(\cup_{i=1}^{n - 1}A_i)$.
Then $\cup_i A_i = \cup_i B_i$, so by countable disjoint additivity, 
\[
\mu(\cup_i A_i) = \mu(\cup_i B_i) = \sum_i\mu(B_i) \leq \sum_i\mu(A_i)
\]
\item[(iii)] Define $B_1 = A_1$, and $B_i = A_i \backslash A_{i - 1}$. Then $A_n = \cup_{i=1}^nB_i$, so $\cup_{i=1}^\oo A_i= \cup_{i=1}^\oo B_i$. Thus $\mu(A_n) = \mu(\cup_{i=1}^nB_i) = \sum_{i=1}^n\mu(B_i)$. Consequently, 
\[
\mu(\cup_{n=1}^\oo A_n) = \mu(\cup_{i=1}^\oo B_i) = \sum_{i=1}^\oo\mu(B_i) = \lim_{n\to\oo}\sum_{i=1}^n\mu(B_i) = \lim_{n\to\oo}\mu(A_n)
\]
\item[(iv)] Next time!
\end{enumerate}













 

 





















\end{document}